\newpage

\chapter{Changes between versions}
\label{sec_changes}

Here is a list of changes between the different official OASIS3
versions.

\section{Changes between {\tt oasis3\_3} and {\tt
oasis3\_prism\_2\_5}}

The changes between version {\tt oasis3\_3} and version {\tt
  oasis3\_prism\_2\_5} delivered in September 2006 are the following:

\begin{itemize}

\item Bug corrections:

  \begin{itemize}

  \item In {\tt oasis3/lib/scrip/src/remap\_bilinear.f, remap\_bicubic.f,
      \break remap\_bilinear\_reduced.f, remap\_bicubic\_reduced.F90}:
    (r2084) we observed a wrong behaviour of routines
    remap\_bilinear.f and remap\_bicubic.f on the NEC SX9 when
    compiled with NEC SX compiler revision 400. We got round this
    problem by adding an explicit instruction to prevent the
    vectorisation of one loop.

    As remap\_bilinear\_reduced.f and remap\_bicubic\_reduced.F90 have
    a very similar loop, we introduced the same instruction in these
    routines, although nothing specific was observed with them.
  
  \item {\tt oasis3/lib/scrip/src/remap\_bicubic\_reduced.F90}:
    (r1883) Two bugfixes: 1) Memory fault when a target point was falling on the
    before-last latitude circle or in the before-last latitude band of
    the source reduced grid; 2) Wrong neighbours for target points
    south of the before-last latitude circle (i.e in the last latitude
    band or Southern).
 
  \item {\tt oasis3/lib/psmile/src/mod\_psmile\_io.F90}: (r2380)
    correction to ensure that when {\tt INVERT} is used, the corner
    latitudes and longitudes are also inverted (and not only the
    center latitudes and longitudes as before).

  \item {\tt oasis3/lib/scrip/src/scriprmp.F}: (r1547) the calculation
    of the average for the line of points at the pole was bugged and
    did not have any effect. It is now debugged but commented.

  \item In {\tt oasis3/lib/scrip/src/vector.F90}: 
    
    - correction of
    wrong sequences in declarations, at least for Intel \& NAG
    compilers (thanks to L. Kornblueh from MPI); bug fix announced to
    the mailing list diff-oasis@cerfacs.fr on 31/10/2006.
    
    - (r1698 - 2008-08-20) bugfix to
    make sure that OASIS does not automatically calculates corners of
    target grid as this calculation is correct only for LR grids and target
    grid type is not known (thanks to S. Calmanti from
    M\'et\'eo-France)
    
    - Modifications so that last 4 arguments of call to grid\_init are 
    always arrays even when corners are not defined (error detected with 
    Intel Fortran V10.1.012 by Mike Rezny, SGI, Australia)

  \item In {\tt oasis3/lib/clim.GSIP/src/CLIM\_Init\_Oasis.F},
    correction of a wrongly positioned \#endif (thanks to L. Kornblueh
    from MPI); bug fix announced to the mailing list
    diff-oasis@cerfacs.fr on 31/10/2006.

  \item In {\tt oasis3/lib/psmile/src/prism\_enddef\_proto.F}, the
    call to MPI\_Errhandler\_set was moved after the test on the value
    of mpi\_err returned by MPI\_Buffer\_Detach (thanks to I. Bethke
    from NERSC); bug
    fix announced to the mailing list diff-oasis@cerfacs.fr on
    14/12/2006.

  \item Routine {\tt
      oasis3/lib/psmile/src/prism\_terminate\_proto.F90} was modified
    to ensure proper deallocation of all allocated arrays (thanks to
    Adam Ralph from ICHEC)
  
% \item In {\tt prism/src/mod/toyatm/src/atm.F90} near the end of the
%    routine, toyatm does not detach anymore from the MPI buffer as it
%    is done automatically in prism\_terminate\_proto (thanks to
%    C. Henriet from Cray); bug
%    fix announced to the mailing list diff-oasis@cerfacs.fr on
%    26/01/2006. 

  \item In {\tt oasis3/lib/scrip/remap\_conserv.F}, small bugfix
    having no impact on the results, it just avoids misleading
    messages of type "Error sum wts map1:grid2\_add ..."  to be printed
    in the cplout log file.

  \item In {\tt oasis3/src/getfld.F}, {\tt givfld.F},
    {\tt driver.f} and {\tt closerst.F}, correction of a bug observed
    by A.Caubel from CEA for coupling fields having a sequence index
    greater than 1. For first iteration, closing of netcdf restart
    files is done in driver.f by calling new routine closerst.F.
    Closing is therefore removed from getfld.F.  A minor correction
    (useless opening and closing of first netcdf restart file) was
    also added to givfld.F. Bug
    fix announced to the mailing list diff-oasis@cerfacs.fr on
    09/02/2006.

  \item In {\tt oasis3/src/inipar.F}, bugfix for SEQMODE
    greater than 9 (thanks to T. Silva, Oregon State U.) and to
    avoid array overbound.

  \item In {\tt
      oasis3/util/make\_dir/TopMakefileOasis3}: typo
    error: libmpp\_io.a instead of libmppio.a (thanks to J.M. Epitalon
    from CERFACS)

  \item In {\tt oasis3/lib/psmile/src/prism\_init\_comp\_proto.F}:
    initialisation of {\tt iprcou} (bug fix thanks to J.M. Epitalon).

  \item In {\tt oasis3/src/filling.f}: rewind of file in
    the "AN" case (thanks to S. Calmanti from  M\'et\'eo-France)

  \item In {\tt oasis3/lib/psmile/src/mod\_prism\_put\_proto.F90}:
    bug fix to avoid array overbounds (bug identified T. van Noije
    from KNMI).

  \item In {\tt oasis3/lib/psmile/src/mod\_psmile\_io.F90}: thanks to
    A. Caubel from CEA, modification so that the grid point longitudes
    and latitudes do not have to appear before the corner longitudes
    and latitudes in the grids.nc file (revision 13/01/2009).
    
  \item Bugfix in {\tt
      oasis3/examples/testNONE/PROG/calc\_errorfield.f90} to use the
    absolute value of the error in the non masked point mean error
    calculation.
 
  \item Modifications in {\tt oasis3/lib/clim/src/CLIM\_Init\_Oasis.F}
    and {\tt oasis3/lib/\break psmile/src/prism\_init\_comp\_proto.F} to
    allow communication log file (*.prt* files) for OASIS and/or
    component models to run on up to 9999 processes.

  \end{itemize}

\item Other major modifications

  \begin{itemize}

  \item New directory structure. See section \ref{sec_Obtaining} for details.

  \item Modification of many routines to allow using more than one
    OASIS3 executable in a coupled model resulting in
    pseudo-parallelisation of OASIS3 on a field-per-field basis.  See
    section \ref{sec_pseudopara_mode} for more detail.
    
  \item New directory structure and update of compiling environement
    to work with the new directory structure. In particular, the
    location of directory created for compilation (see {\tt ARCHDIR}
    in the Makefile headers {\tt make.xxx} in {\tt
      oasis3/util/make\_dir}) can be arbitrarily chosen by the user.

  \item Modification of routine {\tt
      oasis3/lib/scrip/src/remap\_distwgt.F} so that \newline SCRIPR/DISTWGT
    that has by default the same behaviour than SCRIPR/BILINEAR,
    /BICUBIC and /CONSERV (with FRACNNEI option) i.e. the non-masked
    nearest neighbour is used for target grid points having their N
    nearest neighbour all masked. To reproduce the previous default
    behaviour, one has to compile with CPP key {\tt
      NOT\_NNEIGHBOUR}. See section \ref{subsec_interp} for details.

  \item Inclusion of an additional number below the {\tt \$NFIELDS}
    keyword in the {\it namcouple} (after the total number of fields
    exchanged, on the same line). This number, corresponding to the
    maximum number of prism\_def\_var\_proto called by ANY component
    model in the coupled system, is needed only if it is greater than
    twice the number of fields listed in the {\it namcouple}; this may
    be the case if OASIS3 is used in pseudo-parallel mode or if fields
    declared with prism\_def\_var\_proto call (and corresponding to
    sending - prism\_put\_proto call- or receiving - prism\_get\_proto
    call - actions in the component models) do not appear in the {\it
      namcouple} (in this case, the sending and receiving calls simply
    return without any action performed).

   \item added options {\tt GLBPOS}, {\tt BASPOS}, {\tt BASBAL} for the 
  cooking stage transformation {\tt CONSERV}. For details, see section
      \ref{subsec_cooking}.
  
   \item New CPP key {\tt TREAT\_OVERLAY} : to ensure that if two cells
    of the source grid overlay, at least the one with the greater
    numerical index is masked (they also can be both masked).  For
    example, if the grid line with i=1 overlaps the grid line with
    i=imax, it is the latter that must be masked. When this
    is not the case with the mask defined in {\it masks.nc}, this CPP key 
    ensures that these rules are respected. This is mandatory
    for {\tt SCRIPR/CONSERV} remapping, see section \ref{subsec_interp}.

  \item Optimisation of SCRIPR (XXX to be detailed) interpolation
    weigths-and-address files, thanks to CMCC. The weights and
    addresses are now read once per run and stored.

  \item mode stats consommation Eric M (XXX to be detailed)
    
  \item mod\_prism\_get\_comm (XXX to be detailed)
  
  \item Release of a new toy coupled model TOYOASIS3. This new toy
    model is described in \ref{sec_coupled_mode}. The toy model
    sources are available in {\tt oasis3/src/mod/oceoa3/}, {\tt
      atmoa3}, {\tt cheoa3}. It running environment is available in
    {\tt oasis3/src/mod/oasis3/examples/toyoasis3}. The grids of the
    TOYOASIS3 component models and the interpolation performed have
    been updated compared to the previous toy coupled model TOYCLIM
    (which is no longer distributed with the official release).

   \item In {\tt oasis3/src/mod/oasis3/src/extrap.F}: optimisation
     (thanks to T. Schoenemeyer from NEC) and {\tt idoitold} changed
     from 1000000 to 10000000 to support higher resolution grids
     (thanks to E.Maisonnave from CERFACS). XXX and CMCC

   \item In {\tt
       oasis3/lib/psmile/src/mod\_prism\_grids\_writing.F90}:
       
     - routine prism\_write\_angle was added to to allow a
     component model to write the angle of its grid (see section
     \ref{subsubsec_griddef} for more detail).
     
     - changed logical {\tt netcdf} for {\tt l\_netcdf} (to run on NEC
     SX9, thanks to E. Maisonnave, CERFACS)

  \item In may 2007, we moved from CVS to SVN for source management.

  \end{itemize}

\item Other modifications

  \begin{itemize}

  \item The names of the log files for the communication information
    {\it *.prt*} are now always ending with 4 digits indicating the
    rank of the component process (e.g. model1.prt0002 or
    model1.prt9999) or the rank of the oasis process
    (e.g. Oasis.prt0001 or Oasis.prt0010).  These modifications
    impacted routines {\tt oasis3/lib/clim/src/CLIM\_Init\_Oasis.F}
    and \\ {\tt oasis3/lib/psmile/src/prism\_init\_comp\_proto.F}.

  \item The following routines were modified for the NAG compiler:
  {\tt oasis3\break /src/getfld.F}, {\tt inipar.F}, {\tt interp.F} and 
  {\tt oasis3/lib/scrip/src/\break remap\_write.F}
 
  \item Some routines in {\tt oasis3/lib/mpp\_io/src/} were
  modified so that all debug and log messages are now written to
  stdout unit and to include modifications done in the OASIS4 version
  for CRAY pointers and for bundles.

  \item Added CPP key {\tt \_\_SILENT} CPP key to reduce log outputs to
  .prt files during the psmile library exchanges (thanks to S.
  Lorenz from MPI)

  \item In {\tt oasis3/src/chkfld.f}, modified
    misleading comment about masked points written to {\it cplout} log
    file (thanks to T. Craig from BOM).

  \item In  {\tt oasis3/lib/psmile/src/mpp\_io/src/mpp\_io\_mod\_oa.F90}
    modified "lowercase" function so to avoid using
    "transfer" function (thanks to Mike Rezny for SGI Melbourne) which
   causes problem with pgf90 5.2.4, 6.1.3 or 7.0 in 64 bit mode, and with
   gfortran 4.2.1 and 4.2.5 SUSE Linux.

  \end{itemize}

\end{itemize}

\section{Changes between {\tt oasis3\_prism\_2\_5} and {\tt
oasis3\_prism\_2\_4}}

The changes between version {\tt oasis3\_prism\_2\_5} and version {\tt
  oasis3\_prism\_2\_4} delivered in December 2004 are listed here
after. Please note that those modifications should not bring any
difference in the interpolation results, except for SCRIPR/DISTWGT
(see below).

\begin{itemize}

\item Bug corrections:

  \begin{itemize}

  \item In {\tt prism/src/lib/scrip/src/scriprmp.F}: initialisation of
    {\tt dst\_array(:)}; bug fix announced to the mailing list
    diff-oasis@cerfacs.fr on 02/02/2006.

  \item In {\tt prism/src/lib/psmile/src/prism\_enddef\_proto.F} and {\tt
      prism/src/lib/\break clim/src/CLIM\_Start\_MPI.F}: the call to
    MPI\_barrier (that created a deadlock when not all processes of a
    component model were exchanging data with the coupler) was changed
    for a call to MPI\_wait on the previous MPI\_Isend; bug fix
    announced to the mailing list diff-oasis@cerfacs.fr on
    02/23/2006.

  \item For SCRIPR/DISTWGT, in {\tt
      prism/src/lib/scrip/src/remap\_distwgt.f}: line 190 was repeated
    without epsilon modification; bug fix announced to the mailing
    list diff-oasis@cerfacs.fr on 03/21/2006.

  \item In {\tt prism/src/lib/psmile/src/mod\_prism\_put\_proto.F90},
    for {\tt prism\_put\break \_proto\_r28} and {\tt
      prism\_put\_proto\_r24}, the reshape of the 2d field was moved
    after the test checking if the field is defined in the namcouple
    (thanks to Arnaud Caubel from LSCE).

  \end{itemize}

\item Modification in SCRIP interpolations

  \begin{itemize}

  \item For {\tt SCRIPR} interpolations (see section
    \ref{subsec_interp}), the value 1.0E+20 is assigned to target grid
    points for which no value has been calculated if {\tt
      prism/src/lib/scrip/src/scriprmp.f} or {\tt vector.F90} (for
    vector interpolation) are compiled with {\tt ll\_weightot =
      .true.}.

  \item For {\tt SCRIPR/GAUSWGT}: if routine {\tt
      prism/src/lib/scrip/src/remap\_gauswgt.f} is compiled with {\tt
      ll\_nnei=.true.}, the non-masked nearest neighbour is used for
    target point if all original neighbours are masked (see section
    \ref{subsec_interp}).

  \item For {\tt SCRIPR/BICUBIC} (routine {\tt
      prism/src/lib/scrip/src/remap\_bicubic.f}), the convergence
    criteria was modified so to ensure convergence even in single
    precision.

  \item For {\tt SCRIPR/CONSERV} (routine {\tt
      prism/src/lib/scrip/src/remap\_conserv.f}), a test was added for
    non-convex cell so that integration does not stall.

  \item The routine {\tt prism/src/lib/scrip/src/corners.F} was
    modified so to abort if it is called for the target grid, as the
    automatic calculation of corners works only for
    Logically-Rectangular (LR) grids and as the target grid type is
    unknown. If needed, the reverse remapping, in which the current
    target grid become the source grid, can be done .

  \end{itemize}

\item Other important modifications

  \begin{itemize}

  \item A new PSMILe routine {\tt
      prism/src/lib/psmile/src/prism\_get\_freq.F} was added; this
    routine can be used to retrieve the coupling period of field (see
    section \ref{subsec:auxiliary}).

  \item The routines of the {\tt mpp\_io} library in {\tt
      prism/src/lib/mpp\_io} changed name and were merged with the
    OASIS4 {\tt mpp\_io} library.

  \item Routine {\tt prism/src/mod/oasis3/src/extrap.F} was modified
    to ensure that the extrapolation works even if the {\tt MASK}
    value is very big (thanks to J.M. Epitalon).

  \item In the namcouple, there is no need anymore to define a lag
    greater than 0 (e.g.  LAG=+1) for fields in mode NONE.

  \item Diverse modifications were included for successful compilation
    with NAGW compiler: non portable use of ``kind'', etc. (thanks to
    Luis Kornblueh from MPI).

  \item In {\tt prism/src/lib/psmile/mod\_prism\_get\_proto.F90}, a
    potential deadlock was removed (the master process was sending a
    message to itself)(thanks to Luis Kornblueh from MPI).

  \item Routine {\tt prism/src/lib/scrip/src/scriprmp\_vector.F90} was
    completely rewritten for more clarity.

  \item Obsolete transformations INVERT and REVERSE were removed from
    the toy coupled model TOYCLIM (in file {\tt
      prism/util/running/toyclim/input/namcouple}. This change does
    not affect the statistics printed in the {\tt cplout} but changes
    the orientation of some fields in the NetCDF ouput files (see the
    results in {\tt prism/data/toyclim/outdata}).

\end{itemize}

\item Other minor modifications:

  \begin{itemize}

  \item In {\tt prism/src/lib/psmile/src/prism\_enddef\_proto.F},
    allocation is done only for rg\_field\_trans or dg\_field\_trans
    depending on precision for REAL (but not for both, to save
    memory).

  \item In few routines in {\tt prism/src/lib/clim} and in {\tt
      prism/src/mod/oasis3}, parentheses were added to make sure that
    \&\& has priority over $|$$|$ in CPP instructions (thanks to A.
    Caubel from LSCE).

%\item corners.f: changed intent(INOUT) grid\_center\_lon/lat 
% for intent(IN) src\_lon/lat and added case for ymean==0
%\item In inipar.F, modification so that restart file
%number and restart file name arrays are not filled up if the field
%does not have a restart file.

  \item Routines {\tt scrip/src/corners.f}, {\tt netcdf.f}, and {\tt
      scriprmp.f} were renamed \newline {\tt corners.F}, {\tt netcdf.F}, {\tt
      scriprmp.F} and the line ``INCLUDE 'netcdf.inc' '' was changed
    for ``\#include $<$netcdf.inc$>$ ''

  \end{itemize}

\end{itemize}

\section{Changes between {\tt oasis3\_prism\_2\_4} and {\tt
      oasis3\_prism\_2\_3}}

  The changes between versions tagged {\tt oasis3\_prism\_2\_4} and {\tt
    oasis3\_prism\_2\_3} delivered in July 2004 are the following:

\begin{itemize}

\item Update of compiling and running environments with version
  \texttt{prism\_2-4} of PRISM Standard Compiling Environment (SCE) and
  PRISM Standard Running Environment (SRE), which among other
  improvements include the environments to compile and run on the CRAY
  X1 (see the directories with \texttt{<node>=baltic1}), thanks to
  Charles Henriet from CRAY France, and on a Linux station from
  Recherche en Pr\'evision Num\'erique (Environnement Canada, Dorval,
  Canada) (see the directories with \texttt{<node>=armc28}).

\item \texttt{prism/src/mod/oasis3/src/iniiof.F}: the opening of the
  coupling restart files is done only if the corresponding field has a
  lag greater than 0; note that this implies that all fields in mode
  NONE must now have a lag greater than 0 (e.g. LAG=+1) (thanks to
  Veronika Gayler from M\&D).

\item \texttt{prism/src/lib/psmile/src/prism\_def\_var\_proto.F}:
 contrary to what was previously described in the documentation,
 \texttt{PRISM\_Double} is not supported as $7^{th}$ argument to
 describe the field type; {\tt PRISM\_Real} must be given for single
 or double precision real arrays.

\item \texttt{prism/src/mod/oasis3/src/inipar.F90}: For upward
   compatibility of SCRIPR interpolation, ``VECTOR" is still accepted
   in the namcouple as the field type 
   and leads to the same behaviour as before (i.e. each vector
   component is treated as an independent scalar field). To have a
   real vector treatment, one has to indicate "VECTOR\_I" or "VECTOR\_J"
   (see section \ref{subsec_interp}).

\item Bug corrections in: 
\begin{itemize}

\item \texttt{prism/src/lib/scrip/src/scriprmp\_vector.F90}: In some
 cases, some local variables were not deallocated and variable
 \texttt{dimid} was declared twice.

\item 
\texttt{prism/src/lib/psmile/src/mod\_psmile\_io.F90}: correct
allocation of array hosting the longitudes (thanks to Reiner Vogelsang
from SGI Germany).

\item
\texttt{prism/src/lib/psmile/src/write\_file.F90}: to remove a deadlock
on some architecture (thanks to Luis Kornblueh from MPI).

\item
\texttt{prism/src/lib/psmile/src/prism\_enddef\_proto.F}: the error
handler is now explicitely set to \texttt{MPI\_ERRORS\_RETURN} before
the call to \texttt{MPI\_Buffer\_Detach} to avoid abort on some
architecture when the component model is not previously attached to
any buffer (thanks to Luis Kornblueh from MPI). 



\item 
  \texttt{prism/src/lib/scrip/src/remap\_conserv.f} (thanks to Veronika
  Gayler from M\&D).

\item \texttt{prism/src/mod/oasis3/src/inicmc.F}

\item \texttt{prism/src/lib/scrip/src/remap\_distwgt.f}

\end{itemize}

\end{itemize}

\section{Changes between {\tt oasis3\_prism\_2\_3} and {\tt
oasis3\_prism\_2\_2}}

The changes between versions tagged {\tt oasis3\_prism\_2\_3}
delivered in July 2004 and {\tt oasis3\_prism\_2\_2} delivered
in June 2004 are the following:

\begin{itemize}

\item Bug correction of the previous bug fix regarding ordering of grid and
data information contained in I/O files when {\tt INVERT} or {\tt
REVERSE} transformations are used: the re-ordering now occurs only for
source field if {\tt INVERT} is used, and only for target field if
{\tt REVERSE} is used.

\item LGPL license: OASIS3 is now officially released under a Lesser GNU General Public
License (LGPL) as published by the Free Software Foundation
(see {\tt prism/src/mod/oasis3/COPYRIGHT} and {\tt prism/src/mod/oasis3/src/couple.f})

\item Upgrade of compiling and running environments: The compiling and
running environments have been upgraded to the PRISM Standard
Compiling and Running Environment version dated August 5th
2004, that should be very close to ``prism\_2-3''.

\item Treament of vector fields: The interpolation algorithms using the SCRIP library now support
vector fields, including automatic rotation from local to geographic
coordinate system, projection in Cartesian coordinate system and
interpolation of 3 Cartesian components, and support of vector
components given on different grids. New routines have been added in
{\tt prism/src/lib/scrip/src}: {\tt scriprmp\_vector.F90} and {\tt rotations.F90}.
For more detail, see {\tt SCRIPR} in section \ref{subsec_interp}.

\item All include of mpif.h are now written `\#include $<$mpif.h$>$'.

\item The output format of {\tt CHECKIN} and {\tt CHECKOUT} results
is now E22.7

\end{itemize}

\section{Changes between {\tt oasis3\_prism\_2\_2} and {\tt
oasis3\_prism\_2\_1}}

The changes between versions tagged {\tt oasis3\_prism\_2\_2}
delivered in June 2004 and {\tt oasis3\_prism\_2\_1} delivered to
PRISM in April 2004 are the following:

\begin{itemize}

\item Bug corrections

 \begin{itemize}

 \item {\tt INTERP/GAUSSIAN} and {\tt SCRIPR/GAUSWGT} transformations
 work for `U' grids.

 \item The grid and data information contained in I/O files output by
 the PSMILe library have now a coherent ordering even if {\tt INVERT}
 or {\tt REVERSE} transformations are used.

 \end{itemize}

\item OASIS3 and the TOYCLIM coupled model are ported to IBM Power4 and
Linux Opteron, which are now included in the Standard Compiling and
Running Environments (SCE and SRE).

\item SIPC technique communication is re-validated.

\item {\tt Clim\_MaxSegments = 338} in {\tt
prism/src/lib/clim/src/mod\_clim.F90} and in {\tt
prism/src/lib/psmile/src/mod\_prism\_proto.F90}. 338 is presently the largest
value needed by a PRISM model. 

\item {\tt MPI\_BSend}: below the call to {\tt prism\_enddef\_proto},
the PSMILe tests whether or not the model has already attached to an
MPI buffer. If it is the case, the PSMILe detaches from the buffer,
adds the size of the pre-attached buffer to the size needed for the
coupling exchanges, and reattaches to an MPI buffer. The model own
call to {\tt MPI\_Buffer\_Attach} must therefore be done before the
call to {\tt prism\_enddef\_proto}. Furthermore, the model is not
allowed to call {\tt MPI\_BSend} after the call to {\tt
prism\_terminate\_proto}, as the PSMILe definitively detaches from the
MPI buffer in this routine. See the example in the toyatm model in
{\tt prism/src/mod/toyatm/src}.

\end{itemize}

\section{Changes between {\tt oasis3\_prism\_2\_1} and {\tt
oasis3\_prism\_1\_2}}

The changes between versions tagged {\tt oasis3\_prism\_1\_2}
delivered in September 2003 and {\tt oasis3\_prism\_2\_1} delivered to
PRISM in April 2004 are the following:

\begin{itemize}

\item Bug corrections

 \begin{itemize}

 \item Thanks to Eric Maisonnave, a bug was found and corrected in
  \- \- \- \- \- \- \- \- 
  /prism/src/lib/scrip/src/scriprmp.f:
  ``sou\_mask'' and ``tgt\_mask'' were not properly initialised if weights
  and addresses were not calculated but read from file.

 \item Some deallocation were missing in prism\_terminate\_proto.F
  (``ig\_def\_part'', ``ig\_length\_part'', ``cg\_ignout\_field'').

 \item Thanks to Arnaud Caubel, a bug was found and corrected in 
  \- \- \- \- \- \- \- \- /prism/src/lib/psmile/src/write\_file.F90. In case of parallel
  communication between a model and OASIS3 main process, the binary
  coupling restart files were not written properly (NetCDF coupling
  restart files are OK).

 \end{itemize} 

\item Routines renamed

The routines {\tt preproc.f, extrap.f, iniiof.f} in {\tt
prism/src/mod/oasis3/src} were renamed to {\tt preproc.F, extrap.F,
iniiof.F}, as a CPP key `key\_openmp' was added. Please note that this
key, allowing openMP parallelisation, is not fully tested yet.

\item Modifications in the namcouple

\begin{itemize}
\item The third entry on the field first line now corresponds to an index in
the new auxiliary file {\em cf\_name\_table.txt} (see sections
\ref{subsec_namcouplesecond} and \ref{subsec_cfnametable}).

\item For {\tt IGNORED, IGNOUT} and {\tt OUTPUT} fields, the source
and target grid locator prefixes must now be given on the field second
line (see section \ref{subsubsec_secondIGNORED})
\end{itemize}

\item A new auxiliary file {\em cf\_name\_table.txt}

For each field, the CF standard name used in the OASIS3 log file, {\em
cplout}, is now defined in an additional auxiliary file {\em
cf\_name\_table.txt} 
not in {\tt inipar.F} anymore. This auxiliary file must be
copied to the working directory at the beginning of the run. The user
may edit and modify this file at her own risk. In {\em
cf\_name\_table.txt}, an index is given for each field standard name
and associated units.  The appropriate index has to be indicated for
each field in the {\em namcouple} (third entry on the field first
line, see section \ref{subsec_namcouplesecond}). 

This standard name and the associated units are also used to define the
field attributes ``long\_name'' and ``units'' in the NetCDF output files
written by the PSMILe for fields with status {\tt EXPOUT, IGNOUT} and
{\tt OUTPUT}.
 
For more details on this auxiliary file, see section
\ref{subsec_cfnametable}.

\item Many timesteps for mode NONE

In mode NONE, OASIS3 can now interpolate at once all time occurrences
of a field contained in an input NetCDF file. The time variable in the
input file is recognized by its attribute ``units''. The acceptable
units for time are listed in the udunits.dat file \cite{udunits}. This
follows the CF convention.

The keyword {\tt \$RUNTIME} in the namcouple has to be the number of time
occurrences of the field to interpolate from the input file. The
``coupling'' period of the field (4th entry on the field first line)
must be always ``1''. Note that if {\tt \$RUNTIME} is smaller than the
total number of time ocurrences in the input file, the first {\tt \$RUNTIME} 
occurrences will be interpolated.

For more details, see section \ref{subsec_interpolator}.

\item Model grid data file writing

The grid data files {\em grids.nc, masks.nc} and {\em areas.nc} can
now be written directly at run time by the component models, if they
call the new routines prism\_start\_grids\_writing, prism\_write\_grid,
prism\_write\_corner prism\_write\_mask, prism\_write\_area,
prism\_terminate\_grids\_writing. 

The writing of those grid files by the models is driven by the
coupler. It first checks whether the binary file {\em grids} or the
netCDF file {\em grids.nc} exists (in that case, it is assumed that
{\em areas} or {\em areas.nc} and {\em masks} or {\em masks.nc} files
exist too) or if writing is needed. If {\em grids} or {\em grids.nc}
exists, it must contain all grid information from all models; if it does not
exist, each model must write its grid informations in the grid data
files. 

See section \ref{subsubsec_griddef} for more details.

\item Output of CF compliant files

The NetCDF output files written by the PSMILe for fields with status
{\tt EXPOUT, IGNOUT} and {\tt OUTPUT} are now fully CF compliant.

In the NetCDF file, the field attributes ``long\_name'' and ``units''
are the ones corresponding to the field index in {\em
cf\_name\_table.txt} (see above and section
\ref{subsec_cfnametable}). The field index must be given by the user
as the third entry on the field first line in the namcouple.

Also, the latitudes and the longitudes of the fields are now
automatically read from the grid auxiliary data file {\em grids.nc} and
written to the output files. If the latitudes and the longitudes of
the mesh corners are present in {\em grids.nc}, they are also written to the
ouput files as associated ``bounds'' variable.  This works whether the
{\em grids.nc} is given initially by the user or written at run time by the
component models (see above). However, this does not work if the
user gives the grid definition in a binary file {\em grids}.

\item Removal of pre-compiling key ``key\_BSend''


The pre\_compiling key ``key\_BSend'' has been removed.  The default has
changed: by default, the buffered MPI\_BSend is used, unless {\tt NOBSEND}
is specified in the namcouple after MPI1 or MPI2, in which case the
standard blocking send MPI\_Send is used to send the coupling fields.

\end{itemize}



