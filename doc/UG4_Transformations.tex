\newpage
\chapter{The transformations and interpolations in OASIS3}
\label{sec_transformations}

Different transformations and 2D interpolations are available in
OASIS3 to adapt the coupling fields from a source model grid to a target
model grid. They are divided into five general classes that have
precedence one over the other in the following order: time
transformation, pre-processing, 
interpolation, ``cooking'', and post-processing. This order of
precedence is conceptually logical, but is also constrained by the 
OASIS3 software internal structure.

In the following paragraphs, a description of each
transformation with its corresponding configuring lines is given.
Features that are now deprecated (non functional) compared
to prior versions will be noted with the string UNUSED
but not described.

%\section{Using OASIS3 in the interpolator-only mode}
%\label{subsec_interpolator}
%
%OASIS3 can be used in an interpolator-only mode, in which case it 
% transforms fields without running any model.  It is
% recommended to use first OASIS3 in this mode
% to test different transformations and interpolations without
% having to run the whole coupled system. In the interpolator-only mode,
% all transformations, except the time transformations, are available.
%
%To run OASIS3 in an interpolator-only mode, the user has to prepare
%the {\it namcouple} as indicated in sections
%\ref{subsec_namcouplefirst} and \ref{subsec_namcouplesecond}. In
%particular, {\tt NONE} has to be chosen below the keyword {\tt
%\$CHANNEL}; ``0'' (without any model name and Fortran unit number)
%must be given below the keyword {\tt \$NBMODEL}; {\tt \$RUNTIME} has
%to be the number of time occurrences of the field to interpolate from
%the NetCDF input file\footnote{For binary input file, only one time
%occurence may be interpolated}; finally, the ``coupling'' period of
%the field (4th entry on the field first line) must be always ``1''.
%Note that if {\tt \$RUNTIME} is smaller than the total number of time
%ocurrences in the input file, the first {\tt \$RUNTIME} occurrences
%will be interpolated.
%
%The name of the input file which contains the fields to interpolate is
%given by the 6th entry on the field first line (see
%\ref{subsec_namcouplesecond}). After their transformation, OASIS3
%writes them to their output file which name is the 7th entry on the
%first line. Note that there must be one input file and one output file
%per field.
%
%The time variable in the input file, if any, is recognized by the its
%attribute "units".  The acceptable units for time are listed in the
%udunits.dat file \cite{udunits}.  This follows the CF convention.
%
%To compile OASIS3 in interpolator-only mode, see section \ref{sec_notSCE}.
%Practical examples on how to use OASIS3 in a interpolator-only mode are
%given in {\tt oasis3/examples/testinterp} (see also
%section \ref{subsec_running_testinterp}) and {\tt
%  oasis3/examples/testNONE} (see also section
%\ref{subsec_running_testnone})
%
%The configuring parameters that have to be defined in the {\it
%namcouple} for each transformation in the interpolator-only mode or in
%the coupling mode are described here after.
 
%{Running OASIS3 in the interpolator-only mode}

\section{The time transformations}
\label{subsec_timetrans}

\begin{itemize}

\item {\bf LOCTRANS}:

{\tt LOCTRANS} requires one configuring line on which a time
transformation, automatically performed below the call to PSMILe {\tt
  prism\_put\_proto}, should be indicated:

Non INSTANT time transformations are now supported more generally
with use of a restart file.  The restart file allows the partial
time transformation to be saved at the end of the run for exact
restart at the start of the next run.  For that reason, restart
filenames are now required for all {\it namcouple} transforms that use
LOCTRANS with non INSTANT values.  In this mode, oasis will exit
gracefully with an error message if a restart filename is not provided.
This is the reason an optional restart file is now provided on the
OUTPUT {\it namcouple} input line.

  \begin{verbatim}
 # LOCTRANS operation
     $TRANSFORM
  \end{verbatim} 

where {\tt \$TRANSFORM} can be

  \begin{itemize}
    \item {\tt INSTANT}: no time transformation, the instantaneous field is
    transferred; 
    \item {\tt ACCUMUL}: the field accumulated over the previous coupling
    period is exchanged (the accumulation is simply done over the arrays
    {\tt field\_array}  provided as third argument to the {\tt
    prism\_put\_proto} calls, not weighted by the time interval
    between these calls);
    \item {\tt AVERAGE}: the field averaged over the previous coupling
    period is transferred (the average is simply done over the arrays
    {\tt field\_array} provided as third argument to the {\tt
    prism\_put\_proto} calls, not weighted by the time interval
    between these calls);
    \item {\tt T\_MIN}: the minimum value of the field
    for each source grid point over the previous coupling period is
    transferred; 
    \item {\tt T\_MAX}: the maximum value of the field for each source grid
    point over the previous coupling period is transferred;
    \item ONCE: UNUSED
%only one {\tt prism\_put\_proto} or {\tt
%prism\_get\_proto} will be performed; this is equivalent to giving the
%length of the run as coupling or I/O period.
  \end{itemize}
\end{itemize}

\section{The pre-processing transformations}
\label{subsec_preproc}

The following transformations are available in the pre-processing part of
OASIS3, controlled by {\tt preproc.f}.
 
\begin{itemize}

\item {\bf REDGLO} UNUSED
   
%  {\it This transformation is deprecated in the current OASIS3 version as
%  interpolations for Gaussian Reduced grid now exist; this transformation
%  should not be used anymore.}

% {\tt REDGLO} (routine {\tt redglo.f}) performs the interpolation from a
% Reduced grid to a Gaussian one. The interpolation is linear and
% performed latitude circle per
% latitude circle. When present, REDGLO must be the first pre-processing
% transformation performed.
% The configuring line is as follows:
% \begin{verbatim}
% # REDGLO operation
%     $NNBRLAT  $CDMSK \end{verbatim}where {\tt \$NNBRLAT} is NOxxx 
% where xxx is half the number of latitude circles of the Gaussian
% grid. For example, for a T42 with 64 latitude circles, {\tt
% \$NNBRLAT} is ``NO32''. In the current version, it can be either
% NO16, NO24, NO32, NO48, NO80, NO160. {\tt \$CDMSK} is a flag
% indicating if non-masked values have to be extended to masked areas
% before interpolation ({\tt \$CDMSK = SEALAND}) using the Reduced grid
% mask (see section \ref{subsec_griddata}) or if the opposite has to be
% performed ({\tt \$CDMSK = LANDSEA}).  If {\tt \$CDMSK = NOEXTRAP},
% then no extrapolation is performed.

\item {\bf INVERT}: UNUSED

%  {\it This transformation is deprecated in the current OASIS version
%    and should be used anymore. The fields and corresponding grids can
%    be given in any direction as long as they are coherent.}
%
% {\tt INVERT} (routine {\tt invert.f}) reorders a field so that it
% goes from south to north and from west to east (the first point will
% be the southern and western most one; then it goes parallel by parallel
% going from south to north). Note that {\tt INVERT} does not transform the
% associated grid or mask. {\tt INVERT} should be used only for
% fields associated to A, B, G, L, Z, or Y grids (see annexe
% \ref{subsec_gridtypes}) but produced by the source model from North to
% South and/or from East to West. {\tt INVERT} does not work for
% Reduced ('D') or unstructured ('U') grids (see annexe
% \ref{subsec_gridtypes}).
%
% The generic input line is as follows:
% \begin{verbatim}
% # INVERT operation
%     $CORLAT  $CORLON \end{verbatim} where
% {\tt \$CORLAT = NORSUD} or {\tt SUDNOR} and {\tt \$CORLON = ESTWST}
% or {\tt WSTEST} describes the orientation of the source field in
% longitude and latitude, respectively. 

\item {\bf MASK}: UNUSED
 
% {\tt MASK} (routine {\tt masq.f}) is used before the analysis {\tt
% EXTRAP}.  A given {\tt REAL} value {\tt VALMASK} is assigned to all masked
% points following the source grid mask (see section
% \ref{subsec_griddata}), so they can be detected by {\tt EXTRAP}.
%
% The generic input line is as follows:
% \begin{verbatim}
% # MASK operation
%     $VALMASK \end{verbatim}
%   Make sure that \$VALMASK conforms to a {\tt REAL} value, e.g. ``99999999." not ``99999999" . Problems may arise if the value chosen
% approaches the maximum value that your computing platform can
% represent; choose a value well outside the range of
% your field values but not too large.

\item {\bf EXTRAP}: UNUSED

% {\tt EXTRAP} (routine {\tt extrap.f}) performs the extrapolation of a
% field over its masked points.  The analysis {\tt MASK} must be used
% just before, so that {\tt EXTRAP} can identify masked points. Note
% that {\tt EXTRAP} does not work for Reduced ('D') or unstructured
% ('U') grids (see appendix \ref{subsec_gridtypes}).
%
% Two methods of extrapolation are available. With {\tt NINENN}, a
% N-nearest-neighbour method is used. The procedure is iterative and the
% set of remaining masked points evolves at each iteration.  The
% configuring line is:
% \begin{verbatim}
% # EXTRAP operation for $CMETH = NINENN
%     $CMETH  $NV  $NIO  $NID\end{verbatim} where
% {\tt \$CMETH = NINENN}; {\tt \$NV} is the minimum number
% of neighbours required to perform the extrapolation (with a maximum
% of 4)\footnote{For some grids, the extrapolation may not converge if
% {\tt \$NV} is too large.}; {\tt \$NIO} is the flag that
% indicates if the weight-address-and-iteration-number dataset will be
% calculated and written by OASIS3 ({\tt \$NIO}= 1), or only read ({\tt
% \$NIO}= 0) in file {\em nweights} (see section
% \ref{subsec_transformationdata}); {\tt \$NID} is the identificator for
% the weight-address-iteration-number dataset in all the different
% {\tt EXTRAP/NINENN} datasets in the present coupling.\footnote{An
% {\tt EXTRAP/NINENN} analysis is automatically performed within {\tt GLORED}
% analysis but the corresponding datasets have to be distinct; this is 
% automatically checked by OASIS3 at the beginning of the run.}
%
% With {\tt \$CMETH = WEIGHT}, an N-weighted-neighbour extrapolation
% is performed. In that case, the user has to build the grid-mapping
% file, giving for each target grid point the weights and addresses
% of the source grid points used in the extrapolation; the structure of this
% file has to follow the OASIS3 generic structure for transformation
% auxiliary data files (see section \ref{subsec_transformationdata}).
%
% The configuring line is:
% \begin{verbatim}
% # EXTRAP operation for $CMETH = WEIGHT
%     $CMETH  $NV  $CFILE  $NUMLU  $NID\end{verbatim} where
% {\tt \$CMETH = WEIGHT}; {\tt \$NV} is the maximum number of
% neighbours required by the extrapolation operation; {\tt \$CFILE} and
% {\tt \$NUMLU} are the grid-mapping file name and associated logical
% unit; {\tt \$NID} is the identificator for the relevant grid-mapping
% dataset in all different {\tt EXTRAP/WEIGHT} transformations in the present
% coupling.


\item {\bf CHECKIN}:

 {\tt CHECKIN} calculates the mean and extremum
 values of the source field and prints them to the oasis log files
 (this operation does not transform the field).

 The generic input line is as follows:
 \begin{verbatim}
 # CHECKIN operation
     $INT = $NINT  
 \end{verbatim} 
 where {\tt \$NINT} = 1 or 0, depending on whether or not the source
 field integral is also calculated and printed.

\item {\bf CORRECT}: UNUSED
 
% {\tt CORRECT} allows users to scale and add a scalar to a
% field.  The ability to  read external fields and to use them
% to modify the coupling field is no longer available. This transformation
% can be used, for example, to change units on the field.
%
% This transformation requires at least one configuration line with two
% parameters:
% \begin{verbatim}
% # CORRECT operation 
%     $XMULT  $NBFIELDS \end{verbatim} where
% {\tt \$XMULT} is the multiplicative coefficient of the current
% field, and {\tt \$NBFIELDS} is UNUSED and must be set to 1.
% \begin{verbatim}
% # nbfields lines
%     $CLOC  $AMULT  $CFILE  $NUMLU \end{verbatim} where
% {\tt \$AMULT} specifies the additive value.  
% {\tt \$CLOC} (string), {\tt \$CFILE} (string) and {\tt \$NUMLU} (integer)
% are all UNUSED but still required for parsing.  
% \ref{subsec_restartdata}).
%
\end{itemize}

%subsection{The pre-processing transformations}

\section{The interpolation}
\label{subsec_interp}

The following transformations are available in OASIS3.

\begin{itemize}

\item {\bf BLASOLD}: 

{\tt BLASOLD} allows the source fields to be scaled and allows
a scalar to be added to the field.  The prior ability to perform 
a linear combination of the current coupling field with other coupling fields 
has been deprecated.  This transform occurs before the interpolation {\it per se}.

This transformation requires at least one configuring line with two
parameters:
 \begin{verbatim}
# BLASOLD operation
     $XMULT   $NBFIELDS 
 \end{verbatim} 
where {\tt \$XMULT} is the multiplicative coefficient of the current
field, and {\tt \$NBFIELDS} is UNUSED and should be
set to 0 or 1.  To add a scalar, an additional input line is required:
\begin{verbatim}
# nbfields lines
     CONSTANT  $AVALUE
\end{verbatim}
where {\tt CONSTANT} is required and {\tt \$AVALUE} is the
constant that is added to the field.

\item {\bf MAPPING}: 

  The {\tt MAPPING} keyword is used to specify an input file to be
  read and used for mapping (ie. regridding or interpolation); the {\tt MAPPING} file must follow the {\tt SCRIPR} format. 
  This is an alternative method to {\tt SCRIPR} for setting the mapping file.
 
  In the current implementation, each pair of source and target points in the {\tt MAPPING} file can be linked by only one weight, i.e. remappings such as {\tt SCRIPR/BICUBIC} involving at each ssource grid point the value of the field, of the gradients and of the cross-gradient, or second-order conservative remapping are not supported.   

This transformation requires at least one configuring line with one
filename and two optional string values:
\begin{verbatim}
     $MAPNAME  $MAPLOC  $MAPSTRATEGY
\end{verbatim}

  \begin{itemize}
  \item {\tt \$MAPNAME} is the name of the mapping file to read.  This is a netcdf
  file consistent with the OASIS/SCRIPR map file format.

  \item {\tt \$MAPLOC} is optional and can be either {\tt src} or {\tt dst}.  With {\tt src}, the mapping will be done
  in parallel on the source processors before communication on the destination
  grid to the destination model and processors; this is the default.   With {\tt dst}, the mapping is 
  done on the destination processors after the data is sent from the source
  model on the source grid. 

  \item {\tt \$MAPSTRATEGY} is optional and can be either {\tt bfb}, {\tt sum}, or {\tt opt}.  In {\tt bfb} mode, the mapping is
  done using a strategy that produces bit-for-bit identical results regardless
  of the grid decompositions without leveraging a partial sum computation.  With
  {\tt sum}, the transform is done using the partial sum approach which generally
  introduces roundoff level changes in the results on different processor
  counts. Option {\tt opt} allows the coupling layer to choose either approach based on
  an analysis of which strategy is likely to run faster. Usually, partial sums will be used if the source grid has a higher resolution than the target grid as this should reduce the overall communication.

  \end{itemize}

Note that if {\tt SCRIPR} (see below) is used to calculate the remapping file, {\tt MAPPING} can still be listed in the {\tt namcouple} to specify a name for the remapping file generated by {\tt SCRIPR} different from the default and/or to specify a {\tt \$MAPLOC} or {\tt \$MAPSTRATEGY} option.

\item {\bf SCRIPR}: 
 
  {\tt SCRIPR} gathers the interpolation techniques offered by Los
  Alamos National Laboratory SCRIP 1.4 library\footnote{See the
    copyright statement in appendix \ref{sec_SCRIP}.}\cite{SCRIPR}.
  {\tt SCRIPR} routines are in {\tt oasis3/lib/scrip}. See the SCRIP 1.4
  documentation in {\tt oasis3/doc/SCRIPusers.pdf} for more
  details on the interpolation algorithms.  In the current implementation,
  only first order mapping is supported through either scrip weights
  generation or reading from a file generated off-line.  As a result,
  second order {\tt CONSERV} or {\tt BICUBIC} are not currently supported.
  Second order mapping methods will be added in future versions.

  The following types of interpolations are available:

  \begin{itemize}

  \item {\tt DISTWGT} performs a distance weighted nearest-neighbour
    interpolation (N neighbours). All types of grids are supported. 

     \begin{itemize}

     \item Masked target grid points: the zero value is associated to
       masked target grid points.

     \item Non-masked target grid points having some of the N source
       nearest neighbours masked: a nearest neighbour algorithm using
       the remaining non masked source nearest neighbours is applied.

     \item Non-masked target grid points having all of the N source
       nearest neighbours masked: by default, the nearest non-masked
       source neighbour is used. 
% footote about NOT_NNEIGHBOUR and ll_weightot removed

     \end{itemize}

  The configuring line is:

  \begin{verbatim}
 # SCRIPR (for DISWGT) 
     $CMETH $CGRS $CFTYP $REST $NBIN $NV $ASSCMP $PROJCART
  \end{verbatim} where:
  \begin{itemize} 
  \item {\tt \$CMETH = DISTWGT}. 
  \item {\tt \$CGRS} is the source grid
  type ({\tt LR}, {\tt D} or {\tt U})- see annexe
  \ref{subsec_gridtypes}. 

  \item {\tt \$CFTYP} is the field type: {\tt SCALAR} if the field is
  a scalar one, or {\tt VECTOR\_I} or {\tt VECTOR\_J} whether the
  field represents respectively the first or the second component of a
  vector field (see paragraph {\bf Support of vector fields}
  below). The option {\tt VECTOR}, which in fact leads to a scalar treatment
  of the field (as in the previous versions), is still accepted.

  \item {\tt \$REST} is the search restriction type: {\tt LATLON}
  or {\tt LATITUDE} (see SCRIP 1.4 documentation SCRIPusers.pdf). Note that for {\tt
  D} or {\tt U} grid, the restriction may influence sligthly the
  result near the borders of the restriction bins. (XXX to be checked)
  \item {\tt \$NBIN} the number of restriction bins (see SCRIP 1.4
  documentation SCRIPusers.pdf). 
  \item {\tt \$NV} is the number of neighbours used.
  \item {\tt \$ASSCMP}: optional, for {\tt VECTOR\_I} or {\tt VECTOR\_J} vector fields only; the source symbolic name of the associated vector component.
  \item {\tt \$PROJCART}: optional, for vector fields only; should be {\tt PROJCART} if the user wants the vector components to be projected in a Cartesian coordinate system before interpolation (see paragraph {\bf Support of vector fields} below). 
  \end{itemize}

\item {\tt GAUSWGT} performs a N nearest-neighbour interpolation
  weighted by their distance and a gaussian function. All grid types
  are supported.  
  \begin{itemize}  

  \item Masked target grid points: the zero value is associated to masked
  target grid points.

  \item Non-masked target grid points having some of the N source
  nearest neighbours masked: a nearest neighbour algorithm using the
  remaining non masked source nearest neighbours is applied.

  \item Non-masked target grid points having their N nearest neighbours all
  masked: 
% XXXX modifier la routine et la doc XXXXX 
  the zero value will
  be associated to these target points.  
%  The value 1.0E+20 will however be assigned to these non-masked target
%  grid points if routines {\tt scriprmp.f} or {\tt vector.F90} (for
%  vector interpolation) in {\tt oasis3/lib/scrip/src/} are compiled
%  with \newline {\tt ll\_weightot=.true.}.  Furthermore, if {\tt
%    oasis3/lib/scrip/src/remap\_gauswgt.f} is compiled with {\tt
%    ll\_nnei=.true.}, the non-masked nearest neighbour value will be
%  given to these target grid points.
  \end{itemize}

  The configuring line is:
  \begin{verbatim}
 # SCRIPR (for GAUSWGT)
     $CMETH  $CGRS  $CFTYP  $REST  $NBIN  $NV $VAR $ASSCMP $PROJCART 
  \end{verbatim} where:
  all entries are as for  {\tt DISTWGT}, except that:
  \begin{itemize} 
   \item {\tt \$CMETH = GAUSWGT}
   \item {\tt \$VAR}, which must be given as a
    REAL value (e.g 2.0 and not 2), defines the weight given to a
    neighbour source grid point as
    proportional to $exp(-1/2 \cdot d^2/\sigma^2)$ where $d$ is the
    distance between the source and target grid points, and $\sigma^2 =
    \$VAR \cdot \overline{d}^2$ where $\overline{d}^2$ is the average
    distance between two source grid points (calculated automatically
    by OASIS3).
  \end{itemize}

  \item {\tt BILINEAR} performs an interpolation based on a local bilinear approximation
    (see details
    in chapter 4 of SCRIP 1.4 documentation SCRIPusers.pdf)

  For {\tt BILINEAR} and {\tt BICUBIC}, Logically-Rectangular (LR) and
  Reduced (D) source grid types are supported.

  \begin{itemize}  
  \item Masked target grid points: the zero value is associated to masked
  target grid points.

  \item Non-masked target grid points having some of the source points
  normally used in the bilinear or bicubic interpolation masked: a N
  nearest neighbour algorithm using the remaining non masked source
  points is applied.

  \item Non-masked target grid points having their N nearest neighbours all
  masked: 
  the zero value will
  be associated to these target points.  
% The value 1.0E+20 will however be assigned to these non-masked target
%  grid points if routines {\tt scriprmp.f} or {\tt vector.F90} (for
%  vector interpolation) in {\tt oasis3/lib/scrip/src/} are compiled
%  with \newline {\tt ll\_weightot=.true.}.  Furthermore, if {\tt
%  oasis3/lib/scrip/src/remap\_bicubic.f or remap\_bilinear.f} is 
%  compiled with {\tt ll\_nnei=.true.}, the non-masked nearest neighbour 
%  value will be given to these target grid points.
  \end{itemize} 
 
  The configuring line is:

  \begin{verbatim}
 # SCRIPR  (for BILINEAR or BICUBIC)
     $CMETH  $CGRS  $CFTYP  $REST  $NBIN $ASSCMP $PROJCART
  \end{verbatim}where:
  \begin{itemize}
  \item {\tt \$CMETH = BILINEAR} or {\tt BICUBIC}
  \item {\tt \$CGRS} is the source grid type (LR or D)
  \item {\tt \$CFTYP}, {\tt \$NBIN}, {\tt \$ASSCMP} {\tt \$PROJCART} are
  as for {\tt DISTWGT}. 
  \item {\tt \$REST} is as for {\tt DISTWGT}, except that only
  {\tt LATITUDE} is possible for a Reduced (D) source grid.
  \end{itemize}
 
  \item {\tt BICUBIC}
    is currently unsupported as a mapping option becuase it is higher
    order.  It performs an interpolation based on a local bicubic approximation
    (see details
    in chapter 5 of SCRIP 1.4 documentation SCRIPusers.pdf)
   See BILINEAR for configure line
 
  \item {\tt CONSERV} performs 1st or 2nd order conservative remapping,
  which means that the weight of a source cell is proportional to area
  intersected by the target cell.  Note that 2nd order conservative mapping
  files can be generated but not used currently.

  The configuring line is:
  \begin{verbatim}
 # SCRIPR (for CONSERV)
     $CMETH  $CGRS  $CFTYP  $REST  $NBIN  $NORM  $ORDER $ASSCMP $PROJCART
  \end{verbatim}where: 
  \begin{itemize}
  \item {\tt \$CMETH = CONSERV} 
  \item {\tt \$CGRS} is the source grid type: LR, D and U are
  supported for first-order remapping if the grid corners are given by
  the user in the grid data file {\tt grids.nc} ; only LR
  is supported if the grid corners are not available in {\tt grids.nc} 
  and therefore have to be calculated automatically by
  OASIS3. For second-order remapping, only LR is supported because the
  gradient of the coupling field used in the transformation has to be
  calculated automatically by OASIS3.
  \item {\tt \$CFTYP, \$REST}, {\tt \$NBIN}, {\tt \$ASSCMP},and {\tt
    \$PROJCART} are as for {\tt DISTWGT}. 
  \item {\tt \$NORM} is the NORMalization option:
  \begin{itemize}
   \item {\tt FRACAREA}: The sum of the non-masked source cell intersected areas
    is used to NORMalise each target cell field value: the flux is not
    locally conserved, but the flux value itself is reasonable.
   \item {\tt DESTAREA}: The total target cell area is used to NORMalise
    each target cell field value even if it only partly intersects
    non-masked source grid cells: local flux conservation is ensured,
    but unreasonable flux values may result.
   \item {\tt FRACNNEI}: as {\tt FRACAREA}, except that at least the
    source nearest unmasked neighbour is used for unmasked target
    cells that intersect only masked source cells. Note that a zero value
    will be assigned to a target cell that does not intersect any source
    cells (masked or unmasked), even with FRACNNEI option.
  \end{itemize} 
  \item {\tt \$ORDER}: {\tt FIRST} or {\tt SECOND} for first or
  second order remapping respectively (see SCRIP 1.4 documentation).
  Note that {\tt CONSERV/SECOND} is not positive definite and has not been 
  fully validated yet.

\end{itemize}

\end{itemize}

{\bf Precautions related to the use of the SCRIPR/CONSERV remapping in particular}

\begin{itemize}

\item For the 1st order conservative remapping: the weight of a source
  cell is proportional to area of the source cell intersected by
  target cell.  Using the divergence theorem, the SCRIP library
  evaluates this area with the line integral along the cell borders
  enclosing the area. As the real shape of the borders is not known
  (only the location of the 4 corners of each cell is known), the
  library assumes that the borders are linear in latitude and
  longitude between two corners.  In general, this assumption becomes
  less valid closer to the pole and for latitudes above the {\tt
    north\_thresh} or below the {\tt south\_thresh} values specified
  in {\tt oasis3/lib/scrip/remap\_conserv.F}, the library evaluates
  the intersection between two border segments using a Lambert
  equivalent azimuthal projection. Problems have been observed in some
  cases for the grid cell located around this {\tt north\_thresh} or
  {\tt south\_thresh} latitude.

\item Another limitation of the SCRIP 1st order conservative remapping 
  algorithm is that is also supposes, for
  line integral calculation, that $sin(latitude)$ is linear in
  longitude on the cell borders which again is in general not valid
  close to the pole. 
  %A projection or at least a normalization by the
  %true area of the cells (i.e. by the areas as considered by the
  %component models) is needed.

 \item For a proper consevative remapping, the corners of a cell have
   to coincide with the corners of its neighbour cell.
  
\item  If two cells of a grid overlay, at least the one with the
    greater numerical index must be masked (they also can be both
    masked) for a proper conservative remapping.  
    For example, if the grid line with i=1 overlaps the grid
    line with i=imax, it is the latter that must be masked.  When this
    is not the case with the mask defined in {\it masks.nc}, OASIS must 
    be compiled with the CPP key {\tt TREAT\_OVERLAY} which will ensure 
    that these rules are respected. This CPP key was introduced in 
    {\tt oasis3\_3} version.
      
\item A target grid cell intersecting no source cell (either 
    masked or non masked) at all i.e. falling in a ``hole'' of the source grid
    will in all cases get a zero value. 
    
\item If a target grid cell intersects only masked source cells, 
    it will still get a zero value unless:
    
    - the {\tt FRACNNEI} normalisation option is used, in which case 
    it will get the nearest non masked neighbour value, or
    
    - the routines {\tt oasis3/lib/scrip/src/scriprmp.f} or
    {\tt vector.F90} - for vector interpolation - are compiled with {\tt
      ll\_weightot=.true.} in which case, the value 1.0E+20 will be assigned
      to these target grid cell intersecting only masked source cells (for 
      easier identification).
   

%     The target grid mask is never considered in {\tt CONSERV}, except
%     with normalisation option {\tt FRACNNEI} (see below). To have a
%     value calculated, a target grid cell must intersect at least one
%     source cell. However, the NORMlisation option (that takes into
%     account the source grid mask, see below) may result in a null
%     value calculated for those target grid cells. In that case (i.e.
%     at least one intersecting source cell, but a null value finally
%     calculated because of the normalisation option), the value 1.0E+20
%     is assigned to those target grid points if {\tt
%       prism/src/lib/scrip/src/scriprmp.f} or {\tt vector.F90} (for
%     vector interpolation) are compiled with {\tt ll\_weightot=.true.}.
    
   
\end{itemize}

{\bf Precautions related to the use of the SCRIPR remappings in general}

\begin{itemize}

\item For using {\tt SCRIPR} interpolations, linking with the NetCDF library
    is mandatory and the grid data files (see section \ref{subsec_griddata})
    must be NetCDF files (binary files are not supported). 

\item When the SCRIP library performs a remapping, it first checks if the
file containing the corres\-ponding remapping weights and addresses
exists. If it exists, it reads them from the file; if not, it calculates
them and store them in a file. The file is created in the working
directory and is called {\tt
  rmp\_{\it srcg}\_to\_{\it tgtg}\_{\it INTTYPE}\_{\it NORMAOPT}.nc}, where {\it srcg} and
{\it tgtg} are the acronyms of respetively the source and the target
grids, {\it INTTYPE} is the interpolation type (i.e. {\tt DISTWGT},
{\tt GAUSWGT}, {\tt BILINEA}, {\tt BICUBIC}, or {\tt CONSERV}) and
{\it NORMAOPT} is the normalization option (i.e. {\tt DESTAREA},
{\tt FRACAREA} or {\tt FRACNNEI} for {\tt CONSERV} only). The problem comes from the
fact that the weights and addresses will also differ whether or not
the {\tt MASK} and {\tt EXTRAP} transformations are first activated
during the pre-processing phase (see section \ref{subsec_preproc}) and this
option is not stored in the remapping file name. Therefore, the
remapping file used will be the one created for the first field having
the same source grid, target grid, and interpolation type (and the
same normalization option for {\tt CONSERV}), even if the {\tt MASK}
and {\tt EXTRAP} transformations are used or not for that field.
(This inconsistency is however usually not a problem as the {\tt MASK}
and {\tt EXTRAP} transformations are usually used for all fields
having the same source grid, target grid, and interpolation type, or
not at all.)
\end{itemize}

{\bf Support of vector fields with the SCRIPR remappings}

Vector mapping is NOT supported in this version of OASIS.  Vector
fields are treated as if each field were an independent scalar
field.  Vector mapping capabilities will be added in future versions. 

%{\tt SCRIPR} supports 2D vector interpolation. Note however
%that this functionality has been
%validated only in a reduced number of test cases. The two vector
%components have to be identified by assigning {\tt VECTOR\_I} or {\tt VECTOR\_J}
%to {\tt \$CFTYP} and have to be associated by giving, for each component field,
%the source symbolic name of the associated vector component to {\tt \$ASSCMP} 
%(see above). The grids of the
%two vector components can be different but have to have the same
%number of points, the same overlap, the same mask; the same
%interpolation must be used for the two components. A proper example of
%vector interpolation is given in the interpolator-only mode example
%{\tt testinterp} (see section \ref{subsec_running_testinterp}).
%  The details of the vector treatment,
%performed by the routines \newline {\tt scriprmp\_vector.F90} and {\tt
%  rotations.F90} in {\tt oasis3/lib/scrip/src} are the following:

%\begin{itemize}
%\item If the angles of the source grid local coordinate system are
%  defined in the {\it grids.nc} data file (see section
%  \ref{subsec_griddata}), an automatic rotation from the local to the
%  geographic spherical coordinate system is performed.
%\item If the two source vector components are not defined on the same
%  source grid, one component is automatically interpolated on the grid
%  of the other component. Note that if the components are not given in
%  a Cartesian coordinate system, this operation is not exact as the
%  coordinate system is not fixed spatially (for grids not covering the 
%  North or South poles, the error is however 
%  small at the scale of the grid mesh).
%\item If the user puts the {\tt PROJCART} keyword at the end of the
%  {\tt SCRIPR} configuring line (see above), projection of the two
%  vector components in a Cartesian coordinate system, interpolation of
%  the resulting 3 Cartesian components, and projection back in the
%  spherical coordinate system are performed. In debug mode
%  (compilation with {\tt \_\_DEBUG} pre-compiling key), the resulting
%  vertical component in the spherical coordinate system after
%  interpolation is written to a file {\tt projection.nc}; as the
%  source vector is horizontal, this component should be very close to
%  0.
%\item If the user did not put the {\tt PROJCART} keyword at the end of 
%  the {\tt SCRIPR} configuring line, the two spherical coordinate system 
%  components are interpolated. Note that this operation is not exact as the
%  coordinate system is not fixed spatially (in most cases, the error is however 
%  small at the scale of the grid mesh).
%\item If the angles of the target grid local coordinate system are defined in the 
%{\it grids.nc} data file (see section \ref{subsec_griddata}), an automatic rotation 
%from the geographic spherical to the local coordinate system is performed.
%\item The first and second components of the interpolated vector field are then present 
%in the target fields associated respectively to the first and second source vector 
%component. The target grids for the two vector components can be different. 
%\end{itemize}

\item {\bf INTERP}: UNUSED

%-Note sur le traitement des points masqu�s dans OASIS:
%Note: Le champ interpol� fldnew est initialis� � z�ro. La valeur du 
%masque de la grille source n'est pas chang�e ni par masq.f ni par
%extrap.f.
%
%SURFMESH:
% -les points source masqu�s ne sont pas utilis�s.
% -le calcul des poids est fait pour tous les points cible mais les poids et
%  adresses des points masqu�s sont remis � 0 et 1.
% -le calcul du champ interpol� n'est fait que pour les points cible
%  non-masqu�s. 
%BILINEAR, BICUBIC, NNEIBOR:
% -tous les points source, m�me masqu�s, sont utilis�s (c'est pour �a
%  qu'il faut faire MASK et EXTRAP avant).
% -le calcul des poids est fait pour tous les points cibles 
%  masqu�s ou pas.
% -le calcul du champ interpol� est fait pour tous les points cibles 
%  masqu�s ou pas.
%GAUSSIAN: 
% -les points source masqu�s ne sont pas utilis�s.
% -le calcul des poids n'est fait que pour les points cible
% non-masqu�s.
% -le calcul du champ interpol� n'est fait que pour les points cible
% non-masqu�s. 
%
%  {\tt INTERP} gathers different techniques of interpolation controlled by 
%  routine {\tt fiasco.f}. The following interpolations are available:
%
%  \begin{itemize}
%
%   \item {\tt BILINEAR} performs a bilinear interpolation using 4
%    neighbours. 
%
%   \item {\tt BICUBIC} performs a bicubic interpolation.
%
%   \item {\tt NNEIBOR} performs a nearest-neighbour interpolation.
%
%  These three interpolations are performed by routines in {\tt
%  /oasis3/lib/fscint} and support only A, B,
%  G, L, Y, or Z grids (see appendix
%  \ref{subsec_gridtypes}). All sources grid points, masked or not, are
%  used in the calculation. To avoid the `contamination' by masked
%  source grid points, transformations {\tt MASK} and {\tt EXTRAP}
%  should be used. Values are calculated for all target grid points,
%  masked or not.
%
%  The configuring line is as follows:
%    \begin{verbatim}
% # BILINEAR or BICUBIC or NNEIBOR interpolation 
%    $CMETH $CGRS $CFTYP \end{verbatim} where 
%     \begin{itemize}
%     \item {\tt \$CMETH = BILINEAR}, {\tt BICUBIC} or {\tt NNEIBOR} 
%     \item {\tt \$CGRS} is the source grid type (A, B, G, L, Y, or
%     Z, see appendix \ref{subsec_gridtypes})
%     \item {\tt \$CFTYP} the field type ({\tt SCALAR} or {\tt
%  VECTOR}). {\tt VECTOR} has an effect for target grid points located
%  near the pole: the sign of a source value located on
%  the other side of the pole will be reversed. 
%     \end{itemize}
%
%   \item {\tt SURFMESH} (routines in {\tt /oasis3/lib/anaism}) is a
%     first-order conservative remapping from a fine to a coarse grid
%     (the source grid must be finer over the whole domain) and
%     supports only Lat-Lon grids (see appendix
%     \ref{subsec_gridtypes}). For a target grid cell, all the
%     underlying not masked source grid cells are found and the target
%     grid field value is the sum of the source grid field values
%     weighted by the overlapped surfaces. No value is assigned to
%     masked cells. Note that it is not recommended to use this
%     interpolation anymore, as the more general {\tt SCRIPR/CONSERV}
%     remapping is now available.  The configuring line is as follows:
%
%    \begin{verbatim}
% # SURFMESH remapping
%     $CMETH $CGRS $CFTYP $NID $NV $NIO
%    \end{verbatim} where 
%     \begin{itemize}
%     \item {\tt \$CMETH = SURFMESH}
%     \item {\tt \$CGRS} and {\tt \$CFTYP} are as for {\tt BILINEAR}
%     \item {\tt \$NID} is the identificator for the weight-address
%    dataset in all the different {\tt INTERP/SURFMESH} datasets in the
%    present coupling.  This dataset will be calculated by OASIS3 if
%    {\tt \$NIO}= 1, or only read if {\tt \$NIO}= 0.
%    \item {\tt \$NV} is the maximum number of source grid meshes
%    used in the remapping.
%    \end{itemize} 
%
%  \item {\tt GAUSSIAN} (routines in {\tt /oasis3/lib/anaisg}) is a
%    gaussian weighted nearest-neighbour interpolation technique.  The
%    user can choose the variance of the function and the number of
%    neighbours considered. The {\tt \$NV} non masked nearest source
%    grid points are automatically used and no value are calculated for
%    masked target grid points.
%
%    The configuring line is:
%    \begin{verbatim}
%    # GAUSSIAN interpolation
%    $CMETH $CGRS $CFTYP $NID $NV $VAR  $NIO
%    \end{verbatim}where 
%    \begin{itemize}
%    \item {\tt \$CMETH = GAUSSIAN}
%    \item {\tt \$CGRS} is the source grid type; all grids are
%      supported\footnote{Note that for ``U" grids, the average
%        distance between two source grid point $\overline{d}^2$, used
%        in the calculation of the Gaussian weighting function, is not
%        exact but this has only a limited impact as this value is in
%        all cases multiplied by the \$VAR value defined by the user
%        (see {\tt SCRIPR/GAUSWGT} above).}
%    \item {\tt \$CFTYP} is the field type {\tt SCALAR} or {\tt
%        VECTOR}.
%    \item {\tt \$NID} is the identificator for the
%    weight-address dataset in all the different {\tt INTERP/GAUSSIAN}
%    datasets in the present coupling. This weight-address dataset will
%    be calculated by OASIS3 if {\tt \$NIO}= 1, or only read if {\tt
%    \$NIO}= 0. 
%    \item {\tt \$NV} is the number of neighbours used in the
%    interpolation. 
%    \item {\tt \$VAR} is as for {\tt SCRIPR/GAUSWGT} (see above).
%    \end{itemize}
%\end{itemize}

\item {\bf MOZAIC}: UNUSED

%{\tt MOZAIC} performs the mapping of a field from a source to a target
%grid. The grid-mapping dataset, i.e. the weights and addresses of the
%source grid points used to calculate the value of each target grid
%point are defined by the user in a file (see section
%\ref{subsec_transformationdata}).  The configuring line is:
%\begin{verbatim}
%# MOZAIC operation
%     $CFILE  $NUMLU  $NID  $NV
%\end{verbatim}where 
%\begin{itemize}
%\item{\tt \$CFILE} and {\tt \$NUMLU} are the grid-mapping file name
%and associated logical unit on which the grid-mapping dataset is going
%to be read),
%\item {\tt \$NID} the identificator for this grid-mapping dataset in
%all {\tt MOZAIC} grid-mapping datasets in the present coupling
%\item {\tt \$NV} is the maximum number of target grid points use
%in the mapping.
%\end{itemize}
%
%\item {\bf NOINTERP}: 
%
%{\tt NOINTERP} is the analysis that has to be chosen when no other
%transformation from the interpolation class is chosen.
%There is no configuring line.

\item {\bf FILLING}: UNUSED

%{\tt FILLING} (routine {\tt oasis3/src/filling.f})
%performs the blending of a regional data set with a climatological
%global one for a Sea Surface Temperature (SST) or a Sea Ice Extent
%field. This occurs when coupling a non-global ocean model with a
%global atmospheric model. {\tt FILLING} can only handle
%fields on Logically Rectangular grid (LR, but also A, B,
%G, L, Y, and Z grids, see section \ref{subsec_gridtypes}.
%
%The global data set has to be a set of SST
%given in Celsius degrees (for the filling of a Sea Ice Extent field,
%the presence or absence of ice is deduced from the value of the
%SST). The frequency of the global set can be interannual monthly,
%climatological monthly or yearly.
%
%The blending can be smooth or abrupt. If the blending is abrupt, only
%model values are used within the model domain, and only the global
%data set values are used outside. If the blending is smooth, a linear
%interpolation is performed between the two fields within the model
%domain over narrow bands along the boundaries. The linear
%interpolation can also be performed giving a different weight to the
%regional or and global fields. 
%
% The smoothing is defined by parameters in {\tt
%oasis3/src/mod\_smooth.F90}. The lower smoothing band
%in the global model second dimension is defined by {\em nsltb}
%(outermost point) and {\em nslte} (innermost point); the upper
%smoothing band in the global model second dimension is defined by {\em
%nnltb} (outermost point) and {\em nnlte} (innermost point). The
%parameter {\em qalfa} controls the weights given to the regional and
%to the global fields in the linear interpolation. {\em qalfa} has to
%be $1/(nslte-nsltb)$ or $1/(nnltb-nnlte)$. For the outermost points
%({\em nsltb} or {\em nnltb}) in the smoothing band, the weight given
%to the regional and global fields will respectively be 0 and 1; for
%the innermost points ({\em nslte} or {\em nnlte}) in the smoothing
%band, the weight given to the regional and global fields will
%respectively be 1 and 0; within the smoothing band, the weights will
%be a linear interpolation of the outermost and innermost weights.
%
%The smoothing band in the global model first dimension will be a band
%of {\em nliss} points following the coastline. To calculate this band,
%OASIS3 needs {\em nwlgmx}, the greater first dimension index of the
%lower coastline and {\em nelgmx}, the smaller first dimension index on
%the upper coastline. The parameter {\em qbeta} controls the weights
%given to the regional and to the global fields in the linear
%interpolation. {\em qbeta} has to be $1/(nliss-1)$. The weights given
%to the regional and global fields in the global model first dimension
%smoothing bands will be calculated as for the second dimension.
%
%The user must provide the climatological data file with
%a specific format described in \ref{subsec_transformationdata}. 
%When one uses {\tt FILLING} for SST with smooth blending, thermodynamics 
%consistency requires to modify the heat fluxes over the blending 
%regions. The correction term is proportional to the difference between
%the blended SST and the original SST interpolated on the atmospheric 
%grid and can be written out on a file to be read later, for analysis
%{\tt CORRECT} for example. In that case, the symbolic name of the flux
%correction term read through the input file {\it namcouple} must 
%correspond in {\tt FILLING} and {\tt CORRECT} analyses.
%
%In case the regional ocean model includes a coastal part or islands, a
%sea-land mask mismatch may occur and a coastal point correction can be
%performed if the field has been previously interpolated with {\tt
%INTER/SURFMESH}. In fact, the
%mismatch could result in the atmosphere undesirably ``seeing''
%climatological SST's directly adjacent to ocean model SST's.  Where
%this situation arises, the coastal correction consists in identifying
%the suitable ocean model grid points that can be used to extrapolate
%the field, excluding climatological grid points.
%
%This analysis requires one configuring line with 3, 4 or 6 arguments.
%
%\begin{enumerate}
%\item If FILLING performs the blending of a regional data set with a
%global one for the Sea Ice Extent, the 3-argument input line is:
%\begin{verbatim}
%# Sea Ice Extent FILLING operation
%     $CFILE  $NUMLU  $CMETH\end{verbatim} where {\tt \$CFILE} is 
%the file name for the global data set, {\tt \$NUMLU} the associated
%logical unit. {\tt \$CMETH}, the {\tt FILLING} technique, is a {\tt
%CHARACTER*8} variable: the first 3 characters are either {\tt SMO},
%smooth filling, or {\tt RAW}, no smoothing ; the next three characters
%must be {\tt SIE} for a Sea Ice Extent filling operation; the last two
%define the time characteristics of the global data file, respectively
%{\tt MO}, {\tt SE} and {\tt AN} for interannual monthly,
%climatological monthly and yearly. Note that in all cases, the global
%data file has to be a Sea Surface Temperature field in Celsius
%degrees.
%
%\item If FILLING performs the blending of a regional data set with a
%global one for the Sea Surface Temperature without any smoothing, the
%4-argument input line is:
%\begin{verbatim}
%#Sea Surface Temperature FILLING operation without smoothing
%     $CFILE  $NUMLU  $CMETH  $NFCOAST\end{verbatim} where 
%{\tt \$CFILE}, {\tt \$NUMLU} are as for the SIE filling. In this
%case however, {\tt \$CMETH(1:3) = RAW}, {\tt \$CMETH(4:6) = SST}, 
%and the last two characters define the time characteristics of the 
%global data file, as for the SIE filling. {\tt \$NFCOAST} is  
%the flag for the calculation of the coastal correction ( 0 no, 1 yes).
%
%\item If FILLING performs the blending of a regional data set with a
%global one for the Sea Surface Temperature with smoothing, the
%6-argument input line is:
%\begin{verbatim}
%#Sea Surface Temperature FILLING operation with smoothing
%     $CFILE  $NUMLU  $CMETH  $NFCOAST  $CNAME  $NUNIT\end{verbatim}
%     where {\tt \$CFILE}, {\tt \$NUMLU} and {\tt \$NFCOAST} are as for the SST
%filling without smoothing. In this case, {\tt \$CMETH(1:3) = SMO},
%{\tt \$CMETH(4:6) = SST}, and the last two characters define the
%time characteristics of the global data file, as for the SIE
%filling. {\tt \$CNAME} is the symbolic name for the correction
%term that is calculated by OASIS3 and {\tt \$NUNIT} the logical
%unit on which it is going to be written. 
%
%\end{enumerate}

\end{itemize}

%subsection{The interpolation}

\section{The ``cooking'' stage}
\label{subsec_cooking}

The following transformations are available in the ``cooking'' part of
OASIS3.

\begin{itemize}

\item {\bf CONSERV}: 

  {\tt CONSERV} ensures a global modification of the coupling field. 
  This analysis requires one input line with one argument and one optional
  argument:

 \begin{verbatim}
# CONSERV operation
     $CMETH  $CONSOPT
  \end{verbatim}where: 
  \begin{itemize}
  \item {\tt \$CMETH} is the method desired with the following choices
   \begin{itemize}
   \item with {\tt \$CMETH = GLOBAL}, the field is integrated on both
     source and target grids, without considering values of masked
     points, and the residual (target - source) is uniformly
     distributed on the target grid; this option ensures global
     conservation of the field
   \item with {\tt \$CMETH = GLBPOS}, the same operation is performed
     except that the residual is distributed proportionally to the
     value of the original field; this option ensures the global
     conservation of the field and does not change the sign of the field
   \item with {\tt \$CMETH = BASBAL}, the operation is analogous to
     {\tt GLOBAL} except that the non masked surface of the source and
     the target grids are taken into account in
     the calculation of the residual; this option does not ensure
     global conservation of the field but ensures that the energy received is
     proportional to the non masked surface of the target grid
   \item with {\tt \$CMETH = BASPOS}, the non masked surface of the
     source and the target grids are taken into
     account and the residual is distributed proportionally to the
     value of the original field; therefore, this option does not
     ensure global conservation of the field but ensures that the energy received
     is proportional to the non masked surface of the target grid and 
     it does not change the sign of the field.
   \end{itemize}
  \item {\tt \$CONSOPT} is an optional argument specifying the algorithm.  
  {\tt \$CONSOPT} can be bfb or opt.
  The bfb option enforces a bit-for-bit transformation regardless of the
  grid decomposition or process count.  The opt option carries out the conservation using an
  optimal algorithm using less memory and a faster approach.  bfb is the
  default setting.
  \end{itemize}

{\it Note that for this operation to be correct, overlapping grid cells 
on the source grid or on the target grid must be masked.} 

\item {\bf SUBGRID}: UNUSED

%{\tt SUBGRID} can be used to interpolate a field from a coarse grid to a
%finer target grid (the target grid must be finer over the whole
%domain). Two types of subgrid interpolation can be performed,
%depending on the type of the field.
%
%For solar type of flux field ({\tt \$SUBTYPE = SOLAR}), the operation
%performed is:
%$$\Phi_{i} = \frac{1-\alpha_i}{1-\alpha} F$$ where $\Phi_{i}$ ($F$) is
%the flux on the fine (coarse) grid, $\alpha_i$ ($\alpha$) an auxiliary
%field on the fine (coarse) grid (e.g. the albedo).  The whole
%operation is interpolated from the coarse grid with a grid-mapping type
%of interpolation; the dataset of weights and addresses has to be given
%by the user.
%
%For non-solar type of field ({\tt \$SUBTYPE = NONSOLAR}), a
%first-order Taylor expansion of the field on the fine grid relatively
%to a state variable is performed (for instance, an expansion of the
%total heat flux relatively to the SST):
%$$\Phi_{i} = F + \frac{\partial F}{\partial T} ( T_i - T ) $$ where
%$\Phi_{i}$ ($F$) is the heat flux on the fine (coarse) grid, $T_i$
%($T$) an auxiliary field on the fine (coarse) grid (e.g. the SST) and
%$\frac{\partial F}{\partial T}$ the derivative of the flux versus the
%auxiliary field on the coarse grid. This operation is interpolated
%from the coarse grid with a grid-mapping type of interpolation; the
%dataset of weights and addresses has to be given by the user.
%
%This analysis requires one input line with 7 or 8
%arguments depending on the type of subgrid interpolation.
%
%\begin{enumerate}
%\item If the the {\tt SUBGRID} operation is performed on a solar flux,
%the 7-argument input line is:
%\begin{verbatim}
%# SUBGRID operation with $SUBTYPE=SOLAR 
%  $CFILE  $NUMLU  $NID  $NV  $SUBTYPE  $CCOARSE $CFINE\end{verbatim}where
%{\tt \$CFILE} and {\tt \$NUMLU} are the subgrid-mapping file name and
%associated logical unit (see section \ref{subsec_transformationdata} for the
%structure of this file); {\tt \$NID} the identificator for this
%subgrid-mapping dataset within the file build by OASIS based on all
%the different {\tt SUBGRID} analyses in the present coupling; {\tt
%\$NV} is the maximum number of target grid points use in the
%subgrid-mapping; {\tt \$SUBTYPE = SOLAR} is the type of subgrid
%  interpolation; {\tt \$CCOARSE} is the
%auxiliary field name on the coarse grid (corresponding to $\alpha$)
%and {\tt \$CFINE} is the auxiliary field name on fine grid
%(corresponding to $\alpha_i$).
%These two fields needs to be exchanged between their original model
%and OASIS3 main process, at least as {\tt AUXILARY} fields.
%This analysis is performed from the coarse grid with a grid-mapping type
%of interpolation based on the {\tt \$CFILE} file.
%
%\item If the the SUBGRID operation is performed on a nonsolar flux,
%the 8-argument input line is:
%\begin{verbatim}
%# SUBGRID operation with $SUBTYPE=NONSOLAR
%  $CFILE $NUMLU $NID $NV $SUBTYPE $CCOARSE $CFINE $CDQDT
%\end{verbatim} where {\tt \$CFILE},  {\tt \$NUMLU},  {\tt \$NID},  
%{\tt \$NV} are as for a solar subgrid interpolation; {\tt
%\$SUBTYPE = NONSOLAR}; {\tt \$CCOARSE} is the auxiliary
%field name on the coarse grid (corresponding to $T$) and {\tt \$CFINE}
%is the auxiliary field name on fine grid (corresponding to $T_i$); the
%additional argument {\tt \$CDQDT} is the coupling ratio on the coarse
%grid (corresponding to $\frac{\partial F}{\partial T}$) These three
%fields need to be exchanged between their original model and OASIS3
%main process as {\tt AUXILARY} fields. This operation is performed from the
%coarse grid with a grid-mapping type of interpolation based on the {\tt
%\$CFILE} file.
%
%\end{enumerate}

\item {\bf BLASNEW}:
 
{\tt BLASNEW} performs a scalar
multiply or scalar add to any destination field.  This is the equivalent
of BLASOLD on the destination side.  The prior feature that supported
linear combinations of the
current coupling field with any other fields after the
interpolation has been deprecated.

This analysis requires the same input line as {\tt BLASOLD}.

\item {\bf MASKP}: UNUSED

%A new analysis {\tt MASKP} can be used to mask the fields after
%interpolation. {\tt MASKP} has the same generic input line as {\tt MASK}.

\end{itemize}

%subsection{The ``cooking'' stage}

\section{The post-processing}
\label{subsec_postpro}

The following analyses are available in the post-processing part of
OASIS3.

\begin{itemize}

\item {\bf REVERSE}: UNUSED

% {\it This transformation is obsolete in the current OASIS version.}

%{\tt REVERSE} (routine {\tt oasis3/src/reverse.f})
%reorders a field. 

%This analysis requires the same input line as {\tt INVERT}, with {\tt
%  \$CORLON} and {\tt \$CORLAT} being now the resulting orientation.
%{\tt REVERSE} does not work for U and D grids (see appendix
%\ref{subsec_gridtypes}). Note that {\tt INVERT} does not transform the
%associated grid or mask.

\item {\bf CHECKOUT}: 

{\tt CHECKOUT} calculates
the mean and extremum values of an output field and prints them to the
oasis output files (this operation does not transform the field).

The generic input line is as for {\tt CHECKIN} (see above).

\item {\bf GLORED}: UNUSED

% {\it This transformation is obsolete in the current OASIS version as
% coupling fields can be directly interpolated to a target Reduced
% grid, if needed; this transformation should not be used anymore.} 
%
%{\tt GLORED} performs a linear interpolation of field from a full
%Gaussian grid to a Reduced grid. When present, {\tt GLORED} must be the last
%analysis performed.
%
%Before doing the interpolation, non-masked values are automatically
%extrapolated to masked points with {\tt EXTRAP/NINENN} method (see
%above); to do so, the masked grid points are first replaced with a
%predefined value.  The required global grid mask must be
%present in data file {\tt masks} or {\tt masks.nc} (see section
%\ref{subsec_griddata}).
%
%The generic input line is as follows:
%\begin{verbatim}
%# GLORED operation
%     $NNBRLAT  $NV   $NIO   $NID\end{verbatim} where {\tt \$NNBRLAT} 
%is as for {\tt REDGLO} (see {\tt REDGLO} description above). 
%The next 3 parameters refer to the {\tt EXTRAP/NINENN} extrapolation
%(see {\tt EXTRAP/NINENN} description above).  The value assigned to
%all land points before interpolation is given by {\tt amskred} in {\tt
%oasis3/src/blkdata.f}; as for the {\tt \$VALMASK} in
%{\tt MASK} analysis, it has to be chosen well outside the range of
%your field values but not too large to avoid any representation problem. 
%

\end{itemize}


