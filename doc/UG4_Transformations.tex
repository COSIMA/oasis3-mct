\newpage
\chapter{Transformations and interpolations}
\label{sec_transformations}

Different transformations and 2D interpolations are available in
OASIS3-MCT to adapt the coupling fields from a source model grid to a target
model grid. 
In the following paragraphs, a description of each
transformation with its corresponding configuring lines is given.
Features that are now deprecated (non functional) compared
to prior versions will be noted with the string UNUSED
but not described.

\section{Time transformations}
\label{subsec_timetrans}

\begin{itemize}

\item {\bf LOCTRANS}:

{\tt LOCTRANS} requires one configuring line on which a time
transformation, automatically performed below the call to
{\tt oasis\_put} or {\tt
  prism\_put\_proto}, should be indicated:

  \begin{verbatim}
 # LOCTRANS operation
   $TRANSFORM
  \end{verbatim}
%$
\vspace{-0.5cm} 
where {\tt \$TRANSFORM} can be

  \begin{itemize}
    \item {\tt INSTANT}: no time transformation, the instantaneous field is
    transferred; 
    \item {\tt ACCUMUL}: the field accumulated over the previous coupling
    period is exchanged (the accumulation is simply done over the arrays
    {\tt field\_array}  provided as third argument to the {\tt
      oasis\_put} or {\tt
    prism\_put\_proto} calls, not weighted by the time interval
    between these calls);
    \item {\tt AVERAGE}: the field averaged over the previous coupling
    period is transferred (the average is simply done over the arrays
    {\tt field\_array} provided as third argument to the {\tt
      oasis\_put} or {\tt
    prism\_put\_proto} calls, not weighted by the time interval
    between these calls);
    \item {\tt T\_MIN}: the minimum value of the field
    for each source grid point over the previous coupling period is
    transferred; 
    \item {\tt T\_MAX}: the maximum value of the field for each source grid
    point over the previous coupling period is transferred;
    \item {\tt ONCE}: only one {\tt oasis\_put} / {\tt prism\_put\_proto} or
      {\tt oasis\_get} / {\tt
prism\_get\_proto} will be performed; this is equivalent to giving the
length of the run as coupling or I/O period. 
  \end{itemize}

With OASIS3-MCT, time transformations are supported more generally
with use of the coupling restart file.  The coupling restart file allows the partial
time transformation to be saved at the end of a run for exact
restart at the start of the next run.  For that reason, coupling restart
filenames are now required for all {\it namcouple} transformations that use
LOCTRANS (with non INSTANT values).  In this mode, OASIS3-MCT will exit
gracefully with an error message if a restart filename is not provided.
This is the reason an optional restart file is now provided on the
OUTPUT {\it namcouple} input line.
\end{itemize}

\section{The pre-processing transformations}
\label{subsec_preproc}

\begin{itemize}

\item {\bf REDGLO} UNUSED

\item {\bf INVERT}: UNUSED

\item {\bf MASK}: UNUSED
 
\item {\bf EXTRAP}: UNUSED

\item {\bf CHECKIN}:

 {\tt CHECKIN} calculates the global minimum, the maximum and the sum of the
 the source field values and prints them to the OASIS3-MCT debug file (for the master process of the source component model only). This operation does not transform the field.

 The generic input line is as follows, even if {\tt \$NINT} has no impact in OASIS3-MCT:
 \begin{verbatim}
 # CHECKIN operation
     $INT = $NINT  
 \end{verbatim} 

\item {\bf CORRECT}: UNUSED

\item {\bf BLASOLD}: 

{\tt BLASOLD} allows the source field to be scaled and allows
a scalar to be added to the field.  The prior ability to perform 
a linear combination of the current coupling field with other coupling fields 
has been deprecated in OASIS3-MCT.  This transformation occurs before the interpolation {\it per se}.

This transformation requires at least one configuring line with two
parameters:
 \begin{verbatim}
# BLASOLD operation
     $XMULT   $NBFIELDS 
 \end{verbatim}
\vspace{-0.5cm}  
where {\tt \$XMULT} is the multiplicative coefficient of the source
field. {\tt \$NBFIELDS} must be 0 if no scalar needs to be added or 1
if a scalar needs to be added. In this last case, an additional input
line is required where {\tt \$AVALUE} is the scalar to be added to the
field :
\begin{verbatim}
     CONSTANT  $AVALUE
\end{verbatim} 
\end{itemize}

%subsection{The pre-processing transformations}

\section{The remapping (interpolation)}
\label{subsec_interp}

\begin{itemize}

\item {\bf MAPPING}: 

  The {\tt MAPPING} keyword is used to specify an input file to be
  read and used for mapping (ie. regridding or interpolation); the {\tt MAPPING} file must follow the {\tt SCRIPR} format. 
  This is an alternative method to {\tt SCRIPR} for setting the mapping file.
 
  In the current implementation, each pair of source and target points in the {\tt MAPPING} file can be linked by only one weight, i.e. remappings such as {\tt SCRIPR/BICUBIC} involving at each source grid point the value of the field, of the gradients and the cross-gradient, or second-order conservative remapping are not supported.   

This transformation requires at least one configuring line with one
filename and two optional string values:
\begin{verbatim}
     $MAPNAME  $MAPLOC  $MAPSTRATEGY
\end{verbatim}
  \begin{itemize}
  \item {\tt \$MAPNAME} is the name of the mapping file to read.  This
    is a netcdf file consistent with the SCRIPR map file format (see section
  \ref{subsec_inputdata}).

  \item {\tt \$MAPLOC} is optional and can be either {\tt src} or {\tt dst}.  With {\tt src}, the mapping will be done
  in parallel on the source processors before communication to the destination model and processors; this is the default.   With {\tt dst}, the mapping is 
  done on the destination processors after the data is sent from the source
  model on the source grid. 

  \item {\tt \$MAPSTRATEGY} is optional and can be either {\tt bfb}, {\tt sum}, or {\tt opt}.  In {\tt bfb} mode, the mapping is
  done using a strategy that produces bit-for-bit identical results regardless
  of the grid decompositions without leveraging a partial sum computation.  With
  {\tt sum}, the transform is done using the partial sum approach which generally
  introduces roundoff level changes in the results on different processor
  counts. Option {\tt opt} allows the coupling layer to choose either
  approach based on an analysis of which strategy is likely to run
  faster. Usually, partial sums will be used if the source grid has a higher resolution
  than the target grid as this should reduce the overall
  communication. 

  \end{itemize}

Note that if {\tt SCRIPR} (see below) is used to calculate the remapping file, {\tt MAPPING} can still be listed in the {\tt namcouple} to specify a name for the remapping file generated by {\tt SCRIPR} different from the default and/or to specify a {\tt \$MAPLOC} or {\tt \$MAPSTRATEGY} option.

\item {\bf SCRIPR}: 
 
  {\tt SCRIPR} gathers the interpolation techniques offered by Los
  Alamos National Laboratory SCRIP 1.4 library \citep{Jones99}\footnote{See also http://climate.lanl.gov/Software/SCRIP/ and the
    copyright statement in appendix \ref{sec_SCRIP}.}.
  {\tt SCRIPR} routines are in {\tt oasis3-mct/lib/scrip}. See the SCRIP 1.4
  documentation in {\tt oasis3/doc/SCRIPusers.pdf} for more
  details on the interpolation algorithms.  
  In the current implementation, each pair of source and target points
  in the {\tt MAPPING} file can be linked by only one weight,
  i.e. remappings such as {\tt SCRIPR/BICUBIC} involving at each
  source grid point the value of the field, of the gradients and the
  cross-gradient, or second-order conservative {\tt
    SCRIPR/CONSERV/SECOND} remapping are not
  supported.  

When the SCRIP library performs a remapping, it first checks if the
file containing the corresponding remapping weights and addresses
exists. If it exists, it reads them from the file; if not, it calculates
them and store them in a file. The file is created in the working
directory and is called {\tt
  rmp\_{\it srcg}\_to\_{\it tgtg}\_{\it INTTYPE}\_{\it NORMAOPT}.nc}, where {\it srcg} and
{\it tgtg} are the acronyms of respetively the source and the target
grids, {\it INTTYPE} is the interpolation type, i.e. {\tt DISTWGT},
{\tt GAUSWGT}, {\tt BILINEAR} ({\bf not BILINEA as in OASIS3.3}) or
{\tt CONSERV} -see below, and
{\it NORMAOPT} is the normalization option, i.e. {\tt DESTAREA},
{\tt FRACAREA} or {\tt FRACNNEI} for {\tt CONSERV} only -see
below). One has to take care that the remapping file will have the same name
even if other details, like the grid masks, are changed. When reusing
a remapping file, one has to be sure that it was generated in exactly
the same conditions than the ones it is used for.  

  The following types of interpolations are available:

  \begin{itemize}

  \item {\tt DISTWGT} performs a distance weighted nearest-neighbour
    interpolation (N neighbours). All types of grids are supported. 

     \begin{itemize}

     \item Masked target grid points: the zero value is associated to
       masked target grid points.

     \item Non-masked target grid points having some of the N source
       nearest neighbours masked: a nearest neighbour algorithm using
       the remaining non masked source nearest neighbours is applied.

     \item Non-masked target grid points having all of the N source
       nearest neighbours masked: by default, the nearest non-masked
       source neighbour is used (logical {\tt ll\_nnei} hard-coded to
       {\tt .true.} in {\tt oasis3-mct/lib/scrip/src/remap\_distwgt.F};
       same default behaviour as OASIS3.3). 

     \end{itemize}

  The configuring line is:

  \begin{verbatim}
 # SCRIPR (for DISWGT) 
     $CMETH $CGRS $CFTYP $REST $NBIN $NV $ASSCMP $PROJCART
  \end{verbatim} where:
\vspace{-0.5cm} 
  \begin{itemize} 
  \item {\tt \$CMETH = DISTWGT}. 
  \item {\tt \$CGRS} is the source grid
  type ({\tt LR}, {\tt D} or {\tt U})- see annexe
  \ref{subsec_gridtypes}. 

  \item {\tt \$CFTYP} is the field type: {\tt SCALAR}. The option {\tt
    VECTOR}, which in fact leads to a scalar treatment of the field
    (as in the previous versions), is still accepted. {\bf
      VECTOR\_I or VECTOR\_J, i.e. vector fields, are not
    supported anymore in OASIS3-MCT.}. See ``Support of vector fields
    with the SCRIPR remappings'' below.

  \item {\tt \$REST} is the search restriction type: {\tt LATLON}
  or {\tt LATITUDE} (see SCRIP 1.4 documentation SCRIPusers.pdf).
  \item {\tt \$NBIN} the number of restriction bins (see SCRIP 1.4
  documentation SCRIPusers.pdf). Note that for D or U grid, the restriction may influence sligthly the result near the borders of the restriction bins.  
  \item {\tt \$NV} is the number of neighbours used.
  \item {\tt \$ASSCMP}: UNUSED;  {\bf vector fields are not
    supported anymore in OASIS3-MCT.} See ``Support of vector fields
    with the SCRIPR remappings'' below.
  \item {\tt \$PROJCART}: UNUSED;  {\bf vector fields are not
    supported anymore in OASIS3-MCT.} See ``Support of vector fields
    with the SCRIPR remappings'' below.
  \end{itemize}

\item {\tt GAUSWGT} performs a N nearest-neighbour interpolation
  weighted by their distance and a gaussian function. All grid types
  are supported.  
  \begin{itemize}  

  \item Masked target grid points: the zero value is associated to masked
  target grid points.

  \item Non-masked target grid points having some of the N source
  nearest neighbours masked: a nearest neighbour algorithm using the
  remaining non masked source nearest neighbours is applied.

  \item Non-masked target grid points having all of the N source
       nearest neighbours masked: by default, the nearest non-masked
       source neighbour is used, i.e. logical {\tt ll\_nnei} hard-coded to
       {\tt .true.} in {\tt oasis3-mct/lib/scrip/src/remap\_gauswgt.F};
       {\bf this is NOT the same default behaviour as OASIS3.3}; to
       have the same default behaviour as in OASIS3.3, put {\tt ll\_nnei=.false.}. 
  \end{itemize}

  The configuring line is:
  \begin{verbatim}
 # SCRIPR (for GAUSWGT)
     $CMETH  $CGRS  $CFTYP  $REST  $NBIN  $NV $VAR
  \end{verbatim}
\vspace{-0.5cm} 
where:
%$
  all entries are as for  {\tt DISTWGT}, except that:
  \begin{itemize} 
   \item {\tt \$CMETH = GAUSWGT}
   \item {\tt \$VAR}, which must be given as a
    REAL value (e.g 2.0 and not 2), defines the weight given to a
    neighbour source grid point as
    proportional to $exp(-1/2 \cdot d^2/\sigma^2)$ where $d$ is the
    distance between the source and target grid points, and $\sigma^2 =
    \$VAR \cdot \overline{d}^2$ where $\overline{d}^2$ is the average
    distance between two source grid points (calculated automatically
    by OASIS3-MCT).
  \end{itemize}

  \item {\tt BILINEAR} performs an interpolation based on a local bilinear approximation
    (see details
    in chapter 4 of SCRIP 1.4 documentation SCRIPusers.pdf)

  For {\tt BILINEAR}, Logically-Rectangular (LR) and
  Reduced (D) source grid types are supported.

  \begin{itemize}  
  \item Masked target grid points: the zero value is associated to masked
  target grid points.

  \item Non-masked target grid points having some of the source points
  normally used in the bilinear or bicubic interpolation masked: a N
  nearest neighbour algorithm using the remaining non masked source
  points is applied.

 \item Non-masked target grid points having all
       bilinear neighbours masked: by default, the nearest non-masked
       source neighbour is used (logical {\tt ll\_nnei} hard-coded to
       {\tt .true.} in {\tt oasis3-mct/lib/scrip/src/remap\_bilinear.F};
       {\bf this is NOT the same default behaviour as OASIS3.3}; to
       have the same default behaviour as in OASIS3.3, put {\tt ll\_nnei=.false.}. 
  \end{itemize} 
 
  The configuring line is:

  \begin{verbatim}
 # SCRIPR  (for BILINEAR)
     $CMETH  $CGRS  $CFTYP  $REST  $NBIN
  \end{verbatim}
\vspace{-0.5cm} 
where:
%$
  \begin{itemize}
  \item {\tt \$CMETH = BILINEAR}
  \item {\tt \$CGRS} is the source grid type (LR or D)
  \item {\tt \$CFTYP}, {\tt \$NBIN} are
  as for {\tt DISTWGT}. 
  \item {\tt \$REST} is as for {\tt DISTWGT}, except that only
  {\tt LATITUDE} is possible for a Reduced (D) source grid.
  \end{itemize}
 
  \item {\tt BICUBIC}: {\bf as opposed to OASIS3.3, OASIS3-MCT does not
    currently support this mapping option because it is higher
    order.} 

  \item {\tt CONSERV} performs 1st order conservative remapping,
  which means that the weight of a source cell is proportional to area
  intersected by the target cell.  {\bf Note that 2nd order conservative mapping
  is not supported yet with OASIS3-MCT}.

  The configuring line is:
  \begin{verbatim}
 # SCRIPR (for CONSERV)
     $CMETH  $CGRS  $CFTYP  $REST  $NBIN  $NORM  $ORDER 
  \end{verbatim}
%$
\vspace{-0.5cm} 
where: 
  \begin{itemize}
  \item {\tt \$CMETH = CONSERV} 
  \item {\tt \$CGRS} is the source grid type: LR, D and U are
  supported for first-order remapping if the grid corners are given by
  the user in the grid data file {\tt grids.nc} ; only LR
  is supported if the grid corners are not available in {\tt grids.nc} 
  and therefore have to be calculated automatically by
  OASIS3. 
%For second-order remapping, only LR is supported because the
%  gradient of the coupling field used in the transformation has to be
%  calculated automatically by OASIS3.
  \item {\tt \$CFTYP, \$REST}, {\tt \$NBIN} are as for {\tt DISTWGT}. 
  \item {\tt \$NORM} is the NORMalization option:
  \begin{itemize}
   \item {\tt FRACAREA}: The sum of the non-masked source cell intersected areas
    is used to NORMalise each target cell field value: the flux is not
    locally conserved, but the flux value itself is reasonable.
   \item {\tt DESTAREA}: The total target cell area is used to NORMalise
    each target cell field value even if it only partly intersects
    non-masked source grid cells: local flux conservation is ensured,
    but unreasonable flux values may result.
   \item {\tt FRACNNEI}: as {\tt FRACAREA}, except that at least the
    source nearest unmasked neighbour is used for unmasked target
    cells that intersect only masked source cells. 
%Note that a zero value
%    will be assigned to a target cell that does not intersect any source
%    cells (masked or unmasked), even with FRACNNEI option.
  \end{itemize} 
  \item {\tt \$ORDER}: {\tt FIRST} for first order remapping respectively (see SCRIP 1.4 documentation). {\bf Note that 2nd order conservative mapping
  is not supported yet with OASIS3-MCT.}
%  Note that {\tt CONSERV/SECOND} is not positive definite and has not been 
%  fully validated yet.

\end{itemize}

\end{itemize}

{\bf Precautions related to the use of the SCRIPR/CONSERV remapping}

\begin{itemize}

\item For the 1st order conservative remapping: the weight of a source
  cell is proportional to area of the source cell intersected by
  target cell.  Using the divergence theorem, the SCRIP library
  evaluates this area with the line integral along the cell borders
  enclosing the area. As the real shape of the borders is not known
  (only the location of the 4 corners of each cell is known), the
  library assumes that the borders are linear in latitude and
  longitude between two corners.  This assumption becomes
  less valid closer to the pole and for latitudes above the {\tt
    north\_thresh} or below the {\tt south\_thresh} values (see 
{\tt oasis3-mct/lib/scrip/remap\_conserv.F}, the library evaluates
  the intersection between two border segments using a Lambert
  equivalent azimuthal projection. Problems were observed in some
  cases for the grid cell located around this {\tt north\_thresh} or
  {\tt south\_thresh} latitude.

\item Another limitation of the SCRIP 1st order conservative remapping 
  algorithm is that is also supposes, for
  line integral calculation, that $sin(latitude)$ is linear in
  longitude on the cell borders which again is in general not valid
  close to the pole. 
  %A projection or at least a normalization by the
  %true area of the cells (i.e. by the areas as considered by the
  %component models) is needed.

 \item For a proper consevative remapping, the corners of a cell have
   to coincide with the corners of its neighbour cell, with no
   ``holes'' between the cells.
  
\item  If two cells of one same grid overlay, at least the one with the
    greater numerical index must be masked for a proper conservative remapping.  
    For example, if the grid cells with i=1 overlays the grid cells
    with i=imax, the latter must be masked.  If this
    is not the case given the mask defined in {\it masks.nc}, OASIS3-MCT must 
    be compiled with the CPP key {\tt TREAT\_OVERLAY} which will ensure 
    that these rules are respected. This CPP key was introduced in 
    OASIS3.3.
      
\item A target grid cell intersecting no source cell (either 
    masked or non masked) at all i.e. falling in a ``hole'' of the source grid
    will in all cases get a zero value. 
    
\item If a target grid cell intersects only masked source cells, 
    it will still get a zero value unless the {\tt FRACNNEI}
    normalisation option is used, in which case it will get the
    nearest non masked neighbour value. {\bf Note that the option of
      having the value 1.0E+20 assigned to these target grid cell
      intersecting only masked source cells (for easier
      identification) is not yet availble in OASIS3-MCT.}
   

%     The target grid mask is never considered in {\tt CONSERV}, except
%     with normalisation option {\tt FRACNNEI} (see below). To have a
%     value calculated, a target grid cell must intersect at least one
%     source cell. However, the NORMlisation option (that takes into
%     account the source grid mask, see below) may result in a null
%     value calculated for those target grid cells. In that case (i.e.
%     at least one intersecting source cell, but a null value finally
%     calculated because of the normalisation option), the value 1.0E+20
%     is assigned to those target grid points if {\tt
%       prism/src/lib/scrip/src/scriprmp.f} or {\tt vector.F90} (for
%     vector interpolation) are compiled with {\tt ll\_weightot=.true.}.
    
   
\end{itemize}

%{\bf Precautions related to the use of all SCRIPR remappings}

%\begin{itemize}

%\item For using {\tt SCRIPR} interpolations, linking with the NetCDF library
%    is mandatory and the grid data files (see section \ref{subsec_griddata})
%    must be NetCDF files (binary files are not supported). 



%the weights and addresses will also differ whether or not
%the {\tt MASK} and {\tt EXTRAP} transformations are first activated
%during the pre-processing phase (see section \ref{subsec_preproc}) and this
%option is not stored in the remapping file name. Therefore, the
%remapping file used will be the one created for the first field having
%the same source grid, target grid, and interpolation type (and the
%same normalization option for {\tt CONSERV}), even if the {\tt MASK}
%and {\tt EXTRAP} transformations are used or not for that field.
%(This inconsistency is however usually not a problem as the {\tt MASK}
%and {\tt EXTRAP} transformations are usually used for all fields
%having the same source grid, target grid, and interpolation type, or
%not at all.)
%\end{itemize}

{\bf Support of vector fields with the SCRIPR remappings}

Vector mapping is NOT supported is not and will not be supported by
OASIS3-MCT. For proper treatment of vector fields, the component model
has to send the 3 components of the vector projected in a Cartesian
coordinate system.

\item {\bf INTERP}: UNUSED

\item {\bf MOZAIC}: UNUSED; note that {\tt MAPPING} (see above) is the NetCDF
  equivalent to {\tt MOZAIC}.

\item {\bf NOINTERP}: UNUSED

\item {\bf FILLING}: UNUSED

\end{itemize}

\section{The post-processing stage}
\label{subsec_cooking}

\begin{itemize}

\item {\bf CONSERV}: 

  {\tt CONSERV} ensures a global modification of the coupling field. 
  This analysis requires the source and target grid mesh areas
  to be transfered to the coupler with {\tt
    oasis\_write\_area/prism\_write\_area} (see section
  \ref{subsubsec_griddef}). {\bf For a correct CONSERV operation, overlapping grid cells 
on the source grid or on the target grid must be masked.} In the {\it namcouple}, {\tt CONSERV}
  requires one input line with one argument and one optional argument:

 \begin{verbatim}
# CONSERV operation
     $CMETH  $CONSOPT
  \end{verbatim}
\vspace{-0.5cm} 
where: 
  \begin{itemize}
  \item {\tt \$CMETH} is the method desired with the following choices
   \begin{itemize}
   \item with {\tt \$CMETH = GLOBAL}, the field is integrated on both
     source and target grids, without considering values of masked
     points, and the residual (target - source) is uniformly
     distributed on the target grid; this option ensures global
     conservation of the field
   \item with {\tt \$CMETH = GLBPOS}, the same operation is performed
     except that the residual is distributed proportionally to the
     value of the original field; this option ensures the global
     conservation of the field and does not change the sign of the field
   \item with {\tt \$CMETH = BASBAL}, the operation is analogous to
     {\tt GLOBAL} except that the non masked surface of the source and
     the target grids are taken into account in
     the calculation of the residual; this option does not ensure
     global conservation of the field but ensures that the energy received is
     proportional to the non masked surface of the target grid
   \item with {\tt \$CMETH = BASPOS}, the non masked surface of the
     source and the target grids are taken into
     account and the residual is distributed proportionally to the
     value of the original field; therefore, this option does not
     ensure global conservation of the field but ensures that the energy received
     is proportional to the non masked surface of the target grid and 
     it does not change the sign of the field.
   \end{itemize}
  \item {\tt \$CONSOPT} is an optional argument specifying the algorithm.  
  {\tt \$CONSOPT} can be {\tt bfb} or {\tt opt}.
  The {\tt bfb} option enforces a bit-for-bit transformation regardless of the
  grid decomposition or process count.  The {\tt opt} option carries out the conservation using an
  optimal algorithm using less memory and a faster approach. Option {\tt bfb} is the
  default setting.
  \end{itemize}



\item {\bf SUBGRID}: UNUSED

%{\tt SUBGRID} can be used to interpolate a field from a coarse grid to a
%finer target grid (the target grid must be finer over the whole
%domain). Two types of subgrid interpolation can be performed,
%depending on the type of the field.
%
%For solar type of flux field ({\tt \$SUBTYPE = SOLAR}), the operation
%performed is:
%$$\Phi_{i} = \frac{1-\alpha_i}{1-\alpha} F$$ where $\Phi_{i}$ ($F$) is
%the flux on the fine (coarse) grid, $\alpha_i$ ($\alpha$) an auxiliary
%field on the fine (coarse) grid (e.g. the albedo).  The whole
%operation is interpolated from the coarse grid with a grid-mapping type
%of interpolation; the dataset of weights and addresses has to be given
%by the user.
%
%For non-solar type of field ({\tt \$SUBTYPE = NONSOLAR}), a
%first-order Taylor expansion of the field on the fine grid relatively
%to a state variable is performed (for instance, an expansion of the
%total heat flux relatively to the SST):
%$$\Phi_{i} = F + \frac{\partial F}{\partial T} ( T_i - T ) $$ where
%$\Phi_{i}$ ($F$) is the heat flux on the fine (coarse) grid, $T_i$
%($T$) an auxiliary field on the fine (coarse) grid (e.g. the SST) and
%$\frac{\partial F}{\partial T}$ the derivative of the flux versus the
%auxiliary field on the coarse grid. This operation is interpolated
%from the coarse grid with a grid-mapping type of interpolation; the
%dataset of weights and addresses has to be given by the user.
%
%This analysis requires one input line with 7 or 8
%arguments depending on the type of subgrid interpolation.
%
%\begin{enumerate}
%\item If the the {\tt SUBGRID} operation is performed on a solar flux,
%the 7-argument input line is:
%\begin{verbatim}
%# SUBGRID operation with $SUBTYPE=SOLAR 
%  $CFILE  $NUMLU  $NID  $NV  $SUBTYPE  $CCOARSE $CFINE\end{verbatim}where
%{\tt \$CFILE} and {\tt \$NUMLU} are the subgrid-mapping file name and
%associated logical unit (see section \ref{subsec_transformationdata} for the
%structure of this file); {\tt \$NID} the identificator for this
%subgrid-mapping dataset within the file build by OASIS based on all
%the different {\tt SUBGRID} analyses in the present coupling; {\tt
%\$NV} is the maximum number of target grid points use in the
%subgrid-mapping; {\tt \$SUBTYPE = SOLAR} is the type of subgrid
%  interpolation; {\tt \$CCOARSE} is the
%auxiliary field name on the coarse grid (corresponding to $\alpha$)
%and {\tt \$CFINE} is the auxiliary field name on fine grid
%(corresponding to $\alpha_i$).
%These two fields needs to be exchanged between their original model
%and OASIS3 main process, at least as {\tt AUXILARY} fields.
%This analysis is performed from the coarse grid with a grid-mapping type
%of interpolation based on the {\tt \$CFILE} file.
%
%\item If the the SUBGRID operation is performed on a nonsolar flux,
%the 8-argument input line is:
%\begin{verbatim}
%# SUBGRID operation with $SUBTYPE=NONSOLAR
%  $CFILE $NUMLU $NID $NV $SUBTYPE $CCOARSE $CFINE $CDQDT
%\end{verbatim} where {\tt \$CFILE},  {\tt \$NUMLU},  {\tt \$NID},  
%{\tt \$NV} are as for a solar subgrid interpolation; {\tt
%\$SUBTYPE = NONSOLAR}; {\tt \$CCOARSE} is the auxiliary
%field name on the coarse grid (corresponding to $T$) and {\tt \$CFINE}
%is the auxiliary field name on fine grid (corresponding to $T_i$); the
%additional argument {\tt \$CDQDT} is the coupling ratio on the coarse
%grid (corresponding to $\frac{\partial F}{\partial T}$) These three
%fields need to be exchanged between their original model and OASIS3
%main process as {\tt AUXILARY} fields. This operation is performed from the
%coarse grid with a grid-mapping type of interpolation based on the {\tt
%\$CFILE} file.
%
%\end{enumerate}

\item {\bf BLASNEW}:
 
{\tt BLASNEW} performs a scalar
multiply or scalar add to any destination field.  This is the equivalent
of BLASOLD on the destination side.  The prior feature that supported
linear combinations of the
current coupling field with any other fields after the
interpolation has been deprecated.

This analysis requires the same input line(s) as {\tt BLASOLD}.

\item {\bf MASKP}: UNUSED

\item {\bf REVERSE}: UNUSED

\item {\bf CHECKOUT}: 

{\tt CHECKOUT} calculates the global minimum, the maximum and the sum of the
 the target field values and prints them to the OASIS3-MCT debug file (for the master process of the target component model only). This operation does not transform the field. 
The generic input line is as for {\tt CHECKIN} (see above).

\item {\bf GLORED}: UNUSED

\end{itemize}


