\newpage
\chapter{Transformations and interpolations}
\label{sec_transformations}

Different transformations and 2D interpolations are available in
OASIS3-MCT to adapt the coupling fields from a source model grid to a
target model grid.  In the following paragraphs, a description of each
transformation with its corresponding configuration lines that the
user has to write in the {\it namcouple} file is given.  Features that
are now deprecated (non functional) compared to prior versions will be
noted with the string UNUSED but not described.

\section{Time transformations}
\label{subsec_timetrans}

\begin{itemize}

\item {\bf LOCTRANS}:

  {\tt LOCTRANS} requires one configuring line on which a time
  transformation, automatically performed below the call to {\tt
    oasis\_put}, should be indicated:

  \begin{verbatim}
 # LOCTRANS operation
   $TRANSFORM
\end{verbatim}
  % $
  \vspace{-0.2cm} where {\tt \$TRANSFORM} can be
  \begin{itemize}
  \item {\tt INSTANT}: no time transformation, the instantaneous field
    is transferred;
  \item {\tt ACCUMUL}: the field accumulated over the previous
    coupling period is exchanged (the accumulation is simply done over
    the arrays {\tt field\_array} provided as third argument to the
    {\tt oasis\_put} calls, not weighted by the time interval between
    these calls);
  \item {\tt AVERAGE}: the field averaged over the previous coupling
    period is transferred (the average is simply done over the arrays
    {\tt field\_array} provided as third argument to the {\tt
      oasis\_put} calls, not weighted by the time interval between
    these calls);
  \item {\tt T\_MIN}: the minimum value of the field for each source
    grid point over the previous coupling period is transferred;
  \item {\tt T\_MAX}: the maximum value of the field for each source
    grid point over the previous coupling period is transferred;
  \item {\tt ONCE}: UNUSED ; {\bf not supported in OASIS3-MCT.}
  \end{itemize}

  With OASIS3-MCT, time transformations are supported more generally
  with use of the coupling restart file.  The coupling restart file
  allows the partial time transformation to be saved at the end of a
  run for exact restart at the start of the next run. When LOCTRANS 
  transformations are specified, the initial coupling restart file
  should not contain any LOCTRANS restart fields. For the following
  runs, it is mandatory that the coupling restart file contains
  LOCTRANS restart fields coherent with the current namcouple
  entries. For example, it will not be possible to restart a run with
  a multiple field entry in the namcouple with a coupling restart 
  file created by a run not activating this multiple file option.
  This is
  the reason why it is now possible to specify a restart file name
  on the OUTPUT {\it namcouple} input line.

\end{itemize}

\section{The pre-processing transformations}
\label{subsec_preproc}

\begin{itemize}

\item {\bf REDGLO} UNUSED

\item {\bf INVERT}: UNUSED

\item {\bf MASK}: UNUSED
 
\item {\bf EXTRAP}: UNUSED

\item {\bf CHECKIN}:

  {\tt CHECKIN} calculates the global minimum, the maximum and the sum
  of the source field values (not weighted by the grid cell area) and
  prints them to the OASIS3-MCT debug file (for the master process of
  the source component model only under the 
  attribute ``diags''). This operation does not transform the field.
  CHECKIN operation can significantly slow down the simulation. It should
  not be used in production mode.

  The generic input line is as follows:
 \begin{verbatim}
 # CHECKIN operation
     INT = 1  
\end{verbatim} 

\item {\bf CORRECT}: UNUSED

\item {\bf BLASOLD}:

  {\tt BLASOLD} allows the source field to be scaled and allows a
  scalar to be added to the field.  The prior ability to perform a
  linear combination of the current coupling field with other coupling
  fields has been deprecated in OASIS3-MCT.  This transformation
  occurs before the interpolation {\it per se}.

  This transformation requires at least one configuring line with two
  parameters:
 \begin{verbatim}
# BLASOLD operation
     $XMULT   $NBFIELDS 
\end{verbatim}
  % \vspace{-0.2cm}
  where {\tt \$XMULT} is the multiplicative coefficient of the source
  field, which must be given as a REAL value (e.g 2.0 and not 2). {\tt
    \$NBFIELDS} must be 0 if no scalar needs to be added or 1 if a
  scalar needs to be added. In this last case, an additional input
  line is required where {\tt \$AVALUE} is the scalar to be added to
  the field, which must also be given as a REAL value :
\begin{verbatim}
     CONSTANT  $AVALUE
\end{verbatim} 
\end{itemize}

% subsection{The pre-processing transformations}

\section{The remapping (or interpolation or regridding)}
\label{subsec_interp}

\begin{itemize}

\item {\bf MAPPING}:

  The {\tt MAPPING} keyword is used to specify an input file to be
  read and used for mapping (ie. regridding or interpolation); the
  {\tt MAPPING} file must follow the {\tt SCRIPR} format (see
    section \ref{subsec_mapdata}).
 
  Since OASIS3-MCT\_2.0, {\tt MAPPING} can be used for higher order
  remapping. Up to 5 different sets of weights (see section
  \ref{subsec_mapdata} for the weight file format) can be applied to
  up to 5 different fields transfered through the {\tt oasis\_put} argument
  (see section \ref{prismput}).

  This transformation requires at least one configuring line with one
  filename and two optional string values:
\begin{verbatim}
     $MAPNAME  $MAPLOC  $MAPSTRATEGY
\end{verbatim}
  \begin{itemize}
  \item {\tt \$MAPNAME} is the name of the mapping file to read.  This
    is a NetCDF file consistent with the SCRIPR map file format (see
    section \ref{subsec_mapdata}).

  \item {\tt \$MAPLOC} is optional and can be either {\tt src} or {\tt
      dst}.  With {\tt src}, the mapping will be done in parallel on
    the source processors before communication to the destination
    model processes; this is the default.  With {\tt dst}, the mapping
    is done on the destination processes after the source grid data is
    sent from the source model.

  \item {\tt \$MAPSTRATEGY} is optional and can be either {\tt bfb},
    {\tt sum}, or {\tt opt}.  In {\tt bfb} mode, the mapping is done
    using a strategy that produces bit-for-bit identical results
    regardless of the grid decompositions; this is the default.  With {\tt sum}, the
    transform is done using the partial sum approach which generally
    introduces roundoff level changes in the results on different
    processor counts. Option {\tt opt} allows the coupling layer to
    choose either approach based on an analysis of which strategy is
    likely to run faster. Usually, partial sums will be used if the
    source grid has a higher resolution than the target grid as this
    should reduce the overall communication (e.g. for conservative
    remapping).

  \end{itemize}

  Note that if {\tt SCRIPR} (see below) is used to calculate the
  remapping file, {\tt MAPPING} can still be listed in the {\tt
    namcouple} to specify a name for the remapping file generated by
  {\tt SCRIPR} different from the default and/or to specify a {\tt
    \$MAPLOC} or {\tt \$MAPSTRATEGY} option.

\item {\bf SCRIPR}:
 
  {\tt SCRIPR} gathers the interpolation techniques offered by Los
  Alamos National Laboratory SCRIP 1.4 library
  \citep{Jones99}\footnote{See also
    http://climate.lanl.gov/Software/SCRIP/ and the copyright
    statement in appendix \ref{sec_SCRIP}.}.  {\tt SCRIPR} routines
  are in {\tt oasis3-mct/lib/scrip}. See the SCRIP 1.4 documentation
  in {\tt oasis3/doc/SCRIPusers.pdf} for more details on the
  interpolation algorithms.

  When the SCRIP library performs a remapping, it first checks if the
  file containing the corresponding remapping weights and addresses
  exists; if it exists, it reads them from the file; if not, it
  calculates them and store them in a file. The file is created in the
  working directory and is by default called {\tt rmp\_{\it srcg}\_to\_{\it
      tgtg}\_{\it INTTYPE}\_{\it NORMAOPT}.nc}, where {\it srcg} and
  {\it tgtg} are the acronyms of respetively the source and the target
  grids, {\it INTTYPE} is the interpolation type, i.e. {\tt DISTWGT},
  {\tt GAUSWGT}, {\tt BILINEAR} ({\bf not BILINEA as in OASIS3.3}) or
  {\tt CONSERV} -see below, and {\it NORMAOPT} is the normalization
  option, i.e. {\tt DESTAREA}, {\tt FRACAREA} or {\tt FRACNNEI} for
  {\tt CONSERV} only -see below). One has to take care that the
  remapping file will have the same name even if other details, like
  the grid masks or the {\tt \$MAPLOC} or {\tt \$MAPSTRATEGY} options, 
  are changed. When reusing a remapping file, one has
  to be sure that it was generated in exactly the same conditions than
  the ones it is used for.

  The following types of interpolations are available:

  \begin{itemize}

  \item {\tt DISTWGT} performs a distance weighted nearest-neighbour
    interpolation (N neighbours). All types of grids are supported.

    \begin{itemize}

    \item Masked target grid points: the zero value is associated to
      masked target grid points.

    \item Non-masked target grid points having some of the N source
      nearest neighbours masked: a nearest neighbour algorithm using
      the remaining non masked source nearest neighbours is applied.

    \item Non-masked target grid points having all of the N source
      nearest neighbours masked: by default, the nearest non-masked
      source neighbour is used (logical {\tt ll\_nnei} hard-coded to
      {\tt .true.} in {\tt oasis3-mct/lib/scrip/src/remap\_distwgt.F};
      same default behaviour as OASIS3.3).

    \end{itemize}

    The configuring line is:

  \begin{verbatim}
 # SCRIPR (for DISWGT) 
     $CMETH $CGRS $CFTYP $REST $NBIN $NV $ASSCMP $PROJCART
\end{verbatim} 
    where:
    % \vspace{-0.5cm}
    \begin{itemize}
    \item {\tt \$CMETH = DISTWGT}.
    \item {\tt \$CGRS} is the source grid type ({\tt LR}, {\tt D} or
      {\tt U})- see appendix \ref{subsec_gridtypes}.

    \item {\tt \$CFTYP} is the field type: {\tt SCALAR}. The option
      {\tt VECTOR}, which in fact leads to a scalar treatment of the
      field (as in the previous versions), is still accepted. {\bf
        VECTOR\_I or VECTOR\_J, i.e. vector fields, are not supported
        anymore in OASIS3-MCT.}. See ``Support of vector fields with
      the SCRIPR remappings'' below.

    \item {\tt \$REST} is the search restriction type: {\tt LATLON} or
      {\tt LATITUDE} (see SCRIP 1.4 documentation SCRIPusers.pdf).
    \item {\tt \$NBIN} the number of restriction bins (see SCRIP 1.4
      documentation SCRIPusers.pdf). Note that for D or U grid, the
      restriction with more than 1 bin is not allowed : choose LATLON or
      LATITUDE and \$NBIN=1
    \item {\tt \$NV} is the number of neighbours used.
    \item {\tt \$ASSCMP}: UNUSED; {\bf vector fields are not supported
        anymore in OASIS3-MCT.} See ``Support of vector fields with
      the SCRIPR remappings'' below.
    \item {\tt \$PROJCART}: UNUSED; {\bf vector fields are not
        supported anymore in OASIS3-MCT.} See ``Support of vector
      fields with the SCRIPR remappings'' below.
    \end{itemize}

  \item {\tt GAUSWGT} performs a N nearest-neighbour interpolation
    weighted by their distance and a gaussian function. All grid types
    are supported.
    \begin{itemize}

    \item Masked target grid points: the zero value is associated to
      masked target grid points.

    \item Non-masked target grid points having some of the N source
      nearest neighbours masked: a nearest neighbour algorithm using
      the remaining non masked source nearest neighbours is applied.

    \item Non-masked target grid points having all of the N source
      nearest neighbours masked: by default, the nearest non-masked
      source neighbour is used (logical {\tt ll\_nnei} hard-coded to
      {\tt .true.} in {\tt
        oasis3-mct/lib/scrip/src/remap\_gauswgt.f}); {\bf this is NOT
        the same default behaviour as OASIS3.3}; to have the same
      default behaviour as in OASIS3.3, put {\tt ll\_nnei=.false.}.
    \end{itemize}

    The configuring line is:
  \begin{verbatim}
 # SCRIPR (for GAUSWGT)
     $CMETH  $CGRS  $CFTYP  $REST  $NBIN  $NV $VAR
\end{verbatim}
    %
    % \vspace{-0.5cm}
    where:
    % $
    all entries are as for {\tt DISTWGT}, except that:
    \begin{itemize}
    \item {\tt \$CMETH = GAUSWGT}
    \item {\tt \$VAR}, which must be given as a REAL value (e.g 2.0
      and not 2), defines the weight given to a neighbour source grid
      point as proportional to $exp(-1/2 \cdot d^2/\sigma^2)$ where
      $d$ is the distance between the source and target grid points,
      and $\sigma^2 = \$VAR \cdot \overline{d}^2$ where
      $\overline{d}^2$ is the average distance between two source grid
      points (calculated automatically by OASIS3-MCT).
    \end{itemize}

  \item {\tt BILINEAR} performs an interpolation based on a local
    bilinear approximation (see details in chapter 4 of SCRIP 1.4
    documentation SCRIPusers.pdf). Logically-Rectangular (LR) and
    Reduced (D) source grid types are supported.

  \item {\tt BICUBIC} performs an interpolation based on a local
    bicubic approximation for Logically-Rectangular (LR) grids (see
    details in chapter 5 of SCRIP 1.4 documentation SCRIPusers.pdf),
    and on a 16-point stencil for Gaussian Reduced (D) grids.  Note
    that for Logically-Rectangular grids, 4 weights for each of the 4
    enclosing source neighbours are required corresponding to the
    field value at the point, the gradient of the field with respect
    to {\it i}, the gradient of the field with respect to {\it j}, and
    the cross gradient with respect to {\it i} and {\it j} in that
    order. OASIS3-MCT will calculate the remapping weights and
    addresses (if they are not already provided) but will not, at run
    time, calculate the two gradients and the cross-gradient of the
    source field (as was the case with OASIS3.3). These 3 extra fields
    need to be calculated by the source code and transfered as extra
    arguments of the {\tt oasis\_put} (see {\tt fld2, fld3, fld4} in
    section \ref{prismput}).

    For both {\tt BILINEAR} and {\tt BICUBIC}:
    \begin{itemize}
    \item Masked target grid points: the zero value is associated to
      masked target grid points.

    \item Non-masked target grid points having some of the source
      points normally used in the bilinear or bicubic interpolation
      masked: a N nearest neighbour algorithm using the remaining non
      masked source points is applied.

    \item Non-masked target grid points having all bilinear or bicubic
      neighbours masked: by default, the nearest non-masked source
      neighbour is used ({\tt ll\_nnei} hard-coded to {\tt .true.} in
      {\tt oasis3-mct/lib/scrip/src/remap\_bilinear.f}, {\tt
        remap\_bicubic.f} and {\tt remap\_bicubic\_reduced.F90}); {\bf
        this is not the same default behaviour as OASIS3.3}; to have
      the same default behaviour as in OASIS3.3, put {\tt
        ll\_nnei=.false.} in the appropriate routine.
    \end{itemize}
 
    For both {\tt BILINEAR} and {\tt BICUBIC}, the configuring line
    is:

  \begin{verbatim}
 # SCRIPR  (for BILINEAR or BICUBIC)
     $CMETH  $CGRS  $CFTYP  $REST  $NBIN
\end{verbatim}
    \vspace{-0.5cm} where:
    % $
    \begin{itemize}
    \item {\tt \$CMETH = BILINEAR} or {\tt BICUBIC}
    \item {\tt \$CGRS} is the source grid type: LR or D.
    \item {\tt \$CFTYP}, {\tt \$NBIN} are as for {\tt DISTWGT}.
    \item {\tt \$REST} is as for {\tt DISTWGT}, except that only {\tt
        LATITUDE} is possible for a Reduced (D) source grid.
    \end{itemize}

  \item {\tt CONSERV} performs 1st or 2nd order conservative
    remapping, which means that the weight of a source cell is
    proportional to area intersected by the target cell (plus some
    other terms proportional to the gradient of the field in the
    longitudinal and latitudinal directions for the second order).

    The configuring line is:
  \begin{verbatim}
 # SCRIPR (for CONSERV)
     $CMETH  $CGRS  $CFTYP  $REST  $NBIN  $NORM  $ORDER 
\end{verbatim}
    % $
    \vspace{-0.5cm} where:
    \begin{itemize}
    \item {\tt \$CMETH = CONSERV}
    \item {\tt \$CGRS} is the source grid type: LR, D and U. Note that
      the grid corners have to given by the user in the grid data file
      {\tt grids.nc} or by the code itself in the initialisation phase
      by calling routine {\tt oasis\_write\_corner} (see section
      \ref{subsubsec_griddef}) ; OASIS3-MCT will not attempt to
      automatically calculate them as OASIS3.3.
      % For second-order remapping, only LR is supported because the
      % gradient of the coupling field used in the transformation has
      % to be calculated automatically by OASIS3.
    \item {\tt \$CFTYP, \$REST}, {\tt \$NBIN} are as for {\tt
        DISTWGT}.
    \item {\tt \$NORM} is the NORMalization option:
      \begin{itemize}
      \item {\tt FRACAREA}: The sum of the non-masked source cell
        intersected areas is used to NORMalise each target cell field
        value: the flux is not locally conserved, but the flux value
        itself is reasonable.
      \item {\tt DESTAREA}: The total target cell area is used to
        NORMalise each target cell field value even if it only partly
        intersects non-masked source grid cells: local flux
        conservation is ensured, but unreasonable flux values may
        result.
      \item {\tt FRACNNEI}: as {\tt FRACAREA}, except that at least
        the source nearest unmasked neighbour is used for unmasked
        target cells that intersect only masked source cells.  Note
        that a zero value will be assigned to a target cell that does
        not intersect any source cells (masked or unmasked), even with
        FRACNNEI option.
      \end{itemize}
    \item {\tt \$ORDER}: {\tt FIRST} or {\tt SECOND} for first or
      second order conservative remapping respectively (see SCRIP 1.4
      documentation).

      For {\tt CONSERV/SECOND}, 3 weigths are needed; OASIS3-MCT will
      calculate these weights and corresponding addresses (if they are
      not already provided) but will not, at run time, calculate the
      two extra terms to which the second and third weights should be
      applied; these terms, respectively the gradient of the field
      with respect to the longitude ($\theta$) $\frac{\delta f}{\delta
        \theta}$ and the gradient of the field with respect to the
      latitude ($\phi$) $\frac{1}{cos \theta}\frac{\delta f}{\delta
        \phi}$ need to be calculated by the source code and transfered
      as extra arguments of the {\tt oasis\_put} (see {\tt fld2, fld3}
      in section \ref{prismput}).  Note that {\tt CONSERV/SECOND} is
      not positive definite and has not been fully validated yet.

    \end{itemize}

  \end{itemize}

  {\bf Precautions related to the use of the SCRIPR/CONSERV remapping}

  \begin{itemize}

  \item For the 1st order conservative remapping: the weight of a
    source cell is proportional to area of the source cell intersected
    by target cell.  Using the divergence theorem, the SCRIP library
    evaluates this area with the line integral along the cell borders
    enclosing the area. As the real shape of the borders is not known
    (only the location of the 4 corners of each cell is known), the
    library assumes that the borders are linear in latitude and
    longitude between two corners.  This assumption becomes less valid
    closer to the pole; for latitudes above the {\tt north\_thresh}
    or below the {\tt south\_thresh} values (see {\tt
      oasis3-mct/lib/scrip/remap\_conserv.F}, the library evaluates
    the intersection between two border segments using a Lambert
    equivalent azimuthal projection. Problems were observed in some
    cases for the grid cell located around this {\tt north\_thresh} or
    {\tt south\_thresh} latitude.

  \item Another limitation of the SCRIP 1st order conservative
    remapping algorithm is that is also supposes, for line integral
    calculation, that $sin(latitude)$ is linear in longitude on the
    cell borders which again is in general not valid close to the
    pole.
    % A projection or at least a normalization by the
    % true area of the cells (i.e. by the areas as considered by the
    % component models) is needed.

  \item For a proper consevative remapping, the corners of a cell have
    to coincide with the corners of its neighbour cell, with no
    ``holes'' between the cells.
  
  \item If two cells of one same grid overlay, the one with
    the greater numerical index must be masked in {\it
      masks.nc} for a proper conservative remapping.  For example, 
     if the grid cells with {\it i=1} overlays the grid cells with {\it i=imax}, the latter must
    be masked.  If none of overlying cells is masked (given the
    original mask defined in {\it
      masks.nc}), OASIS3-MCT must be compiled with the CPP key {\tt
      TREAT\_OVERLAY} which will ensure that these rules are
    respected. This CPP key was introduced in OASIS3.3.
      
  \item A target grid cell intersecting no source cell (either masked
    or non masked) at all i.e. falling in a ``hole'' of the source
    grid will get the non-masked nearest-neighbour value.
    
  \item If a target grid cell intersects only masked source cells, it
    will still get a zero value unless the {\tt FRACNNEI}
    normalisation option is used, in which case it will get the
    nearest non masked neighbour value. {\bf Note that the option of
      having the value 1.0E+20 assigned to these target grid cell
      intersecting only masked source cells (for easier
      identification) is not yet availble in OASIS3-MCT.}
   

    % The target grid mask is never considered in {\tt CONSERV},
    % except with normalisation option {\tt FRACNNEI} (see below). To
    % have a value calculated, a target grid cell must intersect at
    % least one source cell. However, the NORMlisation option (that
    % takes into account the source grid mask, see below) may result
    % in a null value calculated for those target grid cells. In that
    % case (i.e.  at least one intersecting source cell, but a null
    % value finally calculated because of the normalisation option),
    % the value 1.0E+20 is assigned to those target grid points if
    % {\tt prism/src/lib/scrip/src/scriprmp.f} or {\tt vector.F90}
    % (for vector interpolation) are compiled with {\tt
    %   ll\_weightot=.true.}.
    
   
  \end{itemize}

  % {\bf Precautions related to the use of all SCRIPR remappings}

  % \begin{itemize}

  % \item For using {\tt SCRIPR} interpolations, linking with the
  %   NetCDF library is mandatory and the grid data files (see section
  %   \ref{subsec_griddata}) must be NetCDF files (binary files are
  %   not supported).



  %   the weights and addresses will also differ whether or not the
  %   {\tt MASK} and {\tt EXTRAP} transformations are first activated
  %   during the pre-processing phase (see section
  %   \ref{subsec_preproc}) and this option is not stored in the
  %   remapping file name. Therefore, the remapping file used will be
  %   the one created for the first field having the same source grid,
  %   target grid, and interpolation type (and the same normalization
  %   option for {\tt CONSERV}), even if the {\tt MASK} and {\tt
  %     EXTRAP} transformations are used or not for that field.  (This
  %   inconsistency is however usually not a problem as the {\tt MASK}
  %   and {\tt EXTRAP} transformations are usually used for all fields
  %   having the same source grid, target grid, and interpolation
  %   type, or not at all.)
  % \end{itemize}

  {\bf Support of vector fields with the SCRIPR remappings}

  Vector mapping is NOT supported and will not be supported by
  OASIS3-MCT. For proper treatment of vector fields, the source 
  code has to send the 3 components of the vector projected in a
  Cartesian coordinate system as separate fields. The target 
  code has to received the 3 interpolated Cartesian components and
  recombine them to get the proper vector field.

\item {\bf INTERP}: UNUSED

\item {\bf MOZAIC}: UNUSED; note that {\tt MAPPING} (see above) is the
  NetCDF equivalent to {\tt MOZAIC}.

\item {\bf NOINTERP}: UNUSED

\item {\bf FILLING}: UNUSED

\end{itemize}

\section{The post-processing stage}
\label{subsec_cooking}

\begin{itemize}

\item {\bf CONSERV}:

  {\tt CONSERV} performs a global modification of the coupling field.
  This analysis requires the source and target grid mesh areas to be
  present in the {\it areas.nc}) file or transferred to the coupler with {\tt oasis\_write\_area} (see section
  \ref{subsubsec_griddef}). {\bf For a correct CONSERV operation,
    overlapping grid cells on the source grid or on the target grid
    must be masked.} In the {\it namcouple}, {\tt CONSERV} requires
  one input line with one argument and one optional argument:

 \begin{verbatim}
# CONSERV operation
     $CMETH  $CONSOPT
\end{verbatim}
  \vspace{-0.5cm} where:
  \begin{itemize}
  \item {\tt \$CMETH} is the method desired with the following choices
    \begin{itemize}
    \item with {\tt \$CMETH = GLOBAL}, the field is integrated on both
      source and target grids, without considering values of masked
      points, and the residual (target - source) is uniformly
      distributed on the target grid; this option ensures global
      conservation of the field
    \item with {\tt \$CMETH = GLBPOS}, the same operation is performed
      except that the residual is distributed proportionally to the
      value of the original field; this option ensures the global
      conservation of the field and does not change the sign of the
      field
    \item with {\tt \$CMETH = BASBAL}, the operation is analogous to
      {\tt GLOBAL} except that the non masked surface of the source
      and the target grids are taken into account in the calculation
      of the residual; this option does not ensure global conservation
      of the field but ensures that the energy received is
      proportional to the non masked surface of the target grid
    \item with {\tt \$CMETH = BASPOS}, the non masked surface of the
      source and the target grids are taken into account and the
      residual is distributed proportionally to the value of the
      original field; therefore, this option does not ensure global
      conservation of the field but ensures that the energy received
      is proportional to the non masked surface of the target grid and
      it does not change the sign of the field.
    \end{itemize}
  \item {\tt \$CONSOPT} is an optional argument specifying the
    algorithm.  {\tt \$CONSOPT} can be {\tt bfb} or {\tt opt}.  
\begin{itemize}
\item The {\tt bfb} option enforces a bit-for-bit transformation regardless
    of the component grid decomposition or number of processes. It is currently the default setting. 
With {\tt bfb}, the entire field is gathered from the different component processes to the master process that performs the global integration and then 
broadcasts the resulting sum to all component processes. Compared to {\tt opt}, {\tt bfb} is clearly more expensive with respect to
MPI communications and memory requirement.  
\item The {\tt opt} option
    carries out the global conservation with an optimal algorithm using less
    memory and a faster approach. With {\tt opt}, a local sum is performed on each process that then sends its local sum to all other processes; all processes can then compute the global sum of all local sums. This is more efficient than the {\tt bfb} algorithm but does not ensure bit-for-bit reproducibility when the  grid decomposition or number of processes of the component are changed.
 
\end{itemize}
  \end{itemize}



\item {\bf SUBGRID}: UNUSED

  % {\tt SUBGRID} can be used to interpolate a field from a coarse
  % grid to a finer target grid (the target grid must be finer over
  % the whole domain). Two types of subgrid interpolation can be
  % performed, depending on the type of the field.
%
  % For solar type of flux field ({\tt \$SUBTYPE = SOLAR}), the
  % operation performed is:
%$$\Phi_{i} = \frac{1-\alpha_i}{1-\alpha} F$$ where $\Phi_{i}$ ($F$) is
%the flux on the fine (coarse) grid, $\alpha_i$ ($\alpha$) an
% auxiliary field on the fine (coarse) grid (e.g. the albedo).  The
% whole operation is interpolated from the coarse grid with a
% grid-mapping type of interpolation; the dataset of weights and
% addresses has to be given by the user.
%
% For non-solar type of field ({\tt \$SUBTYPE = NONSOLAR}), a
% first-order Taylor expansion of the field on the fine grid
% relatively to a state variable is performed (for instance, an
% expansion of the total heat flux relatively to the SST):
%$$\Phi_{i} = F + \frac{\partial F}{\partial T} ( T_i - T ) $$ where
%$\Phi_{i}$ ($F$) is the heat flux on the fine (coarse) grid, $T_i$
% ($T$) an auxiliary field on the fine (coarse) grid (e.g. the SST)
% and $\frac{\partial F}{\partial T}$ the derivative of the flux
% versus the auxiliary field on the coarse grid. This operation is
% interpolated from the coarse grid with a grid-mapping type of
% interpolation; the dataset of weights and addresses has to be given
% by the user.
%
% This analysis requires one input line with 7 or 8 arguments
% depending on the type of subgrid interpolation.
%
% \begin{enumerate}
% \item If the the {\tt SUBGRID} operation is performed on a solar
%   flux, the 7-argument input line is:
%\begin{verbatim}
%# SUBGRID operation with $SUBTYPE=SOLAR 
%  $CFILE  $NUMLU  $NID  $NV  $SUBTYPE  $CCOARSE $CFINE\end{verbatim}where
%{\tt \$CFILE} and {\tt \$NUMLU} are the subgrid-mapping file name and
%associated logical unit (see section \ref{subsec_transformationdata} for the
%structure of this file); {\tt \$NID} the identificator for this
%subgrid-mapping dataset within the file build by OASIS based on all
%the different {\tt SUBGRID} analyses in the present coupling; {\tt
%\$NV} is the maximum number of target grid points use in the
%subgrid-mapping; {\tt \$SUBTYPE = SOLAR} is the type of subgrid
%  interpolation; {\tt \$CCOARSE} is the
%auxiliary field name on the coarse grid (corresponding to $\alpha$)
%and {\tt \$CFINE} is the auxiliary field name on fine grid
%(corresponding to $\alpha_i$).
%These two fields needs to be exchanged between their original model
%and OASIS3 main process, at least as {\tt AUXILARY} fields.
%This analysis is performed from the coarse grid with a grid-mapping type
%of interpolation based on the {\tt \$CFILE} file.
%
% \item If the the SUBGRID operation is performed on a nonsolar flux,
%   the 8-argument input line is:
%\begin{verbatim}
%# SUBGRID operation with $SUBTYPE=NONSOLAR
%  $CFILE $NUMLU $NID $NV $SUBTYPE $CCOARSE $CFINE $CDQDT
%\end{verbatim} where {\tt \$CFILE},  {\tt \$NUMLU},  {\tt \$NID},  
%{\tt \$NV} are as for a solar subgrid interpolation; {\tt
%\$SUBTYPE = NONSOLAR}; {\tt \$CCOARSE} is the auxiliary
%field name on the coarse grid (corresponding to $T$) and {\tt \$CFINE}
%is the auxiliary field name on fine grid (corresponding to $T_i$); the
%additional argument {\tt \$CDQDT} is the coupling ratio on the coarse
%grid (corresponding to $\frac{\partial F}{\partial T}$) These three
%fields need to be exchanged between their original model and OASIS3
%main process as {\tt AUXILARY} fields. This operation is performed from the
%coarse grid with a grid-mapping type of interpolation based on the {\tt
%\$CFILE} file.
%
% \end{enumerate}

\item {\bf BLASNEW}:
 
  {\tt BLASNEW} performs a scalar multiply or scalar add to any
  destination field.  This is the equivalent of BLASOLD on the
  destination side.  The prior feature that supported linear
  combinations of the current coupling field with any other fields
  after the interpolation has been deprecated.

  This analysis requires the same input line(s) as {\tt BLASOLD}.

\item {\bf MASKP}: UNUSED

\item {\bf REVERSE}: UNUSED

\item {\bf CHECKOUT}:

  {\tt CHECKOUT} calculates the global minimum, the maximum and the
  sum of the target field values (not weighted by the grid cell area)
  and prints them to the OASIS3-MCT debug file (for the master process
  of the target component model only). These informations are found in 
  the debug file of the master process of the target model under the 
  attribute ``diags''. This operation does not
  transform the field. CHECKOUT operation can significantly slow down the simulation.
  It should not be used in production mode.
  The generic input line is as for {\tt CHECKIN} (see above).

\item {\bf GLORED}: UNUSED

\end{itemize}


