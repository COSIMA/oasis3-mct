\newpage
\chapter{Introduction}
\label{sec_step}

In 1991, CERFACS started the development of
a software interface to couple existing numerical General
Circulation Models of the ocean and of the atmosphere. Today, the
OASIS3.3 coupler, which is the result of more than 20 years of evolution
is used by about 30 modelling groups in Europe, Australia, Asia and
North America on the different computing platforms. The list of
coupled models realized with OASIS3  and OASIS2 can be
found in tables \ref{Tab:OASIS3_a} and \ref{Tab:OASIS2_a} in Appendix \ref{sec_couplings}.

OASIS sustained development is ensured by a collaboration
between CERFACS and the Centre National de la Recherche Scientifique
(CNRS) and its maintainance and user support is presently reinforced
with additinal resources coming from IS-ENES project funded by the EU (FP7 -
GA no 228203).

The current OASIS3-MCT version was significantly refactored with respect to OASIS3.3. OASIS3-MCT is now interfaced with the Model
Coupling Toolkit\footnote{MCT, see www.mcs.anl.gov/research/projects/mct/} \citep{mct_larson} \citep{mct_jacob} developed by the Argonne National Laboratory in the USA. MCT implements fully parallel regridding (as a parallel matrix vector 
multiplication) and parallel distributed exchanges of the coupling
fields, based on pre-computed regridding weights and addresses. 
Its design philosophy, based on flexibility and minimal invasiveness,
is close to the OASIS approach. 
MCT has proven parallel performance and is, most notably, the
underlying coupling software used in National Center for Atmospheric
Research Community Earth System Model 1 (NCAR CESM1).

OASIS3-MCT is a portable set of Fortran 77, Fortran 90 and C routines. Low-intrusiveness, portability and flexibility are OASIS3-MCT key design concepts. At run-time, there is no
longer a separate coupler executable: OASIS3-MCT acts as a coupling library, which main function is to interpolate and exchange the coupling fields between the component models of a coupled system.  OASIS3-MCT supports  coupling of general two dimensional fields.  Unstructured grids are also supported using a one dimension degeneration of the two dimensional structures. Thanks to MCT, all transformations, including regridding, are executed in parallel on the set of source or target component processes and all couplings are now executed in parallel directly between the components via Message Passing Interface
(MPI). OASIS3-MCT also supports parallel file I/O using netcdf.  

In spite of the significant changes in underlying implementation, usage of OASIS3-MCT in the codes has largely remained unchanged with respect to OASIS3.3.
To communicate with another model, or to perform I/O actions, a component model needs to include few specific calls to the OASIS3-MCT coupling library, which Application Programmig Interface used in component models is unchanged.  The use 
statement has been updated and now requires a single ``use\ mod\_prism'' or ``use mod\_oasis'' statement instead of the various use statements required in prior
OASIS3 versions.  The {\it namcouple} configuration file is also largely unchanged
relative to OASIS3, although several options are either not used
or not supported.  There is a new transformation in {\it namcouple}
i.e. {\tt MAPPING} which allows a user to specify a mapping
file generated externally.  Some features like vector mapping
and second order mapping have been delayed in implementation
while other new features like parallel mapping have been added.
And currently, only MPI1 job launching is supported.
 
First tests done with up to
8000 cores on the Bullx Curie machine at the TGCC are very encouraging
and it is therefore very likely that OASIS3-MCT will provide an efficient and
easy-to-use coupling solution. 

%See appendix \ref{sec_changes} for a more detailed list of changes in this %version.

\section{Step-by-step use of OASIS3-MCT}

To use OASIS3-MCT for coupling models (and/or perform I/O
actions), one has to follow these steps:
\begin{enumerate}
\item Obtain OASIS3-MCT source code (see chapter \ref{sec_Obtaining}).
\item Identify the coupling or I/O fields and adapt the component
  models to allow their exchange with the OASIS3-MCT coupling library based on MPI1
  message passing.
  The OASIS3-MCT coupling library uses NetCDF and therefore can be used to perform I/O actions
  from/to disk files.  For more detail on how to interface a model
  with OASIS3-MCT, see chapter \ref{sec_modelinterfacing}.

The tutorial coupled model gives a practical example of a coupled
model; the sources are given in directories {\tt
  examples/tutorial }; more detail on the {\tt tutorial} and
how to compile and run it can be found in chapter
\ref{sec_compilationrunning}.

\item Define all coupling and I/O parameters and the transformations
  required to adapt each coupling field from its source model grid to
  its target model grid; on this basis, prepare OASIS3-MCT configuring file 
  {\it namcouple} (see chapter \ref{sec_namcouple}). 
  
  OASIS3-MCT supports different interpolation algorithms as is described in
  chapter \ref{sec_transformations}.  Regridding files can be compute
  online using the SCRIP options or offline and read using the MAPPING
  transformation.

\item Generate required auxiliary data files (see chapter
  \ref{sec_auxiliary}).
\item Compile OASIS3-MCT, the component models and start the coupled
  experiment. Chapter \ref{sec_compilationrunning} describes how to
  compile and run OASIS3-MCT and the {\tt tutorial} toy coupled model.

\end{enumerate}

If you need extra help, do not hesitate to contact us (see contact
details on the back of the cover page).

\section{OASIS3-MCT sources}
\label{sec_Obtaining}
OASIS3-MCT and {\tt tutorial} toy coupled model sources are available from CERFACS SVN server. To obtain more detail on how to download
the sources, please fill in the registration form at \newline 
https://verc.enes.org/oasis/download/oasis-registration-form .

OASIS3-MCT directory structure is the following one:

\begin{verbatim}
 - oasis3-mct/lib/psmile         OASIS3-MCT coupling library
                 /scrip          SCRIP interpolation library
                 /mct            Model Coupling Toolkit Coupling Software
                  
 - oasis3-mct/doc                OASIS3-MCT User Guide

 - oasis3-mct/util/make_dir      Utilities to compile OASIS3-MCT

 - oasis3-mct/examples/tutorial  Environment to run the tutorial toymodel
\end{verbatim}

\section{Licenses and Copyrights}
 
\subsection{OASIS3-MCT license and copyright statement}

Copyright © 2012 Centre Europ\'een de Recherche et Formation
Avanc\'ee en Calcul Scientifique (CERFACS).  

This software and ancillary information called OASIS3-MCT is free
software.  CERFACS has rights to use, reproduce, and distribute
OASIS3-MCT. The public may copy, distribute, use, prepare derivative works and
publicly display OASIS3-MCT under the terms of the Lesser GNU General
Public License (LGPL) as published by the Free Software Foundation,
provided that this notice and any statement of authorship are
reproduced on all copies. If OASIS3-MCT is modified to produce derivative
works, such modified software should be clearly marked, so as not to
confuse it with the OASIS3-MCT version available from CERFACS.

The developers of the OASIS3-MCT software are researchers attempting to
build a modular and user-friendly coupler accessible to the climate
modelling community. Although we use the tool ourselves and have made
every effort to ensure its accuracy, we can not make any
guarantees. We provide the software to you for free. In return,
you--the user--assume full responsibility for use of the software. The
OASIS3-MCT software comes without any warranties (implied or expressed) and
is not guaranteed to work for you or on your computer. Specifically,
CERFACS and the various individuals involved in development and
maintenance of the OASIS3-MCT software are not responsible for any damage
that may result from correct or incorrect use of this software.

\subsection{MCT copyright statement}
\label{sec_MCT}

                            Modeling Coupling Toolkit (MCT) Software

Copyright © 2011, UChicago Argonne, LLC as Operator of Argonne National Laboratory. All rights reserved.

Redistribution and use in source and binary forms, with or without modification, are permitted provided that the following conditions are met:
\begin{enumerate}
\item Redistributions of source code must retain the above copyright notice, this list of conditions and the following disclaimer.
\item Redistributions in binary form must reproduce the above copyright notice, this list of conditions and the following disclaimer in the documentation and/or other materials provided with the distribution.
\item The end-user documentation included with the redistribution, if any, must include the following acknowledgment: "This product includes software developed by the UChicago Argonne, LLC, as Operator of Argonne National Laboratory." Alternately, this acknowledgment may appear in the software itself, if and wherever such third-party acknowledgments normally appear.

This software was authored by:
\begin{itemize}
\item Argonne National Laboratory Climate Modeling Group, Mathematics and Computer Science Division, Argonne National Laboratory, Argonne IL 60439
\item Robert Jacob, tel: (630) 252-2983, E-mail: jacob@mcs.anl.gov
\item Jay Larson, E-mail: larson@mcs.anl.gov
\item Everest Ong
\item Ray Loy
\end{itemize}

\item WARRANTY DISCLAIMER. THE SOFTWARE IS SUPPLIED "AS IS" WITHOUT WARRANTY OF ANY KIND. THE COPYRIGHT HOLDER, THE UNITED STATES, THE UNITED STATES DEPARTMENT OF ENERGY, AND THEIR EMPLOYEES: (1) DISCLAIM ANY WARRANTIES, EXPRESS OR IMPLIED, INCLUDING BUT NOT LIMITED TO ANY IMPLIED WARRANTIES OF MERCHANTABILITY, FITNESS FOR A PARTICULAR PURPOSE, TITLE OR NON-IN- FRINGEMENT, (2) DO NOT ASSUME ANY LEGAL LIABILITY OR RESPONSIBILITY FOR THE ACCURACY, COMPLETENESS, OR USEFULNESS OF THE SOFTWARE, (3) DO NOT REPRESENT THAT USE OF THE SOFTWARE WOULD NOT INFRINGE PRIVATELY OWNED RIGHTS, (4) DO NOT WARRANT THAT THE SOFTWARE WILL FUNCTION UNINTERRUPTED, THAT IT IS ERROR-FREE OR THAT ANY ERRORS WILL BE CORRECTED. 

\item LIMITATION OF LIABILITY. IN NO EVENT WILL THE COPYRIGHT HOLDER, THE UNITED STATES, THE UNITED STATES DEPARTMENT OF ENERGY, OR THEIR EMPLOYEES: BE LIABLE FOR ANY INDIRECT, INCIDENTAL, CONSEQUENTIAL, SPECIAL OR PUNITIVE DAMAGES OF ANY KIND OR NATURE, INCLUDING BUT NOT LIMITED TO LOSS OF PROFITS OR LOSS OF DATA, FOR ANY REASON WHATSOEVER, WHETHER SUCH LIABILITY IS ASSERTED ON THE BASIS OF CONTRACT, TORT (INCLUDING NEGLIGENCE OR STRICT LIABILITY), OR OTHERWISE, EVEN IF ANY OF SAID PARTIES HAS BEEN WARNED OF THE POSSIBILITY OF SUCH LOSS OR DAMAGES.

\end{enumerate}

\subsection{The SCRIP 1.4 license copyright statement}
\label{sec_SCRIP}

The SCRIP 1.4 copyright statement reads as follows:

 ``Copyright © 1997, 1998 the Regents of the
University of California.  This software and ancillary information
(herein called SOFTWARE) called SCRIP is made available under the
terms described here. The SOFTWARE has been approved for release with
associated LA-CC Number 98-45. Unless otherwise indicated, this
SOFTWARE has been authored by an employee or employees of the
University of California, operator of Los Alamos National Laboratory
under Contract No. W-7405-ENG-36 with the United States Department of
Energy. The United States Government has rights to use, reproduce, and
distribute this SOFTWARE. The public may copy, distribute, prepare
derivative works and publicly display this SOFTWARE without charge,
provided that this Notice and any statement of authorship are
reproduced on all copies. Neither the Government nor the University
makes any warranty, express or implied, or assumes any liability or
responsibility for the use of this SOFTWARE. If SOFTWARE is modified
to produce derivative works, such modified SOFTWARE should be clearly
marked, so as not to confuse it with the version available from Los
Alamos National Laboratory.''
