\newpage
\chapter{Introduction}
\label{sec_step}

In 1991, CERFACS decided to tackle coupled climate modelling and to
develop a software interface to couple existing numerical General
Circulation Models of the ocean and of the atmosphere. Today, the
OASIS3 coupler, which is the result of more than 15 years of evolution
is used by about 30 modelling groups in Europe, Australia, Asia and
North America, on the different computing platforms used by the
climate modelling community. The list of
coupled models realized with OASIS3 and previous versions and the
platforms onto which they were run on in the few past years can be
found in Appendix \ref{sec_couplings}.

OASIS3 sustained development is ensured by a collaboration
between CERFACS and the Centre National de la Recherche Scientifique
(CNRS) and its maintainance and user support is presently reinforced
with additinal resources coming from IS-ENES project funded by the EU (FP7 -
GA no 228203), into which the parallel OASIS4 version of the
coupler is also currently being developed. 

OASIS3 is a portable set of Fortran 77, Fortran 90 and C routines.
Portability and flexibility are OASIS3 key design concepts. At
run-time, OASIS3 acts as a separate executable, which main function is
to interpolate the coupling fields exchanged between the component
models, and as a communication library linked to the component models, the OASIS3
PRISM Model Interface Library (PSMILe). OASIS3 supports 2D
coupling fields in the longitude and latitude dimensions. To communicate with OASIS3, or directly with
another model, or to perform I/O actions, a component model needs to
include few specific PSMILe calls. OASIS3 PSMILe supports in
particular parallel communication between a parallel component model
and OASIS3 interpolation executable based on Message Passing Interface
(MPI) and file I/O using the mpp\_io library from GFDL. New with this
version, the OASIS3 interpolation executable can be run on many processes, each process interpolating a
subset of the coupling fields, resulting in a
parallelisation of OASIS3 on a field-per-field basis. For each
coupling exchange, OASIS3 performs the transformations and regridding
needed to express the source field on the grid of the target model.
The current OASIS3\_3 version and its toy coupled model TOYOASIS3 were
compiled and run on NEC SX6, IBM Power4, CRAY XD1, and Linux PC (XXX
list to be updated).

% To exchange coupling information with other components, a component
% model needs to call few specific routines of the OASIS coupling
% library for its initialisation, grid and partition definition, field
% declaration, field {\bf Get} and {\bf Put} actions (to receive or send
% a field by respectively) and termination. 
 
% In OASIS, the communication follows the ``end-point'' principle, i.e.
% there is no reference in the component model code to the origin of a
% Get action or to the destination of a Put action; the source and
% target component models (coupling exchange) or the input or output
% file (I/O) are set externally by the user.  This ensures an easy
% transition between different coupling configurations, in particular
% from the coupled mode (Get/Put actions leading to a coupling exchange
% performed using MPI) to the forced mode (Get/Put action leading to
% reading/writing from/to a file using the GFDL mpp\_io library, Balaji
% 2001), totally transparent for the component model itself.
% Furthermore, the Get/Put routines can be called at each time step in
% the component model code but the receiving/sending actions will
% effectively be performed only at appropriate times from/to the
% appropriate source/target following the configuration externally
% defined by the user.

% For each coupling exchange, OASIS performs the transformations and
% regridding needed to express the source field on the grid of the
% target model. The following 2D regridding algorithms based on the
% SCRIP library (Jones 1999) are available (for more details, the reader
% is referred to the SCRIP User Guide available at
% http://climate.lanl.gov/Software/SCRIP/):

% \begin{itemize}
% \item N nearest-neighbour: the N closest source neighbours are used.
%   The weight of each neighbour is inversely proportional to $d$, its
%   Great Circle distance to the target point, or to $exp(-1/2 \cdot
%   d^2/\sigma^2)$ where $\sigma^2$ is the variance of a Gaussian function.
% \item bilinear: the 4 enclosing source neighbour points are used and
%   their respective weight is evaluated using a general bilinear
%   iteration in a continuous local coordinate system.
% \item bicubic: the value of the source field, its gradients and cross
%   gradient with respect to the local directions {\it i} and {\it j} at
%   the 4 enclosing source neighbour points are used. For Reduced
%   Gaussian grid, a standard bicubic algorithm with the 16 enclosing
%   source neighbours is used.
% \item 1st order conservative remapping: the weight of a source cell is
%   proportional to area of the source cell intersected by target cell.
%   Using the divergence theorem, the SCRIP library evaluates this area
%   with the line integral along the cell borders enclosing the area. 

% \end{itemize}

% (XXX to be added: a word on subgrid, Interpolation of vectors)

\section{Step-by-step use of OASIS3}

To use OASIS3 for coupling models (and/or perform I/O
actions), one has to follow these steps:
\begin{enumerate}
\item Obtain OASIS3 sources (see chapter \ref{sec_Obtaining}).
\item Identify the coupling or I/O fields and adapt the component
  models to allow their exchange with the PSMILe library based on MPI1
  or MPI2 message passing\footnote{The SIPC, PIPE and GMEM
    communication techniques available in previous versions are not maintained anymore.}.
  The PSMILe library is interfaced with the {\tt mpp\_io} library from
  GFDL \cite{mpp_io} and therefore can be used to perform I/O actions
  from/to disk files.  For more detail on how to interface a model
  with the PSMILe, see chapter \ref{sec_modelinterfacing}.

The TOYOASIS3 coupled model gives a practical example of a coupled
model; the sources are given in directories {\tt
  /oasis3/examples/toyoasis3/src }; more detail on TOYOASIS3 and
how to compile and run it can be found in chapter
\ref{sec_compilationrunning}.

\item Define all coupling and I/O parameters and the transformations
  required to adapt each coupling field from its source model grid to
  its target model grid; on this basis, prepare OASIS3 configuring file {\it
  namcouple} (See chapter \ref{sec_namcouple}). 
  
  OASIS3 supports different interpolation algorithms as is described in
  chapter \ref{sec_transformations}. We strongly recommend that one tests off-line the quality
  of the chosen transformations and regriddings in the ``testNONE'' environment (see section \ref{subsec_running_testnone}).

\item Generate required auxiliary data files (see chapter
  \ref{sec_auxiliary}).
\item Compile OASIS3, the component models and start the coupled
  experiment. Chapter \ref{sec_compilationrunning} describes how to
  compile and run OASIS3 and the TOYOASIS3 coupled model.

\end{enumerate}

If you need extra help, do not hesitate to contact us (see contact
details on the back of the cover page).

