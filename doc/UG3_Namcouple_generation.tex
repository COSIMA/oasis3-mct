\newpage
\chapter{The configuration file {\it namcouple}}
\label{sec_namcouple}

The OASIS3-MCT configuration file {\it namcouple} contains, below
pre-defined keywords, all user's
defined information necessary to configure a particular coupled
run. 

The {\it namcouple} is a text file with the following
characteristics:

\begin{itemize}
\item the keywords used to separate the information
can appear in any order;
\item the number of blanks between two character strings is
non-significant; 
\item all lines beginning with \# are ignored and considered as
comments.
\item {\bf blank lines are not allowed.}
\end{itemize}

The first part of {\it namcouple } is devoted to configuration of
general parameters such as the number of models involved in the
simulation or the number of fields.
The second part gathers specific information on each coupling (or I/O)
field, e.g. their coupling period, the list of transformations or
interpolations to be performed by OASIS3-MCT and associated
configuring lines (described in more details in chapter
\ref{sec_transformations}), etc.

In OASIS3-MCT, several {\it namcouple} inputs have been deprecated but, for backwards compatibility, they are still 
allowed.  These inputs will be noted in the following text using the
notation ``UNUSED'' and not fully described.Information below these keywords is
  obsolete in OASIS3-MCT; it will not be read and will not be used. 

In the next sections, a simple {\it namcouple} example is given and
all configuring parameters are described. The additional lines
containing the different parameters required for each transformation
are described in section \ref{sec_transformations}. A
realistic {\it namcouple} can be found in 
{\tt
oasis3-mct/examples/tutorial/data\_oasis3/} directory.
% XXX and {\tt
%/prism/util/running\break/adjunct\_files/oasis3/namcouple\_toyclim\_use}.

\section{An example of a simple {\it namcouple}}
\label{subsec_examplenamcouple}

The following simple {\it namcouple} configures a run into which an
ocean, an atmosphere and an atmospheric chemistry models are
coupled. The ocean provides only the SOSSTSST field to the atmosphere,
which in return provides the field CONSFTOT to the ocean. One field
(COSENHFL) is exchanged from the atmosphere to the
atmospheric chemistry, and one field (SOALBEDO) is read from a file by
the ocean.

\begin{verbatim}

######################################################################
# First section
#
 $SEQMODE
#
 $CHANNEL
#
 $NFIELDS
    4  
#
 $JOBNAME
#
 $NBMODEL
    3  ocemod   atmmod  chemod  55  70   99 
#
 $RUNTIME
    432000
#
 $INIDATE
#
 $MODINFO
#
 $NLOGPRT
   2
#
 $CALTYPE
#
######################################################################
# Second section 
#
 $STRINGS
#
# Field 1
 SOSSTSST SISUTESU 1 86400  5  sstoc.nc  EXPORTED
 182  149  128  64  toce  atmo   LAG=+14400  SEQ=+1
 P 2 P 0
 LOCTRANS CHECKIN MAPPING  BLASNEW CHECKOUT 
#
  AVERAGE 
  INT=1
  map_toce_atmo_120315.nc src opt
  CONSTANT     273.15 
  INT=1
#
# Field 2
 CONSFTOT SOHEFLDO 6 86400  4   flxat.nc  EXPORTED
 atmo   toce  LAG=+14400  SEQ=+2
 P 0 P 2
 LOCTRANS  CHECKIN  SCRIPR CHECKOUT
#
  ACCUMUL 
  INT=1
  BILINEAR LR SCALAR LATLON 1
  INT=1
#
# Field 3
 COSENHFL  SOSENHFL  37  86400   1  flda3.nc  IGNOUT 
 atmo   atmo LAG=+7200 
 LOCTRANS
 AVERAGE
#
# Field 4
 SOALBEDO SOALBEDO  17  86400  0  SOALBEDO.nc  INPUT
#
#####################################################################
\end{verbatim}

%section{An example of a simple {\it namcouple}}

\section{ First section of {\it namcouple} file}
\label{subsec_namcouplefirst}

The first section of {\it namcouple } uses some predefined keywords
prefixed by the \$ sign to locate the related information. The
\$ sign must be in the second column. The first ten keywords
are described hereafter:

\begin{itemize}

\item {\tt \$SEQMODE}: UNUSED 

\item {\tt \$CHANNEL}: UNUSED

\item {\tt \$NFIELDS}: On the line below this keyword is the total
number of fields exchanged and described in the second part of
the {\it namcouple}. 

\item {\tt \$JOBNAME}: UNUSED

\item {\tt \$NBMODEL}: On the line below this keyword is the number of
models running in the given experiment followed by {\tt
CHARACTER$\star$6} variables giving their names, which must correspond
to the name announced by each model when calling {\tt
  oasis\_init\_comp} or {\tt prism\_init\_comp\_proto} (second argument, see section \ref{subsubsec_Initialisation}). 

Then the user may
indicate on the same line the maximum Fortran unit number used by the models. In  the
example, Fortran units above 55, 70, and 99 are free for respectively
the ocean, atmosphere, and atmospheric chemistry models. {\bf In all cases,
OASIS3-MCT library assumes, during the initialization phase, that units 1025 and 1026 are free and
temporarily uses these units to read the {\it
  namcouple} and to write corresponding log messages to file {\tt
  nout.000000}.} After the initialization phase, OASIS3-MCT will
still suppose that units above 1024 are free, unless
maximum unit numbers are indicated here in the {\it namcouple}.
%If {\tt \$CHANNEL}
%is {\tt NONE}, {\tt \$NBMODEL} has to be 0 and there should be no
%model name and no unit number.

\item {\tt \$RUNTIME}: On the line below this keyword is the total
simulated time of the run, expressed in seconds. 
%If {\tt \$CHANNEL}
%is {\tt NONE}, {\tt \$RUNTIME} has to be the number of time occurrences
%of the field to interpolate from the restart file. 

\item {\tt \$INIDATE}: UNUSED

\item {\tt \$MODINFO}: UNUSED
 
\item {\tt \$NLOGPRT}: The line below this keyword refers to the
  amount of information that will be written to the OASIS3-MCT debug files
  for each model and processor. The value of {\tt \$NLOGPRT} may be:  
  \begin{itemize}
  \item 0 : one file debug.root.xx is open by the master proces of
    each model and one file debug\_notroot.xx is open for all the
    other processes of each model to write only error information.
  \item 1 : one file debug.root.xx is open by the master proces of each model to write information equivalent to level 10 (see below); one file debug\_notroot.xx is open for all the other processes of each model to write error information.
  \item 2 : one file debug.xx.xxxxxx is open by each process of each model to write normal production diagnostics
  \item 5 : as for 2 with in addition some initial debug info
  \item 10: as for 5 with in addition the routine calling tree
  \item 12: as for 10 with in addition some routine calling notes
  \item 15: as for 12 with even more debug diagnostics
  \item 20: as for 15 with in addition some extra runtime analysis
  \item 30: full debug information
 \end{itemize}
This value can be changed at runtime with the {\tt oasis\_set\_debug}
or {\tt prism\_set\_debug} routine (see section \ref{subsubsec_auxroutines}).

\item {\tt \$CALTYPE}: UNUSED

\end{itemize}

%{Description of {\it namcouple} first section}

\section{Second section of {\it namcouple} file }
\label{subsec_namcouplesecond}

The second part of the {\it namcouple}, starting after the keyword
{\tt \$STRINGS}, contains coupling information for each coupling (or
I/O) field.  Its format depends on the field status given by the last
entry on the field first line ({\tt EXPORTED}, {\tt IGNOUT} or {\tt
INPUT} in the example above). The field may be :

\begin{itemize}
\item {\tt AUXILARY}: UNUSED
\item {\tt EXPORTED}: exchanged between component models and
  transformed by OASIS3-MCT
\item {\tt EXPOUT}: exchanged, transformed and also written to two
  debug NetCDF files, one before the sending action in the source model
  below the {\tt oasis\_put} or {\tt prism\_put\_proto} call (after local transformations {\tt LOCTRANS} and {\tt BLASOLD} if present), and one after the receiving
  action in the target model below the {\tt prism\_get\_proto} call (after all transformations). This status should
  be used when debugging the coupled model only. The name of the debug NetCDF file (one per field) is
  automatically defined based on the field and component model names.
\item {\tt IGNORED}: with OASIS3-MCT, this setting is equivalent to and converted to EXPORTED
\item {\tt IGNOUT}: with OASIS3-MCT, this setting is equivalent to and converted to EXPOUT
\item {\tt INPUT}: read in from the input file by the target
  model below the {\tt oasis\_get} or {\tt prism\_get\_proto} call at appropriate
  times corresponding to the input period indicated by the user in the
  {\it namcouple}. See section
  \ref{subsec_inputdata} for the format of the input file.
\item {\tt OUTPUT}: written out to an output debug NetCDF file by the source
  model below the {\tt oasis\_put} or {\tt prism\_put\_proto} call, after local transformations {\tt LOCTRANS} and {\tt BLASOLD}, at appropriate
  times corresponding to the output period indicated by the user in
  the {\it namcouple}. 

\end{itemize}

\subsection{Second section of {\it namcouple} for {\tt EXPORTED} and {\tt EXPOUT} fields}
\label{subsubsec_secondEXPORTED}

  The first 3 lines for fields with status {\tt EXPORTED} and 
  {\tt EXPOUT} are as follows:
  \begin{verbatim}
   SOSSTSST SISUTESU 1 86400  5  sstoc.nc  sstat.nc EXPORTED
   182  149    128  64  toce  atmo   LAG=+14400 SEQ=+1
   P 2 P 0 
  \end{verbatim}
  where the different entries are:
    \begin{itemize}
      \item Field first line:
        \begin{itemize}
        \item {\tt SOSSTSST} : symbolic name for the field in the
              source model ({\tt CHARACTER*8}). It has to match the
              argument {\tt name} of the corresponding field
              declaration in the source model; see {\tt
                oasis\_def\_var} or {\tt
              prism\_def\_var\_proto} in section
              \ref{subsubsec_Declaration}.
        \item {\tt SISUTESU} : symbolic name for the field in the
              target model ({\tt CHARACTER*8}).  It has to match the
              argument {\tt name} of the corresponding field
              declaration in the target model; see {\tt
                oasis\_def\_var} or {\tt
              prism\_def\_var\_proto} in section
              \ref{subsubsec_Declaration}.
        \item 1 : UNUSED but still required for parsing
        \item 86400 : coupling and/or I/O period for the field, in
        seconds. 
        \item 5 : number of transformations to be performed by OASIS3 on this field.  
        \item sstoc.nc : name of the coupling restart file for the
          field ({\tt CHARACTER*8}); 
          mandatory even if no coupling restart file is effectively
          used. (for more detail, see section \ref{subsec_restartdata});
        \item sstat.nc : UNUSED but still required for parsing
        \item {\tt EXPORTED} : field status.
        \end{itemize}
      \item Field second line:
        \begin{itemize}
        \item 182 : number of points for the source grid first
        dimension (optional)\footnote{The 2D dimensions of the grids
          must be provided in the {\it namcouple} so to have 2D fields
          in the debug files; otherwise, the fields in the debug files
          will be 1D.}. 
        \item 149 : number of points for the source grid second
        dimension (optional)$^{1}$.    
        \item 128 : number of points for the target grid first
        dimension (optional)$^{1}$. 
        \item 64 : number of points for the target grid second
        dimension (optional)$^{1}$.  
        \item toce : prefix of the source grid name in grid data files (see section
        \ref{subsec_griddata}) ({\tt CHARACTER*4})
        \item atmo : prefix of the target grid name in grid data files
        ({\tt CHARACTER*4})
        \item {\tt LAG=+14400}: optional lag index for the field
        expressed in seconds 
        \item {\tt SEQ=+1}: optional sequence index for the field (see section  \ref{subsubsec_Algoritms})
        \end{itemize}
      \item Field third line
        \begin{itemize}
         \item P : source grid first dimension characteristic
            (`P': periodical; `R': regional).
         \item 2 : source grid first dimension number of overlapping grid points.
         \item P : target grid first dimension characteristic (`P':
         periodical; `R': regional).
         \item 0 : target grid first dimension number of overlapping grid points.
        \end{itemize}
     
      \end{itemize}
    
  The fourth line gives the list of transformations to be performed
  for this field. In addition, there is one or more configuring lines
  describing some parameters for each transformation. These
  additional lines are described in more details in the chapter
  \ref{sec_transformations}.

\subsection{Second section of {\it namcouple} for {\tt OUTPUT} fields}
\label{subsubsec_secondOUTPUT}
  The first 2 lines for fields with status {\tt OUTPUT} are as follows:
  \begin{verbatim}
  COSHFTOT  COSHFTOT   7   86400  0  fldhftot.nc OUTPUT 
  atmo   atmo 
  \end{verbatim}
where the different entries are as for {\tt EXPOUT} fields, except
that the source symbolic name must be repeated twice on the field
first line, the restart file name is needed only if {\tt LOCTRANS}
transformations are present, there is no output file
name on the first line and no LAG or  SEQ index at the end of the second line.
The name of the output file is automatically defined based on the field and component model names.

The third line is {\tt LOCTRANS} if this transformation is chosen for
the field. Note that {\tt LOCTRANS} is the only transformation
supported for {\tt OUTPUT} fields.

\subsection{Second section of {\it namcouple} for {\tt
  INPUT} fields}
\label{subsubsec_secondINPUT}

  The first and only line for fields with status {\tt INPUT} is:

  \begin{verbatim}
  SOALBEDO SOALBEDO  17  86400  0  SOALBEDO.nc  INPUT\end{verbatim} where the different entries are:
  \begin{itemize}
  \item  {\tt SOALBEDO}: symbolic name for the field in the target
  model ({\tt CHARACTER*8} repeated twice)
  \item 17:  index in auxiliary file cf\_name\_table.txt (see above for EXPORTED fields)
  \item 86400: input period in seconds
  \item 0: number of transformations (always 0 for {\tt INPUT} fields)
  \item {\tt SOALBEDO.nc}: {\tt CHARACTER*32} giving the input file
  name (for more detail on its format, see section
  \ref{subsec_inputdata})
  \item {\tt INPUT}: field status.
  \end{itemize}

