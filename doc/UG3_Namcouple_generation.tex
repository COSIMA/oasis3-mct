\newpage
\chapter{The configuration file {\it namcouple}}
\label{sec_namcouple}

The OASIS3-MCT configuration file {\it namcouple} contains, below
pre-defined keywords, all user-defined information necessary to
configure a particular coupled run.

A Graphical User Interface is now available to build
the {\it namcouple} for one's specific coupled
system. The OASIS3-MCT GUI is an application of OPENTEA, a graphical
interface used by different codes and developed at CERFACS. The
sources of the GUI are located in the {\tt oasis3-mct/util/oasisgui}
directory; a short description of the OASIS GUI is given in the README
file therein and the GUI itself includes all relevant explanations
on-line.

The {\it namcouple} is a text file with the following characteristics:

\begin{itemize}
\item the keywords used to separate the information can appear in any
  order;
\item the number of blanks between two character strings is
  non-significant;
\item all lines beginning with \# are ignored and considered as
  comments;
\item blank lines are supported, but only since OASIS3-MCT4.0 version.
\end{itemize}

The first part of {\it namcouple } is devoted to configuration of
general parameters such as the total run time or the desired debug level.  
The second part gathers specific
information on each coupling (or I/O) field, e.g. their coupling
period, the list of transformations or interpolations to be performed
by OASIS3-MCT and associated configuring lines (described in more
details in chapter \ref{sec_transformations}).

In OASIS3-MCT, several {\it namcouple} inputs have been deprecated
but, for backwards compatibility, they are still allowed.  These
inputs will be noted in the following text using the notation
``UNUSED'' and not fully described. Information below these keywords
is obsolete; it will not be read and will not be used.

In the next sections, a {\it namcouple} example is given and
all configuring parameters are described. Additional lines
containing different parameters for each transformation
are described in section \ref{sec_transformations}. A realistic {\it
  namcouple} can be found in {\tt
  oasis3-mct/examples/tutorial/data\_oasis3/} directory.

\section{An example of a simple {\it namcouple}}
\label{subsec_examplenamcouple}

The following simple {\it namcouple} configures a run into which e.g. an
ocean, an atmosphere and an atmospheric chemistry components are
coupled. The ocean running on grid {\tt toce} provides only the SOSSTSST field to the atmosphere (grid {\tt atmo}),
which in return provides the field CONSFTOT to the ocean. One field
COSENHFL is exchanged from the atmosphere to the atmospheric
chemistry (also running on grid {\tt atmo}), and one field SOALBEDO is read from a file by the ocean.

\begin{verbatim}
########## First section #############################################
 $NFIELDS
    4  
#
 $RUNTIME
    432000
#
 $NLOGPRT
   2     1
#
 $NUNITNO
   901     920
#
 $SEQMODE
 $CHANNEL
 $JOBNAME
 $NBMODEL
 $INIDATE
 $MODINFO
 $CALTYPE
#
########## Second section #############################################
#
 $STRINGS
#
# Field 1
 SOSSTSST SISUTESU 1 86400  5  sstoc.nc  EXPORTED
 182  149  128  64  toce  atmo   LAG=+14400  SEQ=+1
 P 2 P 0
 LOCTRANS CHECKIN MAPPING  BLASNEW CHECKOUT 
#
  AVERAGE 
  INT=1
  map_toce_atmo_120315.nc src opt
  1.0  1
  CONSTANT     273.15 
  INT=1
#
# Field 2
 CONSFTOT SOHEFLDO 6 86400  4   flxat.nc  EXPORTED
 atmo   toce  LAG=+14400  SEQ=+2
 P 0 P 2
 LOCTRANS  CHECKIN  SCRIPR CHECKOUT
#
  ACCUMUL 
  INT=1
  BILINEAR LR SCALAR LATLON 1
  INT=1
#
# Field 3
 COSENHFL  SOSENHFL  37  86400   1  flda3.nc  IGNOUT 
 atmo   atmo LAG=+7200 
 LOCTRANS
 AVERAGE
#
# Field 4
 SOALBEDO SOALBEDO  17  86400  0  SOALBEDO.nc  INPUT
#####################################################################
 $END
\end{verbatim}

% section{An example of a simple {\it namcouple}}

\section{ First section of {\it namcouple} file}
\label{subsec_namcouplefirst}

The first section of {\it namcouple } uses some predefined keywords
prefixed by the \$ sign to locate the related information. The \$ sign
must be the first non-blank character on the line but can be in any column.
Only 5 keywords are really used since OASIS3-MCT\_3.0 version and 2 of these 5 are optional :

\begin{itemize}

\item {\tt \$NFIELDS}: On the line below this keyword, put a number equal (or greater) to the total
  number of field entries in the second part of the {\it
    namcouple}. If more than one field are described on the same line, this
  counts as only one entry.

%\item {\tt \$NBMODEL}: On the line below this keyword is the number of
%  models running in the given experiment followed by {\tt
%    CHARACTER$\star$6} variables giving their names, which must
%  correspond to the name announced by each model when calling {\tt
%    oasis\_init\_comp} (second argument, see section
%  \ref{subsubsec_Initialisation}).
%
%  Then the user may indicate on the same line the maximum Fortran unit
%  number used by the models. In the example, Fortran units above 55,
%  70, and 99 are free for respectively the ocean, atmosphere, and
%  atmospheric chemistry models. {\bf In all cases, OASIS3-MCT library
%    assumes, during the initialization phase, that units 1025 and 1026
%    are free and temporarily uses these units to read the {\it
%      namcouple} and to write corresponding log messages to file {\tt
%      nout.000000}.} After the initialization phase, OASIS3-MCT will
%  still suppose that units above 1024 are free, unless maximum unit
%  numbers are indicated here in the {\it namcouple}.
%  % If {\tt \$CHANNEL} is {\tt NONE}, {\tt \$NBMODEL} has to be 0 and
%  % there should be no model name and no unit number.

\item {\tt \$RUNTIME}: On the line below this keyword, put the total
  simulated time of the run, expressed in seconds (or any other time
  units as long as the same are used in all components and in the {\it
    namcouple}, see \ref{subsubsec_sendingreceiving}).
  % If {\tt \$CHANNEL} is {\tt NONE}, {\tt \$RUNTIME} has to be the
  % number of time occurrences of the field to interpolate from the
  % restart file.
 
\item {\tt \$NLOGPRT}: The first and second numbers on the line below
  this keyword refer to the amount of debug and time statistic
  information written by OASIS3-MCT for each component and process.

  The first number (that can be modified at runtime with the {\tt
    oasis\_set\_debug} routine, see section
  \ref{subsubsec_auxroutines}) may be:
  \begin{itemize}
  \item 0 : production mode. One file debug.root.xx is open by the master process of
    each component and one file debug\_notroot.xx is open for all the
    other processes of each component to write only error information.
  \item 1 : one file debug.root.xx is open by the master process of
    each component to write information equivalent to level 10 (see
    below) and also to write memory usage information; 
    one file debug\_notroot.xx is open for all the other
    processes of each component to write error information.
  \item 2 : one file debug.yy.xxxxxx is open by each process of each
    component (with ``yy” being the component number and ``xxxxxx” the process number) 
    to write normal production diagnostics and memory usage information
  \item 5 : as for 2 with in addition some initial debug info
  \item 10: as for 5 with in addition the routine calling tree
  \item 12: as for 10 with in addition some routine calling notes
  \item 15: as for 12 with even more debug diagnostics
  \item 20: as for 15 with in addition some extra runtime analysis
  \item 30: full debug information
  \end{itemize}
  The second number defines how time statistics are written out to
  file {\it comp\_name}.timers\_xxxx (with {\it comp\_name} being the component name, see section \ref{init_comp}); it can be:
  \begin{itemize}
  \item 0 : nothing is calculated or written.
  \item 1 : some time statistics are calculated and written in a
    single file by the processor 0 as well as the min and the max
    times over all the processors.
  \item 2 : some time statistics are calculated and each processor
    writes its own file ; processor 0 also writes the min and the max
    times over all the processors in its file.
  \item 3 : some time statistics are calculated and each processor
    writes its own file ; processor 0 also writes in its file the min
    and the max times over all processors and also writes in its file
    all the results for each processor.
  \end{itemize}
 For more information on the time statistics written out, see section
  \ref{timestat}.

The second number can also be set to -1 to activate the {\it lucia}
tool that can be used to perform an analysis of the coupled components
load balance. More information can be found in the README file in {\tt
  oasis3-mct/util/lucia} directory and report mentioned therein.
 
\item {\tt \$NUNITNO}: Optional; on the line below this keyword are two integers
  that indicate the minimum and maximum unit numbers to be used for
  input and output files in the coupling layer.  The user should
  choose values that will NOT conflict or overlap with unit numbers in 
  use in any of the component models. The defaults are 1024 for the minimum and 9999
  for the maximum unit number if not explicitly set by the user.

\item {\tt \$NNOREST}: Optional; on the line below this keyword is a character
  string that can override the requirement that restart files must exist
  if they are needed.  If the character string value starts with T, t, .T, 
  or .t (as in true), then OASIS3-MCT will initialise with zero any variable that normally requires
  a restart (for instance, variables with LAG > 0) if the restart file does not exist. By default, missing
  restart files will cause the model to abort.  It is strongly recommended
  that this keyword NOT be used in production runs.  It exists to provide a 
  quick shortcut for generating initial restarts or running technical tests. 
  Note that if {\tt \$NNOREST{} is true but the restart file nonetheless exists, it will be used.
 
\item {\tt \$SEQMODE, \$CHANNEL, \$JOBNAME, \$NBMODEL, \$INIDATE, \$MODINFO, \$CALTYPE:} UNUSED

\end{itemize}

% {Description of {\it namcouple} first section}

\section{Second section of {\it namcouple} file }
\label{subsec_namcouplesecond}

The second part of the {\it namcouple}, starting after the keyword
{\tt \$STRINGS}, contains coupling information for each coupling (or
I/O) field.  Its format depends on the field status given by the last
entry on the field first line ({\tt EXPORTED}, {\tt IGNOUT} or {\tt
  INPUT} in the example above). The field may be :

\begin{itemize}
\item {\tt AUXILARY}: UNUSED
\item {\tt EXPORTED}: exchanged between components and
  transformed by OASIS3-MCT
\item {\tt EXPOUT}: exchanged, transformed and also written to two
  debug NetCDF files, one before the sending action in the source
  component below the {\tt oasis\_put} call (after local transformations
  {\tt LOCTRANS} and {\tt BLASOLD} if present), and one after the
  receiving action in the target component below the {\tt oasis\_get} call
  (after all transformations). {\tt EXPOUT} should be used only when
  debugging the coupled model. The name of the debug NetCDF file
  (one per field) is automatically defined based on the field and
  component names.
\item {\tt IGNORED}: with OASIS3-MCT, this setting is equivalent to
  and converted to EXPORTED
\item {\tt IGNOUT}: with OASIS3-MCT, this setting is equivalent to and
  converted to EXPOUT
\item {\tt INPUT}: read in from the input file by the target component
  below the {\tt oasis\_get} call at
  appropriate times corresponding to the input period indicated by the
  user in the {\it namcouple}. See section \ref{subsec_inputdata} for
  the format of the input file.
\item {\tt OUTPUT}: written out to an output debug NetCDF file by the
  source component below the {\tt oasis\_put} call, after local
  transformations {\tt LOCTRANS} and {\tt BLASOLD}, at appropriate
  times corresponding to the output period indicated by the user in
  the {\it namcouple}.

\end{itemize}

\subsection{Second section of {\it namcouple} for {\tt EXPORTED} and
  {\tt EXPOUT} fields}
\label{subsubsec_secondEXPORTED}

The first 3 lines for fields with status {\tt EXPORTED} and {\tt
  EXPOUT} are as follows:
  \begin{verbatim}
   SOSSTSST SISUTESU 1 86400  5  sstoc.nc  EXPORTED
   182  149    128  64  toce  atmo   LAG=+14400 SEQ=+1
   P 2 P 0 
\end{verbatim}
%\vspace{-0.2cm} 
where the different entries are:
\begin{itemize}
\item Field first line:
  \begin{itemize}

  \item {\tt SOSSTSST} : symbolic name for the field in the source
    component (80 characters maximum). It has to match the argument {\tt name}
    of the corresponding field declaration in the source component; see
    {\tt oasis\_def\_var} in section \ref{subsubsec_Declaration}
  \item {\tt SISUTESU} : symbolic name for the field in the target
    component (80 characters maximum).  It has to match the argument {\tt
      name} of the corresponding field declaration in the target
    component; see {\tt oasis\_def\_var} in section
    \ref{subsubsec_Declaration}
  \item 1 : UNUSED but still required for parsing
  \item 86400 : coupling and/or I/O period for the field, in seconds
  \item 5 : number of transformations to be performed by OASIS3 on
    this field
  \item sstoc.nc : name of the coupling restart file for the field
    (32 characters maximum); mandatory even if no coupling restart file is
    effectively used (for more detail, see section
    \ref{subsec_restartdata})
  \item {\tt EXPORTED} : field status
  \end{itemize}
\item Field second line:
  \begin{itemize}
  \item 182 : number of points for the source grid first dimension
    (optional)
  \item 149 : number of points for the source grid second dimension
    (optional)\footnote{For 1D field, put {\tt 1} as the second dimension}
  \item 128 : number of points for the target grid first dimension
    (optional)
  \item 64 : number of points for the target grid second dimension
    (optional)$^{1}$

    These source and target grid dimensions are optional but note that
    in order to have 2D fields written as 2D arrays in the debug
    files, these dimensions must be provided in the {\it namcouple};
    otherwise, the fields will be written out as 1D arrays.
  
  \item toce : prefix of the source grid name in grid data files (see
    section \ref{subsec_griddata}) (80 characters maximum)
  \item atmo : prefix of the target grid name in grid data files (80 characters maximum) 
  \item {\tt LAG=+14400}: optional lag index for the field (see section \ref{subsub_lag})
  \item {\tt SEQ=+1}: optional sequence index for the field (see
    section \ref{subsec_sec})
  \end{itemize}
\item Field third line
  \begin{itemize}
  \item P : source grid first dimension characteristic (`P':
    periodical; `R': regional).
  \item 2 : source grid first dimension number of overlapping grid
    points.
  \item P : target grid first dimension characteristic (`P':
    periodical; `R': regional).
  \item 0 : target grid first dimension number of overlapping grid
    points.
  \end{itemize}
     
\end{itemize}
    
The fourth line gives the list of transformations to be performed for
this field. In addition, there is one or more configuring lines
describing some parameters for each transformation. These additional
lines are described in more details in the chapter
\ref{sec_transformations}.

{\bf Support to couple multiple fields via a single communication}

With OASIS3-MCT, it is possible to couple mutilple fields via a
single communication. To activate this option, the user must list the
related fields on a single entry line through a colon
delimited list in the {\it namcouple}\footnote{The maximum length of a line in the
{\it namcouple} is of 5000 characters.}, for example:

{\tt ATMTAUX:ATMTAUY:ATMHFLUX  TAUX:TAUY:HEATFLUX 1 3600 3 rstrt.nc EXPORTED}

All fields will then use the same
{\it namcouple} settings (source and target grids, transformations, etc.) for that entry. In the component model codes,
these fields are still apparently sent or received one at a
time through individual {\tt oasis\_put} and {\tt oasis\_get}. Inside OASIS3-MCT, the fields are stored and a single mapping
and send or receive instruction is executed for all fields. This is
useful in cases where multiple fields have the same coupling
transformations and to reduce communication costs by aggregating multiple 
fields into a single communication. 

This option does not put any constraint
on the order of the related {\tt oasis\_put} and {\tt oasis\_get} in the codes.

As they appear in one single entry line, these fields must share the same coupling restart file 
but this restart file may contain other fields.

\subsection{Second section of {\it namcouple} for {\tt OUTPUT} fields}
\label{subsubsec_secondOUTPUT}
The first 2 lines for fields with status {\tt OUTPUT} are as follows:
  \begin{verbatim}
  COSHFTOT  COSHFTOT   7   86400  0  fldhftot.nc OUTPUT 
  atmo   atmo 
\end{verbatim}
% \vspace{-0.2cm}
where the different entries are as for {\tt EXPOUT} fields, except
that:
\begin{itemize}
\item the source symbolic name must be repeated twice on the field
  first line,
\item the restart file name (here {\tt fldhftot.nc}) is needed only if
  a {\tt LOCTRANS} transformation is present,
\item there is no grid dimension\footnote{This means that all output
    fields will be written out in the output files as 1D arrays.} and no LAG or SEQ
  index on the second line; ;
\end{itemize}
The name of the output file is automatically defined based on the
field and component names.

The third line is {\tt LOCTRANS} if this transformation is chosen for
the field. Note that {\tt LOCTRANS} is the only transformation
supported for {\tt OUTPUT} fields.

\subsection{Second section of {\it namcouple} for {\tt INPUT} fields}
\label{subsubsec_secondINPUT}

The first and only line for fields with status {\tt INPUT} is:

  \begin{verbatim}
  SOALBEDO SOALBEDO  17  86400  0  SOALBEDO.nc  INPUT
  \end{verbatim}
\vspace{-0.5cm}
where the different entries are:
\begin{itemize}
\item {\tt SOALBEDO}: symbolic name for the field in the target component
  (80 characters maximum, repeated twice)
\item 17: index in auxiliary file cf\_name\_table.txt (see above for
  EXPORTED fields)
\item 86400: input period in seconds
\item 0: number of transformations (always 0 for {\tt INPUT} fields)
\item {\tt SOALBEDO.nc}:  the input file name (32 characters maximum)
  (for more detail on its format, see section \ref{subsec_inputdata})
\item {\tt INPUT}: field status.
\end{itemize}

