\newpage
%

\chapter{Compiling, running and debugging}
\label{sec_compilationrunning}

\section{Compiling OASIS3-MCT}
\label{subsec_compile}

Compiling OASIS3-MCT can be done in directory {\tt oasis3-MCT/util/make\_dir}
with Makefile \\ {\tt TopMakefileOasis3} which must be completed with a header file {\tt
  make.{\it your\_platform}} specific to the compiling platform used
and specified in {\tt oasis3-MCT/util/make\_dir/make.inc}.  One of the
header files distributed with the release can by used as a template.  The root 
of the OASIS3-MCT tree
can be anywhere and must be set in the variable {\tt COUPLE} in the
{\tt make.{\it your\_platform}} file. 

The following commands are available:

\begin{itemize}
\item {\tt make -f TopMakefileOasis3} or {\tt make oasis3\_psmile -f
  TopMakefileOasis3} (for backward compatibility):

  compiles all OASIS3-MCT libraries {\it mct}, {\it scrip} and {\it psmile}; 

\item {\tt make realclean -f  TopMakefileOasis3}: 

  removes all OASIS3-MCT compiled sources and librairies.

\end{itemize}

Log and error messages from compilation are saved in the files
COMP.log and COMP.err in make\_dir.

During compilation, a new compiling directory, defined by variable {\tt ARCHDIR}
is created.  After successful
compilation, resulting executables are found in the compiling directory in {\tt /bin}, libraries in {\tt /lib} and object
and module files in {\tt /build}.

The following CPP keys are coded in OASIS3-MCT and
can be specified in {\tt CPPDEF} in {\tt make.{\it your\_platform}} file.

\begin{itemize}

\item {\tt TREAT\_OVERLAY}: To ensure, in {\tt SCRIPR/CONSERV} remapping (see section
  \ref{subsec_interp}), that if two cells of the source grid overlay,
  at least the one with the greater numerical index is masked (they
  also can be both masked); this is mandatory for this remapping. For
  example, if the grid line with i=1 overlaps the grid line with
  i=imax, it is the latter that must be masked; when this is not the
  case with the mask defined in {\it masks.nc}, this CPP key forces
  these rules are to be respected.
\end{itemize}

\section{Debugging}

If you experience problems while running your coupled model with
OASIS3, you can obtain more information on what is happening by increasing {\tt \$NLOGPRT} in your {\it namcouple} (see section \ref{subsec_namcouplefirst}).

\subsection{Running OASIS3-MCT}

A practical example on how to run OASIS3-MCT and running it in a
coupled system is provided in {\tt oasis3-mct/examples/tutorial}. For
more details, see the README there in.

