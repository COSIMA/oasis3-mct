\newpage
%

\chapter{Compiling, running and debugging}
\label{sec_compilationrunning}

\section{Compiling OASIS3-MCT}
\label{subsec_compile}

OASIS3-MCT is an MPI code. Compiling OASIS3-MCT libraries can be done
in {\tt oasis3-MCT/util/make\_dir} with Makefile {\tt TopMakefileOasis3}
which must include a header file {\tt make.{\it
    your\_platform}} specific to the compiling platform used and
specified in {\tt oasis3-MCT/util/make\_dir/make.inc}.  

One of the
header files distributed with the release can by used as a template
in particular to see how to use MPI librairies and NetCDF librairies.
The root of the OASIS3-MCT tree can be anywhere and must be set in the
variable {\tt COUPLE} in the {\tt make.{\it your\_platform}} file.

The coupled models must be compiled with the same libraries as the ones
used to compile OASIS3-MCT.

The following commands are available:

\begin{itemize}
\item {\tt make -f TopMakefileOasis3} or {\tt make oasis3\_psmile -f
    TopMakefileOasis3} (for backward compatibility):

  compiles all OASIS3-MCT libraries {\it mct}, {\it scrip} and {\it
    psmile};

\item {\tt make realclean -f TopMakefileOasis3}:

  removes all OASIS3-MCT compiled sources and librairies.

\end{itemize}

Log and error messages from compilation are saved in {\tt /util/make\_dir} in the files
{\tt COMP.log} and {\tt COMP.err} .

During compilation, a new compiling directory, defined by variable
{\tt ARCHDIR} is created.  After successful compilation, resulting
libraries are found in the compiling directory in {\tt /lib} and
object and module files in {\tt /build}. 

To interface  a component code with OASIS3-MCT and use the module {\tt mod\_oasis} (see section \ref{subsubsec_Use}), it is enough to include the path of the psmile objects and
modules ({\tt \$ARCHDIR/build/lib/psmile.MPI1}) during the compilation and to use the psmile library {\tt \$ARCHDIR/lib/libpsmile.MPI1.a} when
linking\footnote{If module {\tt mod\_prism} is used in the models, it is necessary to
include the path of the psmile and of the mct objects and modules for
the compilation (i.e. {\tt \$ARCHDIR/build/lib/psmile.MPI1} and {\tt /mct} and to use both the psmile and mct libraries {\tt \$ARCHDIR/lib/libpsmile.MPI1.a} and {\tt libmct.a} and {\tt libmpeu.a} when linking.}.

Exchange of coupling fields in single and double precision is now supported directly through the interface 
(see section \ref{subsubsec_Declaration}), even if all real variables are now treated in double precision internally.
For double precision coupling field, there is no need anymore to promote REAL variables to double precision at compilation.

\section{CPP keys}
\label{subsec_cpp}

The first CPP key coded in OASIS3-MCT and that can be specified in
{\tt CPPDEF} in {\tt make.{\it your\_platform}} file is {\tt
  TREAT\_OVERLAY} that ensures, in {\tt SCRIPR/CONSERV}
  remapping (see section \ref{subsec_interp}), that if two cells of
  the source grid overlay and none is masked a priori, the one with the greater numerical
  index will not be considered (they also can be both masked); this is mandatory
  for this remapping. For example, if the grid line with {\it i=1} overlaps
  the grid line with {\it i=imax}, it is the latter that must be masked;
  when this is not the case with the mask defined in {\it masks.nc},
  this CPP key forces these rules to be respected.

The second CPP key coded in OASIS3-MCT and that {\bf must} be specified in
{\tt CPPDEF} in {\tt make.{\it your\_platform}} file if you compile
with {\bf PGF90} is {\tt \_\_NO\_16BYTE\_REALS}.

%\item {\tt balance}: Add a MPI\_Wtime() function before and after
%  mct\_isend (MPI put) and mct\_recv (MPI get) to calculate the time
%  of the send and receive of a coupling field. This option can be used
%  to produce timestamps in OASIS3-MCT debug files. During a post-processing
%  phase, this information can be used to perform an analysis of the
%  coupled components load (un)balance; specific tools have been
%  developed to do this and will be released with a further version of
%  OASIS3-MCT. {\bf This option is temporarily not recommended as it was observed that
%  it was increasing the simulation time of coupled models on
%  the PRACE computer MareNostrum.}

\section{Running OASIS3-MCT}
\label{subsec_running}

Examples of running environments are provided with the sources. See also the instructions on OASIS web site at
{\tt
  https://verc.enes.org/oasis/oasis-dedicated-user-support-1/user-toys}
for more details.

\subsection{The tutorial toy model}
\label{subsec_tutorial}

A practical example  is provided in {\tt
  oasis3-mct/examples/tutorial} reproducing the coupling between two
simple codes, model1 and model2, with OASIS3-MCT. 

See
the document {\tt tutorial.pdf} there in and the on-line
tutorial at \\
{\tt
  https://verc.enes.org/oasis/oasis-dedicated-user-support-1/user-toys}

{\tt
 /tutorial-and-test\_interpolation-of-oasis3-mct-1}
for more details.

\subsection{The test\_1bin\_ocnice toy model}
\label{subsec_1bin_ocnice}

Another practical example on how to use OASIS3-MCT is provided in \\ {\tt oasis3-mct/examples/test\_1bin\_ocnice}. 
This toy model reproduces the coupling between 5 sub-components within
the “ocnice” executable. See
the document {\tt test\_1bin\_ocnice.pdf} there in for more details.

\subsection{The test\_interpolation environment}
\label{subsec_testinterpolation}

This environment available with the sources in {\tt
  oasis3-mct/examples/test\_interpolation} allows the user to test the
quality of the interpolations and transformations between his source
and target grids by calculating the error of interpolation on the
target grid. For more details, see also the document {\tt test\_interpolation.pdf} there in.

\section{Debugging}
\label{subsec_debug}

\subsection{Debug files}
If you experience problems while running your coupled model with
OASIS3-MCT, you can obtain more information on what is happening by
increasing the {\tt \$NLOGPRT} value in your {\it namcouple} (see also section
\ref{subsec_namcouplefirst}).

Different outputs are written depending on {\tt \$NLOGPRT} value:
\begin{itemize}
\item {0} : production mode. One file debug.root.xx is opened by the master process of
  each model and one file debug\_notroot.xx is opened for all the other
  processes of each model to write only error information if an error
  occurs.
\item {1} : one file debug.root.xx is opened by the master process of
  each model to write information equivalent to level 10 (see below)
  and also to write memory usage information;
  one file debug\_notroot.xx is opened for all the other processes of
  each model to write only error information if an error occurs.
\item {2} : one file debug.xx.xxxxxx is opened by each process of each
  model to write normal production diagnostics .
\item {5} : as for 2 with in addition some initial debug info.
\item {10} : as for 5 with in addition the routine calling tree.
\item {12} : as for 10 with in addition some routine calling notes.
\item {15} : as for 12 with even more debug diagnostics and memory usage information.
\item {20} : as for 15 with in addition some extra runtime analysis.
\item {30} : full debug information.
\end{itemize}

\subsection{Time statistics files}
\label{timestat}

The variable TIMER\_Debug, defined in the {\it namcouple} (second
number on the line below \$NLOGPRT keyword), is used to obtain time
statistics over all the processors for each routine.

Different output are written (in files named {\tt *.timers\_xxxx})
depending on TIMER\_Debug value :

\begin{itemize}
\item {TIMER\_Debug=0} : nothing is calculated, nothing is written.
\item {TIMER\_Debug=1} : the times are calculated and written in a
  single file by the process 0 as well as the min and the max times
  over all the processes.
\item {TIMER\_Debug=2} : the times are calculated and each process
  writes its own file ; process 0 also writes the min and the max
  times over all the processes in its file.
\item {TIMER\_Debug=3} : the times are calculated and each process
  writes its own file ; process 0 also writes in its file the min
  and the max times over all processes and also writes in its file
  all the results for each process.
\end{itemize}

Note that TIMER\_Debug can also be set to -1 to activate the {\it lucia}
tool that can be used to perform an analysis of the coupled components
load balance. More information can be found in the README file in {\tt
  oasis3-mct/util/lucia} directory and report mentioned therein.
 
The time given for each timer is calculated by the difference between
calls to {\tt oasis\_timer\_start()} and {\tt oasis\_timer\_stop()}
and is the accumulated time over the entire run. Here is an overview
of the meaning of the different timers as implemented by default.
\footnote{Many other measures can be obtained by defining the logical
{\tt local\_timer} XXXX as {\tt .true.} in different routines or by
implementing other timers. Of course, OASIS3\_MCT and the model code
would then have to be recompiled.}

\begin{itemize}

\item {'total'} : total time of the simulation, implemented
  in {\tt mod\_oasis\_method}, i.e. between the end of {\tt
    oasis\_init\_comp} and the {\tt
    mpi\_finalize} in routine {\tt oasis\_terminate}.

\item {'init\_thru\_enddef'} : time between the end of {\tt
    oasis\_init\_comp} and the end of {\tt oasis\_enddef}, implemented
  in {\tt mod\_oasis\_method}.

\item {'part\_definition'} : time spent in routine {\tt oasis\_def\_partition}.

\item {'oasis\_enddef'} : time spent in 
  routine {\tt oasis\_enddef}; this routine performs basically all the
  important steps to initialize the coupling exchanges, e.g. the
  internal management of the partition and variable definition, the
  definition of the patterns of communication between the source and
  target processes, the reading of the regridding weight-and-address
  file and the initialisation of the sparse matrix vector multiplication.
\item {'grcv\_00x'} : time spent in the reception of field x in {\tt
    mct\_recv} (including communication and possible waiting time
  linked to unbalance of components).
\item {'wout\_00x'} : time spent in the I/O for field x in routine
  {\tt oasis\_advance\_run}.
\item {'gcpy\_00x'} : time spent in routine {\tt oasis\_advance\_run}
  in copying the field x just received in a local array.
\item {'pcpy\_00x'} : time spent in routine {\tt oasis\_advance\_run}
  in copying the local field x in the array to send (i.e. with local
  transformation besides division for averaging).
\item {'pavg\_00x'} : time spent in routine {\tt oasis\_advance\_run}
  to calculate the average of field x (if done).
\item {'pmap\_00x'/'gmap\_00x'} : time spent in routine {\tt
    oasis\_advance\_run} for the matrix vector multiplication for
  field x on the source/target processes.
\item {'psnd\_00x'} : time spent in routine {\tt oasis\_advance\_run}
  for sending field x (i.e. including call to {\tt mct\_waitsend} and
  {\tt mct\_isend}).
\item {'wtrn\_00x'} : time spent in routine {\tt oasis\_advance\_run}
  to write fields associated with non-instant loctrans operations to
  restart files  (see section \ref{subsec_restartdata} for details).
\item {'wrst\_00x'} : time spent in routine {\tt oasis\_advance\_run}
  to write fields to
  restart files (see section \ref{subsec_restartdata} for details).
\end{itemize}

