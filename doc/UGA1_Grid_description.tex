\newpage
\appendix
\chapter{The grid types for the transformations}
\label{subsec_gridtypes}

As described in section \ref{sec_transformations}, the different
SCRIP remappings OASIS3-MCT support different types of grids. The
characteristics of these grids are detailed here:

\begin{itemize}

\item {\tt `LR' grid}: The longitudes and the latitudes of
  2D Logically-Rectangular (LR) grid points can be described by two arrays
  {\tt longitude(i,j)} and {\tt latitude(i,j)}, where i and j
  are respectively the first and second index dimensions. Streched
  or/and rotated grids are LR grids. Note that A, B, G, L, Y, or Z
  grids are all particular cases of LR grids.

\item {\tt `U' grid}: Unstructured (U) grids do have any particular
      structure. The longitudes and the latitudes of 2D Unstructured
      grid points must be described by two arrays {\tt
      longitude(nbr\_pts,1)} and {\tt latitude(nbr\_pts,1)}, where nbr\_pts
      is the total grid size.

\item {\tt `D' grid} The Reduced (D) grid is composed of a certain
number of latitude circles, each one being divided into a varying
number of longitudinal segments. In OASIS3, the grid data (longitudes,
latitudes, etc.) must be described by arrays dimensioned {\tt
(nbr\_pts,1)}, where {\tt nbr\_pts} is the total number of grid
points. There is no overlap of the grid, and no grid point at the
equator nor at the poles. There are grid points on the Greenwich
meridian.
 
\end{itemize}



