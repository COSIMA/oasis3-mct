\newpage
\chapter{OASIS3-MCT auxiliary data files}
\label{sec_auxiliary}

OASIS3-MCT may use auxiliary data files, e.g. defining the grids of
the models being coupled, containing the field coupling restart values
or input data values, or the remapping weights and addresses.

\section{Grid data files}
\label{subsec_griddata}

With OASIS3-MCT, the grid data files {\em grids.nc, masks.nc} and {\em
  areas.nc} are required only for certain operations, i.e.  {\em grids.nc}, and {\em
  masks.nc} for {\tt SCRIPR} (see section
\ref{subsec_interp}) and {\em masks.nc} and {\em areas.nc} 
for {\tt CONSERV} (see section \ref{subsec_cooking}). These grid data files can be
created by the user before the run or can be written directly at run
time by the {\bf master process of each component model} using the
grid data definition routines (see section \ref{subsubsec_griddef}).
These routines can be used by the component models to add
grid fields to the grid files but grid fields
are {\bf never} overwritten in the grid files. These files are netCDF.

The arrays containing the grid information are dimensioned {\tt (nx, ny)},
where {\tt nx} and {\tt ny} are the grid first and second dimension.
Unstructured grids are supported by setting the ny dimension to 1
and then nx is the total number of grid points.

\begin{enumerate}

\item {\em grids.nc}: contains the model grid
  longitudes and latitudes in single or double
  precision {\tt REAL} arrays (depending on OASIS3-MCT compilation
  options). The array names must be composed of a prefix (4
  characters), given by the user in the {\it namcouple} on the second
  line of each field (see section \ref{subsec_namcouplesecond}), and
  of a suffix (4 characters); this suffix is ``.lon'' or ``.lat'' for
  respectively the grid point longitudes or latitudes.

  If the {\tt SCRIPR/CONSERV} remapping is used, longitudes and
  latitudes for the source and target grid {\bf corners} must also be
  available in the {\em grids.nc} file as arrays dimensioned {\tt
    (nx,ny,4)} or {\tt (nbr\_pts,1,4)} where {\tt 4} is the number
  of corners (in the counterclockwize sense). The names of the arrays
  must be composed of the grid prefix and the suffix ``.clo'' or
  ``.cla'' for respectively the grid corner longitudes or latitudes.
  As for the other grid information, the corners can be provided in
  {\em grids.nc} before the run by the user or directly by the model
  through specific calls (see section \ref{subsubsec_griddef}).

%  For source grids of Logically Rectangular LR type only, the
%  grid corners will however be automatically calculated and stored by
%  OASIS if they are not initially available in {\em grids.nc}\footnote{Tip: 
%  to automatically calculate the corners of a Logically Rectangular LR target
%  grid, use the corresponding reverse remapping in which the current
%  target grid becomes the source grid.}.
 
 Longitudes must be given in degrees East in the interval -360.0 to
 720.0. Latitudes must be given in degrees North in the interval -90.0
 to 90.0. Note that if some grid points overlap, it is recommended to
 define those points with the same number (e.g. 360.0 for both, not
 450.0 for one and 90.0 for the other) to ensure automatic detection
 of overlap by OASIS3-MCT. 
 
 The corners of a cell cannot be defined modulo
 360 degrees. For example, a cell located over Greenwich will have to be defined
 with corners at -1.0 deg and 1.0 deg but not with corners at 359.0 deg and 1.0 deg.
 
 Cells larger than 180.0 degrees in longitude are not supported. 
 
\item {\em masks.nc}: contains the masks for all
  component model grids in {\tt INTEGER} arrays. {\bf Be careful to note
  the historical OASIS convention: 0 -not masked i.e.
  active- or 1 -masked i.e. not active- for each grid point}. The
  array names must be composed of the grid prefix and the suffix
  ``.msk''. This file, {\em masks} or {\em masks.nc}, is mandatory.

\item {\em areas.nc}: this file contains mesh surfaces
for the component model grids in single or double precision {\tt REAL}
arrays (depending on OASIS3-MCT compilation options). The array names must be
composed of the grid prefix and the suffix ``.srf''.  The surfaces may
be given in any units but they must be all the same. This file {\em
areas.nc} is mandatory for {\tt
CONSERV/GLOBAL},{\tt /GLBPOS}, {\tt /BASBAL}, and {\tt /BASPOS}; it is not required otherwise.

\end{enumerate}

\section{Coupling restart files}
\label{subsec_restartdata}

At the beginning of a coupled run, some coupling fields may have to be
initially read from their coupling restart file on their source grid
(see the LAG concept in section \ref{subsubsec_Algoritms}). When needed, these files are also automatically
updated by the last active {\tt oasis\_put} or {\tt prism\_put\_proto}
call of the run (see section \ref{prismput}) . 
%To force the writing of the field in its
%coupling restart file, one can use the routine {\tt
%  prism\_put\_restart\_proto} (see section \ref{subsec:auxiliary}).
{\bf Warning}: the date is not written or read to/from the restart file;
therefore, the user has to make sure that the appropriate coupling restart file
is present in the working directory. The coupling restart files must
follow the NetCDF format.

%Note that all restart files have to be present in the working directory at the %beginning of the
%run even if one model is delayed with respect to the others.

The name of the coupling restart file is
given by the 6th character string on the first configuring line for
each field in the {\it namcouple} (see section
\ref{subsec_namcouplesecond}). Coupling fields coming from different
models cannot be in the same coupling restart files, but for each
model, there can be an arbitrary number of fields written in one
coupling restart file. 

In the coupling restart files, the fields must be provided on the source grid in single or double
precision REAL arrays and, as the grid data files, must be dimensioned {\tt (nx,
ny)}, where {\tt nx} and {\tt ny} are the grid first and second
dimension, except for fields given on unstructured 
for which the arrays are dimensioned {\tt (nt,1)},
where {\tt nt} is the total number of grid points.  The shape
and orientation of each restart field (and of the corresponding
coupling fields exchanged during the simulation) must be coherent with
the shape of its grid data arrays. 


\section{Input data files}
\label{subsec_inputdata}

Fields with status {\tt INPUT} in the {\it namcouple} will, at
  runtime, simply be read in from a NetCDF input file by the target
  model below the {\tt oasis\_get} or {\tt prism\_get\_proto} call, at appropriate
  times corresponding to the input period indicated by the user in the
  {\it namcouple}. 

The name of the file must be the one given on the field first
configuring line in the {\it namcouple} (see section
\ref{subsubsec_secondINPUT}). There must be one input file per {\tt
INPUT} field, containing a time sequence of the field in a single or
double precision REAL array
named with the field symbolic name in the {\it namcouple} and
dimensioned {\tt (nx,ny,time)} or {\tt (nbr\_pts,1,time)}.  The time
variable has to be an array {\tt time(time)} expressed in
``seconds since beginning of run''. The ``time'' dimension has to
be the unlimited dimension. 
%For a practical example, see the file SOALBEDO.nc in 
%{\tt oasis3/examples/toyoasis3/data}.

%subsection{Input data files}

\section{Transformation auxiliary data files}
\label{subsec_transformationdata}

\subsection{Files containing the remapping weights and addresses for {\tt
  MAPPING}}
\label{subsec_mapdata}

The mapping files to be read using the MAPPING option are consistent
with the files generated in OASIS3.3 or OASIS3-MCT using the SCRIP
library (see \ref{subsec_auxilscripr} below). The files are netcdf and the
key fields are num\_links (the number of weights for each grid pair), src\_grid\_size
(the 2d size of the source grid), dst\_grid\_size
(the 2d size of the destination grid), num\_wgts (the
total number of weights), remap\_matrix (the weights), dst\_address
(the global destination index for the weight), src\_address (the global
source index for the weight).

\subsection{Auxiliary data files for {\tt SCRIPR}}
\label{subsec_auxilscripr}

The NetCDF files containing the weights and addresses for the {\tt
  SCRIPR} remappings (see section \ref{subsec_interp})  are
  automatically generated at runtime by OASIS3-MCT. Their structure is
  described in detail in section 2.2.3 of the SCRIP documentation available
  in {\tt oasis3-mct/doc/SCRIPusers.pdf}.  In particular, they are netCDF
  files with the following one fields
\begin{itemize}
\item src\_grid\_size is a scalar integer indicating the total number of
  global grid cells for the source grid.  This field
  in a netCDF dimension.
\item dst\_grid\_size is a scalar integer indicating the total number of
  global grid cells for the destination (or target) grid.  This field
  in a netCDF dimension.
\item num\_links is a scalar integer indicating the total number of associated
  grid pairs in the file.  This is typically a large number.  This field
  is a netCDF dimension.
\item num\_wgts is a scalar integer indicating the number of weights per
  associated grid pair.  For first order mapping, this is 1.  This field
  in a netCDF dimension.
\item src\_address is a one dimensional array of size num\_links.  It contains
  the integer source address associated with each weight.  This field is a
  netCDF variable.
\item dst\_address is a one dimensional array of size num\_links.  It contains
  the integer destination address associated with each weight.  This field is a
  netCDF variable.
\item remap\_matrix is a two dimensional array of size num\_links, num\_wgts.  It contains
  the real weight value(s) associated with the source and destination address. 
  This field is a netCDF variable.
\end{itemize}

